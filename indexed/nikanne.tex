\documentclass[output=paper]{langsci/langscibook} 
\author{Urpo Nikanne \affiliation{Åbo Akademi University, Finland}}
\title{Finite sentences in Finnish: Word order, morphology, and information structure} 
% \epigram{Change epigram}
\abstract{According to \citet{HolmbergEtAl1993} the finite sentence of Finnish is a structure with 2--6 functional heads. In this article, the theory is developed further and the functional heads are reanalyzed. The functional categories are divided into two categories: (i) lexical categories Neg, Aux, V, and C; (ii) morphological categories: AgrS, T, and Ptc. These categories are in separate tiers, and the tiers are linked to each other. Both lexical and morphological categories are hierarchically organized, and the linking between the tiers follows these hierarchies. The result of the reanalysis is a system that does not involve movement nor a complicated constituent structure of functional categories even though the desired properties of the previous analysis remain.}

\ChapterDOI{10.5281/zenodo.1117710}
\maketitle

\begin{document}
% [Warning: Draw object ignored][Warning: Draw object ignored][Warning: Draw object ignored][Warning: Draw object ignored][Warning: Draw object ignored][Warning: Draw object ignored][Warning: Draw object ignored][Warning: Draw object ignored][Warning: Draw object ignored][Warning: Draw object ignored][Warning: Draw object ignored][Warning: Draw object ignored][Warning: Draw object ignored][Warning: Draw object ignored][Warning: Draw object ignored][Warning: Draw object ignored][Warning: Draw object ignored][Warning: Draw object ignored][Warning: Draw object ignored][Warning: Draw object ignored][Warning: Draw object ignored][Warning: Draw object ignored][Warning: Draw object ignored][Warning: Draw object ignored][Warning: Draw object ignored]

% [Warning: Draw object ignored][Warning: Draw object ignored][Warning: Draw object ignored][Warning: Draw object ignored][Warning: Draw object ignored][Warning: Draw object ignored][Warning: Draw object ignored][Warning: Draw object ignored][Warning: Draw object ignored][Warning: Draw object ignored][Warning: Draw object ignored][Warning: Draw object ignored][Warning: Draw object ignored][Warning: Draw object ignored][Warning: Draw object ignored][Warning: Draw object ignored][Warning: Draw object ignored][Warning: Draw object ignored][Warning: Draw object ignored][Warning: Draw object ignored][Warning: Draw object ignored]\textbf{Finite\is{finiteness} sentences} \textbf{in \ili{Finnish}: Word order, \isi{morphology}, and information~structure}


 
%%please move the includegraphics inside the {figure} environment
%%\includegraphics[width=\textwidth]{a4Nikanne-img1}

 
%%please move the includegraphics inside the {figure} environment
%%\includegraphics[width=\textwidth]{a4Nikanne-img2}

 
%%please move the includegraphics inside the {figure} environment
%%\includegraphics[width=\textwidth]{a4Nikanne-img3}

 
%%please move the includegraphics inside the {figure} environment
%%\includegraphics[width=\textwidth]{a4Nikanne-img4}

 
%%please move the includegraphics inside the {figure} environment
%%\includegraphics[width=\textwidth]{a4Nikanne-img5}

 
%%please move the includegraphics inside the {figure} environment
%%\includegraphics[width=\textwidth]{a4Nikanne-img6}

 
%%please move the includegraphics inside the {figure} environment
%%\includegraphics[width=\textwidth]{a4Nikanne-img7}

 
%%please move the includegraphics inside the {figure} environment
%%\includegraphics[width=\textwidth]{a4Nikanne-img8}

 
%%please move the includegraphics inside the {figure} environment
%%\includegraphics[width=\textwidth]{a4Nikanne-img9}

 
%%please move the includegraphics inside the {figure} environment
%%\includegraphics[width=\textwidth]{a4Nikanne-img10}

 
%%please move the includegraphics inside the {figure} environment
%%\includegraphics[width=\textwidth]{a4Nikanne-img11}

 
%%please move the includegraphics inside the {figure} environment
%%\includegraphics[width=\textwidth]{a4Nikanne-img12}

 
%%please move the includegraphics inside the {figure} environment
%%\includegraphics[width=\textwidth]{a4Nikanne-img13}

 
%%please move the includegraphics inside the {figure} environment
%%\includegraphics[width=\textwidth]{a4Nikanne-img14}

 
%%please move the includegraphics inside the {figure} environment
%%\includegraphics[width=\textwidth]{a4Nikanne-img15}

 
%%please move the includegraphics inside the {figure} environment
%%\includegraphics[width=\textwidth]{a4Nikanne-img16}

 
%%please move the includegraphics inside the {figure} environment
%%\includegraphics[width=\textwidth]{a4Nikanne-img17}

 
%%please move the includegraphics inside the {figure} environment
%%\includegraphics[width=\textwidth]{a4Nikanne-img18}

 
%%please move the includegraphics inside the {figure} environment
%%\includegraphics[width=\textwidth]{a4Nikanne-img19}

 
%%please move the includegraphics inside the {figure} environment
%%\includegraphics[width=\textwidth]{a4Nikanne-img20}

 
%%please move the includegraphics inside the {figure} environment
%%\includegraphics[width=\textwidth]{a4Nikanne-img21}

 
%%please move the includegraphics inside the {figure} environment
%%\includegraphics[width=\textwidth]{a4Nikanne-img22}

 
%%please move the includegraphics inside the {figure} environment
%%\includegraphics[width=\textwidth]{a4Nikanne-img23}

 
%%please move the includegraphics inside the {figure} environment
%%\includegraphics[width=\textwidth]{a4Nikanne-img24}

 
%%please move the includegraphics inside the {figure} environment
%%\includegraphics[width=\textwidth]{a4Nikanne-img25}

 
%%please move the includegraphics inside the {figure} environment
%%\includegraphics[width=\textwidth]{a4Nikanne-img26}

 
%%please move the includegraphics inside the {figure} environment
%%\includegraphics[width=\textwidth]{a4Nikanne-img27}

 
%%please move the includegraphics inside the {figure} environment
%%\includegraphics[width=\textwidth]{a4Nikanne-img28}

 
%%please move the includegraphics inside the {figure} environment
%%\includegraphics[width=\textwidth]{a4Nikanne-img29}

 
%%please move the includegraphics inside the {figure} environment
%%\includegraphics[width=\textwidth]{a4Nikanne-img30}

 
%%please move the includegraphics inside the {figure} environment
%%\includegraphics[width=\textwidth]{a4Nikanne-img31}

 
%%please move the includegraphics inside the {figure} environment
%%\includegraphics[width=\textwidth]{a4Nikanne-img32}




\section{Introduction} %1
Anders Holmberg and his colleagues came up with an analysis of the \ili{Finnish} finite\is{finiteness} sentence in the early 1990s \citep{HolmbergEtAl1993}. The analysis was based on the so-called incorporation theory in which finite\is{finiteness} verb \isi{morphology} assumed to be a result of a head-to-head movement of the verb: the verb was raised from one functional head (e.g. \isi{tense}, mood, subject \isi{agreement}, etc.) to another, and the functional heads were attached to the verb. The finite\is{finiteness} verb \isi{morphology} was therefore a mirror image of the syntactic structure \citep{Pollock1989,Baker1988incorporation,Chomsky1995}, etc.). In the \isi{Minimalist}\is{Minimalist/minimalist} Theory \citep{Chomsky1995}, the basic idea has remained the same, but, instead of picking up affixes along its head-to-head movement upwards in the syntactic structure, the verb checks that the \isi{morphological} features it is carrying are compatible with the features in the syntactic tree. 

Traditionally, the word order of \ili{Finnish} has been characterized as free.  According to \citet{Vilkuna1989}, the word order in \ili{Finnish} finite\is{finiteness} sentences is constraint by information structure. There are designated word order positions for the topic of the sentence and a phrase that carries a contrastive focus.  \citet{HolmbergNikanne1994,HolmbergNikanne2002,HolmbergNikanne2008} have shown that the word (verb or \isi{negation} word) carrying the subject \isi{agreement} suffixes has its own designated position in the finite\is{finiteness} sentence word order.  

The theory presented by Anders Holmberg and his colleagues (\citealt{HolmbergEtAl1993,HolmbergNikanne2002}, etc.) is so far the most advanced model of the finite\is{finiteness} sentence of \ili{Finnish}. It is able to combine the \ili{Finnish} finite\is{finiteness} sentence \isi{morphology} and syntax in an elegant way.  As linguists, however, it is our duty always to seek for new ways to see language and try to come up with theories that can replace the old ones. That is the purpose of this article. 

At first, I explain how the theory by Holmberg and his colleagues works. Then, I discuss how it can be improved. After that, I suggest improvements that are based on a “micro-modular” theory of language. The micro-modular theory, Tiernet, is a version of Conceptual Semantics\is{semantics} (\citealt{Jackendoff1983,Jackendoff1990,Jackendoff2002}, etc.) explained and motivated in detail in \citet{Nikannefc}.

\section{Finnish finite sentence: the basic facts}%2

The \ili{Finnish} finite\is{finiteness} verb has the \isi{morphological} structure given in \tabref{ex:nikanne:1}.

\begin{table}
\caption{Morphological structure of the Finnish verb}
\label{ex:nikanne:1}
\small
\begin{tabularx}{\textwidth}{llllQ}
\lsptoprule
{verb stem}  & {(+ \isi{passive})} & {+ \isi{tense}/mood}  & {+ subject agreement} & \\\midrule
\textit{istu} [‘sit’] &  & + \textit{i} & + \textit{mme} [{\scshape\oldstylenums{1}pl subj. agr}] & ‘we sat down’\\
\textit{istu} [‘sit’] &  & + \textit{isi} & + \textit{mme} [{\scshape\oldstylenums{1}pl subj. agr}] & ‘we would sit down'\\
\textit{istu}  [‘sit’] & + \textit{tt}  [{\scshape passive}] & + \textit{i} [{\scshape past}] & + \textit{in} [{\scshape pass subj. agr.}] & ‘it was sat down’\\
\textit{istu} [‘sit’] & + \textit{tta} [{\scshape passive}] & + \textit{isi} [{\scshape conditional}] & + \textit{in} [{\scshape pass subj. agr.}] & ‘it would have been sat down’\\
\lspbottomrule
\end{tabularx}
\end{table}%%\todo{this should probably be a table, not an example}


There are two things in the \ili{Finnish} finite\is{finiteness} \isi{morphology} that might be confusing: (i) the \isi{tense} and mood markers are in complementary distribution; and (ii) in addition to the \isi{passive} marker \textit{ttA} the \isi{passive} form has an AgrS suffix -\textit{Vn} when the \isi{negation} word or the auxiliary\is{Auxiliary} are not present. 

In addition to the predicate verb, there are two more words that may carry finite\is{finiteness} affixes. The auxiliary\is{Auxiliary} \textit{ole}{}- ‘be’ in the perfect and pluperfect \isi{tenses} and the \isi{negation} word \textit{e(i)}{}- ‘not’ in negated sentences. In the perfect and pluperfect \isi{tenses}, the predicate verb is in the perfect \isi{participle} form. Here is an example of the paradigm (the finite\is{finiteness} morphemes are separated with a dash):
% \todo[inline]{We changed the representation here from table to example form}

\newpage 
\ea\label{ex:nikanne:paradigm1}
\ea Present active (3rd \isi{person} plural)\\
\gll  Tytöt istu-vat tuolilla. \\
girls sit-\oldstylenums{3}\textsc{pl} chair.\textsc{ade}\\
\glt  ‘The girls sit on the chair.’
\ex Present \isi{passive}\\
\gll  Istu-ta-an tuolilla. \\
sit-\textsc{pass}-\textsc{pass} chair.\textsc{ade}\\
\glt ‘It is sat on the chair.'\\
     ‘One sits on the chair.'\\
\z
\z

\ea\label{ex:nikanne:paradigm2}
\ea Simple past active (3rd \isi{person} plural)\\
\gll  Tytöt istu-i-vat tuolilla. \\
girls sit-\textsc{past}-\oldstylenums{3}\textsc{pl} chair.\textsc{ade}\\
\glt ‘The girls sat on the chair.'
\ex Simple past \isi{passive}\\
\gll  Istu-tt-i-in tuolilla. \\
sit-\textsc{pass}-\textsc{past}-\textsc{pass} chair.\textsc{ade}\\
\glt ‘It was sat on the chair.’\\
 ‘One sat on the chair.'\\
\z
\z

\ea\label{ex:nikanne:paragidm3}
\ea Conditional present active (3rd \isi{person} plural)\\
\gll  Tytöt istu-isi-vat tuolilla. \\
girls sit-\textsc{cond}-\oldstylenums{3}\textsc{pl} chair.\textsc{ade}\\
\glt ‘The girls would sit on the chair.’
\ex Conditional present \isi{passive}\\
\gll  Istu-tta-isi-in tuolilla. \\
 sit-\textsc{pass}-\textsc{cond}-\textsc{pass} chair.\textsc{ade}\\
\glt  ‘One would sit on the chair.’\\
\z
\z

\ea\label{ex:nikanne:paradigm4}
\ea Perfect \isi{tense} active (3rd \isi{person} plural)\\
\gll  Tytöt o-vat istu-nee-t tuolilla. \\
girls be-\oldstylenums{3}\textsc{pl} sit-\textsc{ptc}-\oldstylenums{3}\textsc{pl} chair.\textsc{ade}\\
\glt ‘The girls have sat on the chair.’
\ex Perfect \isi{tense} \isi{passive}\\
\gll  On istu-tt-u tuolilla. \\
be-\oldstylenums{3}\textsc{sg} sit-\textsc{pass}-\textsc{ptc} chair.\textsc{ade}\\
\glt  `It has been sat on the chair.’\\ ‘One has sat on the chair.'\\
\z
\z

\ea\label{ex:nikanne:paradigm5}
\ea Pluperfect \isi{tense} active (3rd \isi{person} plural)\\
\gll  Tytöt ol-i-vat istu-nee-t tuolilla. \\
girls be-\textsc{past}-\oldstylenums{3}\textsc{pl} sit-\textsc{ptc}-\oldstylenums{3}\textsc{pl} chair-\textsc{ade}\\
\glt  ‘The girls had sat on the chair.’
\ex Pluperfect \isi{tense} \isi{passive}\\
\gll  Ol-i istu-tt-u tuolilla. \\
 be-\textsc{past}.\oldstylenums{3}\textsc{sg} sit-\textsc{pass}-\textsc{ptc} chair.\textsc{ade}\\
\glt ‘It was sat on the chair.’\\
 ‘One had sat on the chair.’\\
\z
\z

\ea\label{ex:nikanne:paradigm6}
\ea Negative past active (3rd \isi{person} plural)\\
\gll  Tytöt ei-vät istu-nee-t tuolilla. \\
     girls not-\oldstylenums{3}\textsc{pl} sit-\textsc{ptc}-\textsc{pl} chair.\textsc{ade}\\
 \glt ‘The girls did not sit on the chair.'
\ex Negative past \isi{passive}\\
\gll  Ei istu-tt-u tuolilla. \\
not.\oldstylenums{3}\textsc{sg} sit-\textsc{pass}-\textsc{ptc} chair.\textsc{ade}\\
\glt ‘It was not sat on the chair.’\\
\z
\z

\ea\label{ex:nikanne:paradigm7}
\ea Negative perfect \isi{tense} active (3rd \isi{person} plural)\\
\gll  Tytöt ei-vät ole istu-neet tuolilla. \\
    girls not-\oldstylenums{3}\textsc{pl} be sit-\textsc{ptc}.\textsc{pl} chair.\textsc{ade}\\
\glt  ‘The girls have not sat on the chair.’
\ex Negative perfect \isi{tense} \isi{passive}\\
\gll  Ei ole istu-tt-u tuolilla. \\
 not-\oldstylenums{3}\textsc{sg} be sit-\textsc{pass}-\textsc{ptc} chair.\textsc{ade}\\
\glt ‘It has not been sat on the chair.’\\
\z
\z

\ea\label{ex:nikanne:paradigm8}
\ea Negative pluperfect \isi{tense}:\\
\gll  Tytöt ei-vät ol-leet istu-neet tuolilla. \\
  girls not-\oldstylenums{3}\textsc{pl} be-\textsc{ptc}-\textsc{pl} sit-\textsc{ptc}-\textsc{pl} chair.\textsc{ade}\\
\glt  ‘The girls had not sat on the chair.’
\ex Negative pluperfect \isi{tense} \isi{passive}\\
\gll  Ei ol-lut istu-tt-u tuolilla. \\
 not.\oldstylenums{3}\textsc{sg} be-\textsc{ptc} sit-\textsc{pass}-\textsc{ptc} chair.\textsc{ade}\\
\glt ‘It had not been sat on the chair.’\\
\z
\z
 

\section{Anders Holmberg’s et al. theory of Finnish finite sentence}\label{sec:nikanne:3}  %3

According to \citet{HolmbergNikanne2002} (based on the analysis of \citealt{HolmbergEtAl1993}), the \ili{Finnish} finite\is{finiteness} sentence in its fullest possible form is as in \REF{ex:nikanne:3}. The category F (= finite\is{finiteness}) in  \citegen{HolmbergNikanne2002} analysis is marked in \REF{ex:nikanne:3} as AgrS in order to show the relation between the \isi{morphology} and syntactic structure. (This is not a radical difference; see the discussion of the node F instead of AgrS in \citealt{HolmbergNikanne2002}.)

\newpage 
\ea%3
    \label{ex:nikanne:3}
    The maximal structure of the \ili{Finnish} finite\is{finiteness} sentence\\
    \begin{forest}
     [CP\is{complementizer} [Spec] [C' [C] [AgrSP [Spec] [AgrS' [AgrS] [NegP [Spec] [Neg\is{negation}' [Neg\is{negation}] [TP [Spec] [T' [T] [AuxP [Spec] [Aux\is{Auxiliary}' [Aux\is{Auxiliary}] [PtcP [Spec] [Ptc' [Ptc] [VP [Spec] [V' [V] [{[}...{]}] ] ] ] ] ] ] ] ] ] ] ] ] ] ]
    \end{forest}
\z

``C'' stands for Complementizer\is{complementizer},  ``AgrS'' for subject \isi{agreement} (i.e. \isi{person} \oldstylenums{1}\textsc{sg}, \oldstylenums{2}\textsc{sg}, \oldstylenums{3}\textsc{sg}, \oldstylenums{1}\textsc{pl}, \oldstylenums{2}\textsc{pl}, \oldstylenums{3}\textsc{pl}, and the  \isi{passive} \isi{agreement} ending),  ``Neg\is{negation}'' for \isi{negation},  ``T'' for tempus (i.e. present, past) and in \ili{Finnish} also modus (i.e. conditional, potential, imperative), ``Aux\is{Auxiliary}'' for auxiliary\is{Auxiliary} verb (olla `be’),  ``Ptc'' for participial (past, present), and ``Spec'' for specifier. NB:  ``Constituency'' is marked according to the convention introduced by \citet{Petrova2011}: the “ball” at the end of the line indicates the end in which the dominated element (the daughter) is. The benefit of this convention is that it does not require that the mother phrase is above the daughter. 

The \isi{passive} marker \textit{ttA} is base generated in the Spec(VP), the assumed original position for the subject. In standard \ili{Finnish}, the \isi{passive} voice is in a complementary distribution with an overt subject.

Only AgrS and T are obligatory. Those are the morphemes that are in an affirmative present or simple past \isi{tense} forms (see examples \ref{ex:nikanne:paradigm1}--\ref{ex:nikanne:paradigm8} above).

The D-structure of the sentence \textit{Istu-i-mme tuolilla} [sit-\textsc{past}-\oldstylenums{1}\textsc{pl} chair.\textsc{ade}] ‘We sat on the chair’ is given in \REF{ex:nikanne:4}. The information on the finite\is{finiteness} sentence \isi{morphology} is in the functional positions, and the subject NP is in the Spec(VP) position; note that then both arguments of the verb \textit{istu-} ‘sit’ are in the maximal \isi{projection} whose head the verb is.  

The derivation from D-structure to S-structure is illustrated in \REF{ex:nikanne:4}: The verb undergoes a head movement from the head of the VP position (V) via the head of the Tense\is{tense}/Mood phrase (T) to the head of the AgrSP position (AgrS). The \isi{morphological} structure is a mirror image of the head-to-head movement chain. The subject NP is assumed to be base generated in the Spec(VP) position. As the subject NP is in the nominative case and the AgrS feature (\oldstylenums{1}\textsc{pl}) is compatible with the \isi{person} and \isi{number} of the subject NP, the sentence is grammatical. The verb and the subject NP leave behind \isi{traces} in the positions in which they land on the way to their S-structure positions. 

\ea%4
\label{ex:nikanne:4} 
\begin{forest}
[AgrSP [Spec] [AgrS' [AgrS\\\oldstylenums{1}\textsc{pl}\\\textit{mme},base=top,align=center] [TP [Spec] [T' [T\\\textsc{past}\\\textit{i},base=top,align=center] [VP [Spec\\\textit{me}\\`we',base=top,align=center] [V' [V\\\textit{istu-}\\`sit-',base=top,align=center] [PP\\\textit{tuolilla}\\`chair.\textsc{ade}',base=top,align=center] ] ] ] ] ] ] 
\end{forest}
\z

\largerpage
The S-structure is given in \REF{ex:nikanne:5}, with the solid arrows indicating the movements. The finite\is{finiteness} \isi{morphology}, as it appears in the surface structure, is given in the grey box, and the dashed arrow points to syntactic position of the inflected element. (In \ref{ex:nikanne:6}--\ref{ex:nikanne:8}, in order to avoid too many arrows, only the movements of V and Aux\is{Auxiliary} are shown.)

\ea%5
\label{ex:nikanne:5} 
\begin{forest}
[AgrSP [Spec,name=Spec] [AgrS' [AgrS\\\textsc{v}+\textsc{past}+\oldstylenums{1}\textsc{pl},name=AgrS] [TP [Spec] [T' [T\\t,name=t,base=top,align=center] [VP [Spec\\\textit{me}\\`we',base=top, ,name=SpecWe,align=center] [V' [V\\\textit{istu-}\\`sit-',base=top,align=center,name=V] [PP\\\textit{tuolilla}\\`chair.\textsc{ade}',base=top,align=center] ] ] ] ] ] ]
\path[-{Stealth[]}] (V) edge [bend left=90] (t);
\path[-{Stealth[]}] (SpecWe) edge [bend left=90] (Spec);
\path[-{Stealth[]}] (t.200) edge [bend left] (AgrS);
\node[left=3cm of V,draw,fill=gray!20,align=left] (istu) {istu-i-mme\\sit-\textsc{past}-\oldstylenums{1}\textsc{pl}};
\draw[dashed,-{Straight Barb[]}] (istu) -- (AgrS);
\end{forest}
\z

\largerpage[2]
The \isi{passive} marker is base generated in the specifier position of the VP. This is the position in which the subject argument is supposed to be base generated. The surface structure is derived as follows:

\ea%6
\label{ex:nikanne:6}
\begin{forest}
[AgrSP [Spec] [AgrS' [AgrS\\\textsc{pass}\\\textit{Vn},base=top,align=center,name=AgrS] [TP [Spec] [T' [T\\\textsc{cond}\\\textit{isi},name=cond,base=top,align=center] [VP [Spec\\\textsc{pass}\\\textit{ttA},name=tta] [V' [V\\\textit{istu-}\\`sit-',base=top,align=center,name=istu] [PP\\\textit{tuolilla}\\`chair.\textsc{ade}'] ] ] ] ] ] ] 
\path[-{Stealth[]}] (istu.south) edge [bend left,out=90] (tta);
\path[-{Stealth[]}] (tta.220) edge [bend left,out=90] (cond);
\path[-{Stealth[]}] (cond.200) edge [bend left,out=90] (AgrS.south);
\node[left=3cm of istu,draw,fill=gray!20,align=left] (istu2) {istu-tta-isi-in\\sit-\textsc{past}-\textsc{cond}-\textsc{pass}};
\draw[dashed,-{Straight Barb[]}] (istu2) -- (AgrS);
\end{forest}
\z

\newpage If the auxiliary\is{Auxiliary} \textit{olla} `be’ is present, i.e. in the perfect or pluperfect \isi{tense}, the Aux\is{Auxiliary} undergoes a head movement from the head of the auxiliary\is{Auxiliary} phrase position (Aux) to AgrS. Then, the verb moves from V to the head of the participial phrase position (Ptc). 

\ea%7
\label{ex:nikanne:7}
\begin{forest}
 [AgrSP [Spec] [AgrS' [AgrS\\\oldstylenums{1}\textsc{pl},base=top,align=center,name=AgrS] [TP [Spec] [T' [T\\\textsc{cond}\\\textit{isi},align=center,base=top,name=cond] [AuxP [Spec] [Aux\is{Auxiliary}' [Aux\is{Auxiliary}\\ole-,name=ole] [PtcP [Spec] [Ptc' [Ptc\\neet,base=top,align=center,name=neet] [VP [Spec] [V' [V\\\textit{istu-}\\`sit',base=top,align=center,name=istu] [PP\\\textit{tuolilla}\\`chair.\textsc{ade}',base=top,align=center] ] ] ] ] ] ] ] ] ] ]
\path[-{Stealth[]}] (istu) edge [bend left,out=90] (neet);
\path[-{Stealth[]}] (ole) edge [bend left=90] (cond);
\path[-{Stealth[]}] (cond) edge [bend left,out=90] (AgrS);
\node[left=2.5cm of istu,baseline,draw,fill=gray!20,align=left] (istu2) {istu-neet\\sit-\textsc{ptc(pl)}};
\node[left=.1cm of istu2,baseline,draw,fill=gray!20,align=left] (istu3) {ol-isi-mee\\be-\textsc{cond}-\oldstylenums{1}\textsc{pl}};
\draw[dashed,-{Straight Barb[]}] (istu2) -- (neet);
\draw[dashed,-{Straight Barb[]}] (istu3.150) -- (AgrS.250);
\end{forest}
\z 

If the \isi{negation} is present, the \isi{negation} word \textit{ei} undergoes a movement from the head of the \isi{negation} phrase (Neg\is{negation}) to AgrS. Then, the auxiliary\is{Auxiliary} moves from Aux\is{Auxiliary} to T and the predicate verb from V to Ptc.

\ea%8
\label{ex:nikanne:8}
\begin{forest}
[AgrSP [Spec] [AgrS' [AgrS\\\oldstylenums{1}\textsc{pl}\\\textit{mme},name=AgrS] [NegP [Spec] [Neg\is{negation}' [Neg\is{negation}\\\textit{e-},base=top,align=center,name=Neg] [TP [Spec] [T' [T\\\textsc{cond}\\\textit{isi},name=isi] [AuxP [Spec] [Aux\is{Auxiliary}' [Aux\is{Auxiliary}\\\textit{ole-},base=top,align=center,name=Aux] [PtcP [Spec] [Ptc' [Ptc\\neet,base=top,align=center,name=neet] [VP [Spec] [V' [V\\\textit{istu-}\\`sit',name=istu,base=top,align=center] [PP\\\textit{tuolilla}\\`chair.\textsc{ade}'] ] ] ] ] ] ] ] ] ] ] ] ]
\path[-{Stealth[]}] (istu) edge [bend left,out=90] (neet);
\path[-{Stealth[]}] (Aux) edge [bend left,out=90] (isi);
\path[-{Stealth[]}] (Neg) edge [bend left,out=90] (AgrS);
\node[left=2.5cm of istu,baseline,draw,fill=gray!20,align=left] (istu2) {istu-neet\\sit-\textsc{ptc(pl)}};
\node[left=.1cm of istu2,baseline,draw,fill=gray!20,align=left] (istu3) {ol-isi\\be-\textsc{cond}};
\node[left=.1cm of istu3,baseline,draw,fill=gray!20,align=left] (istu4) {e-mme\\neg-\oldstylenums{1}\textsc{pl}};
\draw[dashed,-{Straight Barb[]}] (istu2) -- (neet);
\draw[dashed,-{Straight Barb[]}] (istu3) -- (isi);
\draw[dashed,-{Straight Barb[]}] (istu4) -- (AgrS);
\end{forest}
\z

The Complementizer\is{complementizer} phrase (CP\is{complementizer}) is understood traditionally as a \isi{projection} of a Complementizer\is{complementizer} word, such as a subordinating conjunction \REF{ex:nikanne:9a}, wh-word \REF{ex:nikanne:9b}, or contrastively focused element (\ref{ex:nikanne:9}c,d).  According to Maria \citeauthor{Vilkuna1989} (\citeyear{Vilkuna1989} etc.), the two initial positions of the \ili{Finnish} finite\is{finiteness} sentence are reserved for a contrastively focused element (the first position) and the topic of the sentence (the second position). The topic of the sentence can be the first element if there is no contrastively focused element present. According to \citet{HolmbergNikanne1994}, the contrast position is Spec(CP\is{complementizer}) position and the topic position is Spec(AgrSP).


\ea\label{ex:nikanne:9}
\begin{forest} %for tree={fit=band}
[CP\is{complementizer} [Spec [millä\textsubscript{i},tier=word,edge=dashed] ] [C' [C] [AgrSP [Spec [tytöt,tier=word,edge=dashed]] [AgrS' [AgrS [eivät,tier=word,edge=dashed]] [TP [Spec] [T' [T [olisi,tier=word,edge=dashed]] [PtcP [Spec] [Ptc' [Ptc [istuneet,tier=word,edge=dashed]] [{[}...{]}] ] ] ] ] ] ] ] ] 
\end{forest}
\begin{xlista}
\ex  \label{ex:nikanne:9a}
\gll   \tikzmark{millae}{millä\textsubscript{i}}                  \tikzmark{tytoet}tytöt   \tikzmark{eivaet}eivät              \tikzmark{olisi}olisi       \tikzmark{istuneet}istuneet                   t\textsubscript{i}?’\\
what.\textsc{ade}   girl.\textsc{pl}(\textsc{nom}) not.\oldstylenums{3}\textsc{pl}         be.\textsc{cond}        sit.\textsc{ptc}.\textsc{pl} \\
\glt  ‘What would the girls not have sat on?’ \\
({\scshape wh-word as adverbial}; \textsc{topic} ‘girls’)

\ex \label{ex:nikanne:9b}
\glll Spec(CP\is{complementizer})  Spec(AgrSP)   AgrS            T       Ptc\\
ketkä\textsubscript{i}     t\textsubscript{i}   eivät   olisi   istuneet   tuolilla?\\
  who.\textsc{pl}(\textsc{nom}) \textit{t} not.\oldstylenums{3}\textsc{pl} be.\textsc{cond} sit.\textsc{ptc}.\textsc{pl}  chair.\textsc{ade}\\
\glt  ‘Who(PL) would not have sat on the chair?’ 

({\scshape wh-word as subject}; \textsc{topic} ‘who’)

\ex \label{ex:nikanne:9c}
\glll Spec(CP\is{complementizer})  Spec(AgrSP)   AgrS            T       Ptc\\
tuolilla\textsubscript{i}  tytöt   eivät   olisi   istuneet   t\textsubscript{i}.\\
chair.\textsc{ade}  girl.\textsc{pl}(\textsc{nom}) not.\oldstylenums{3}\textsc{pl} be.\textsc{cond} sit.\textsc{ptc}.\textsc{pl} \textit{t}\\
\glt ‘It is the chair that the girls would not have sat on.’ 

(\textsc{contrastive focus on} ‘on the chair’; \textsc{topic}: ‘girls’)

\largerpage[2]
\ex \label{ex:nikanne:9d}
\glll   Spec(CP\is{complementizer})  Spec(AgrSP)   AgrS            T       Ptc\\
        eivät\textsubscript{i}  tytöt     t\textsubscript{i}  olisi   istuneet   tuolilla.\\
        not.\oldstylenums{3}\textsc{pl}  girl.\textsc{pl}(\textsc{nom})  \textit{t}   be.\textsc{cond}   sit.\textsc{ptc}.\textsc{pl} chair.\textsc{ade}\\
\glt ‘It is not the case that the girls would have sat on the chair.’ 


({\scshape focus on negation}; \textsc{topic}: ‘girls’)
\end{xlista}
\z

Similar models based on functional categories have been suggested for many other languages besides \ili{Finnish}, e.g. \ili{French} (starting \citealt{Pollock1989}), \ili{Swedish} (\citealt{HolmbergPlatzack1995}, \ili{Italian} (e.g. \citealt{Cinque1999}), etc. The suggested sentence structures are very similar to that proposed for \ili{Finnish}. The differences suggested for different languages have to do with the exact set and the mutual order of the categories.

\section{What can be done better?} %4

The model of finite\is{finiteness} sentence based on functional categories works well, and it has without any doubt been the most advanced model of the \ili{Finnish} finite\is{finiteness} sentence so far. However, there is always room for progress. In the sections that follow, I will show that the benefits of the theory by Holmberg et al. can be developed into a simpler theory that even better shows the relationship between finite\is{finiteness} sentence syntax and \isi{morphology}.

The two areas that need further development are the theory of constituents and keeping \isi{morphological} and lexical categories apart from each other:

\begin{description}\item[Constituents:] Traditionally, a constituent is defined as a unit that moves as a whole, is deleted as a whole, etc. In addition, in the X’-theory, a constituent is a \isi{projection} of its head. This definition fits well with the good old-fashioned constituents like NP, PP, AP, and AdvP. The constituents headed by functional categories are much more abstract and they have been introduced to the theory mostly for theory internal reasons. The functional categories C and I and their respective projections CP\is{complementizer} and IP in \citet{Chomsky1986} enabled the X’-theory cover the sentence structure in its entirety. Before that, the sentence (S) was the only constituent that did not have head and an X’-structure. These new “functional” constituents differ from the old-fashioned constituents, which are based on lexical categories.  As the door was open for abstract functional constituents, they have been assumed to play a role even in non-finite\is{finiteness} categories, such as NPs (or DPs). The development has led to a more and more abstract syntax, and at the same time, the idea of constituency has shifted further and further away from its original definition, particularly when it comes to the sentence level constituents headed by functional categories C, I, and the categories suggested to be parts of I (see the analysis of \ili{Finnish} in \sectref{sec:nikanne:3} above). The theory of syntax should make a difference between the old fashioned constituents and functional categories as they are two different things, at least in the finite\is{finiteness} sentence. 

\item[Separating lexical and \isi{morphological} categories:] In \ili{Finnish}, for instance, the categories Neg\is{negation}, Aux\is{Auxiliary}, and V must always be raised from their original positions, and they never appear without \isi{morphological}\largerpage[2] suffices AgrS, T or Ptc. The categories AgrS, T or Ptc on the other hand cannot appear alone. The “mirror image” effect is explained but it is difficult to justify a complicated model of constituent structure in which the head nodes of the lexical categories must always be moved out of their projections (constituents which they are head words of) and at the same time there are constituents headed by heads that never appear alone without a lexical category.\end{description}
                                                                                                                                                                                                                                                                                                                                                                                                                                                                                                                                                                                                                                                                                             

The expansion of the \isi{number} of functional categories may be a consequence of aiming at a \isi{universal} description of grammar. A \isi{universal} (or cross-linguistically relevant) description of the grammar requires that the overall systems of world’s languages are described in a comparable manner, not that for instance each part of the grammar, e.g. finite\is{finiteness} sentences, must be assumed to have the same underlying structure in all languages. Thus, even if we can argue for a category, feature or element in one language, we do not need to generalize the same analysis to all languages. In mainstream generative grammar, syntactic constituent structure has been the most important part of grammar, and many phenomena have been analyzed as syntactic. That leads to, as it seems to me, unnecessarily, complicated syntactic constituent analyses of constituent structure. (For arguments against unnecessarily abstract syntax in mainstream generative grammar, see also \citealt{CulicoverJackendoff2005}.)

One motivation in generative grammar for analysing finite\is{finiteness} sentence as a constituent tree has been that the grammatical functions subject and object as well as assigning grammatical cases can be defined as positions in the constituent tree. Two most important morpho-syntactic feature of the grammatical subject is that the finite\is{finiteness} predicate verb agrees with the nominative subject in \isi{person} and \isi{number}. In \ili{Finnish}, the word order may vary because the information structure is marked in the word order (see \citealt{Vilkuna1989}), and still the predicate verb of a finite\is{finiteness} sentence agrees with the (nominative) subject, no matter where the subject is located. For instance in \REF{ex:nikanne:10}, the verb \textit{istua} `sit’ agrees with the subject \textit{tytöt} [girl.\textsc{pl}.\textsc{nom}] despite the word order (S = subject, V = predicate verb, X = object or adverbial): 

\ea
\begin{xlist}[XVS:]%10
    \label{ex:nikanne:10} 
    \exi{SVX:} 
    \gll Tytöt istuivat tuolilla.    \\    
         girl.\textsc{pl}.\textsc{nom} sit.\textsc{past}.\oldstylenums{3}\textsc{pl} chair.\textsc{ade}  \\
    \glt  `The girls sat on the chair.’    
    \exi{XVS:} 
    \gll Tuolilla istuivat tytöt.\\
         chair.\textsc{ade} sit.\textsc{past}.\oldstylenums{3}\textsc{pl} girl.\textsc{pl}.\textsc{nom}  \\
    \exi{SXV:} 
    \gll   Tytöt tuolilla istuivat.\\
           girl.\textsc{pl}.\textsc{nom} chair.\textsc{ade} sit.\textsc{past}.\oldstylenums{3}\textsc{pl}  \\
    \exi{XSV:}
    \gll     Tuolilla tytöt istuivat.\\
              chair.\textsc{ade} girl.\textsc{pl}.\textsc{nom} sit.\textsc{past}.\oldstylenums{3}\textsc{pl}  \\
\end{xlist}
\z

There is, thus, no obvious reason to assume that the subject of the sentence must have a particular syntactic position or that the grammatical cases for the subject (\textsc{nom}) and the object (\textsc{par} or \textsc{acc}) are assigned to particular positions in the constituent tree.

\section{A new look at the finite sentence of Finnish}%5
In this section, I will suggest an alternative way to analyse the morpho-syntactic structure of the \ili{Finnish} finite\is{finiteness} sentence. 
\subsection{From constituents to tiers}

The analysis is based on Tiernet Theory \citep{Nikanne1990,Nikanne2002,Nikanne2008,Nikannefc,Pörn2004,Paulsen2011,Petrova2011}, which is a generative theory of grammar and based on Jackendoff’s Conceptual Semantics\is{semantics} (\citeyear{Jackendoff1972,Jackendoff1983,Jackendoff1987,Jackendoff2002}). The characteristic property of Tiernet is that the grammar is based on several very simple micro-modules and links between the micro-representations generated by these micro-modules. (See \citealt{Nikannefc}, for a detailed introduction to the theory.)

The binary constituent structure in \REF{ex:nikanne:3} (based on \citealt{HolmbergEtAl1993}) can be presented in a horizontal position so that the maximal and middle nodes are at one level, and heads and specifiers on another level, as shown in \figref{ex:nikanne:11}. The head and specifier nodes are linked to other domains of the language system: information structure, (inflectional) \isi{morphology}, and lexical categories. The sentence initial positions Spec(CP\is{complementizer}) and Spec(AgrSP) are reserved for information structure.  The functional head positions AgrS, Neg\is{negation}, T, Aux\is{Auxiliary}, Ptc, and V are linked to \isi{morphological} and lexical categories. It is worth pointing out that every second functional head, AgrS, T, and Ptc, are linked to \isi{morphological} categories and every second functional head, Neg\is{negation}, Aux\is{Auxiliary}, and V, are linked to lexical categories. The category C is typically understood to be linked to conjunctions, but there are theories in which C is associated with abstract features of various kinds (having to do with questions, emphasis, etc.).

\begin{sidewaysfigure}\small
\caption{Information structure, morphology, and lexical categories in the Finnish finite sentence.\label{ex:nikanne:11}}
\begin{tikzpicture}[%
baseline=0pt,
column sep=.25cm,
every node/.style={anchor=base,
text height=.8em,text depth=.2em}
]    
\matrix (information) [matrix of nodes,nodes in empty cells]{
CP\is{complementizer} & C' & AgrSP & AgrS' & NegP & Neg\is{negation}' & TP & T' & AuxP & Aux\is{Auxiliary}' & PtcP & Ptc' & VP & V'\\[1cm,between origins]
Spec & C & Spec & AgrS & Spec & Neg\is{negation} & Spec & T & Spec & Aux\is{Auxiliary} & Spec & Ptc & Spec & V\\[1.5cm,between origins]
& & & & & |[text width=1.25cm,align=left]| \isi{negation} word \textit{ei} & & & & |[text width=1.25cm,align=left]| auxiliary\is{Auxiliary} \textit{ole} `be' & & & & verb\\[1.5cm,between origins]
& & & |[text width=1.6cm,align=left]| \isi{Person}\is{person} \isi{agreement} & & &  & |[text width=.75cm,align=left]| Tense\is{tense}\slash mood & & & & \isi{Participle}\is{participle} & & \\[1.5cm,between origins]
\textsc{focus\oldstylenums{1}} & & \textsc{topic} & & & & & & & & & & & \\
}; 
\foreach \x [remember=\x as \lastx (initially 1)] in {2,...,14} \draw (information-1-\lastx) -> (information-1-\x);
\foreach \y in {1,...,14} \draw (information-1-\y) -> (information-2-\y);
\draw (information-2-1) -> (information-5-1);
\draw (information-2-3) -> (information-5-3);
\draw (information-2-4) -> (information-4-4);
\draw (information-2-6) -> (information-3-6);
\draw (information-2-8) -> (information-4-8);
\draw (information-2-10) -> (information-3-10);
\draw (information-2-12) -> (information-4-12);
\draw (information-2-14) -> (information-3-14);
\coordinate[below left=\baselineskip and .5cm of information-2-1.south west] (lexicalUL);\coordinate[below right=\baselineskip and .5cm of information-2-14] (lexicalUR);\draw[dashed] (lexicalUL) -- (lexicalUR);
\coordinate[below=4.5\baselineskip of lexicalUL] (lexica2UL);\coordinate[below=4.5\baselineskip of lexicalUR] (lexica2UR);\draw[dashed] (lexica2UL) -- (lexica2UR);
\coordinate[below=4\baselineskip of lexica2UL] (lexica3UL);\coordinate[below=4\baselineskip of lexica2UR] (lexica3UR);\draw[dashed] (lexica3UL) -- (lexica3UR);
\coordinate[below=3.5\baselineskip of lexica3UL] (lexica4UL);\coordinate[below=3.5\baselineskip of lexica3UR] (lexica4UR);\draw[dashed] (lexica4UL) -- (lexica4UR);
\node [below=.5\baselineskip of information-3-14.south east, anchor=east] (lexical)       {\normalsize\sffamily\bfseries Lexical}; 
\node [below=5\baselineskip of information-3-14.south east, anchor=east] (morphology)     {\normalsize\sffamily\bfseries Morphology}; 
\node [below=8.5\baselineskip of information-3-14.south east, anchor=east] (informations) {\normalsize\sffamily\bfseries Information structure}; 
\end{tikzpicture}
\end{sidewaysfigure}


In the analyses that follow, we abandon constituent structure as the \isi{universal} architecture of syntax and functional nodes as syntactic categories. Information structure, \isi{morphology}, and word order are treated as separate tiers. In this way, we are able to reach the goals set above: (i) to avoid unnecessarily abstract syntactic constituents, and (ii) to keep the lexical and \isi{morphological} categories apart when it comes to finite\is{finiteness} sentence. In order to do this, we need to use a slightly different set of tools than before: 

\begin{enumerate}
\item[(i)] Hierarchies and linking are applied instead of movements.
\item[(ii)] Lexical categories, \isi{morphological} categories, and finite\is{finiteness} features are kept in separate tiers instead of putting them all in the same syntactic constituent structure.
\end{enumerate}
\subsection{Morphology} %5.2.

The \textbf{finite\is{finiteness} sentence \isi{morphological} categories} (\textbf{fsm-categories}) of \ili{Finnish} are AgrS, T, Ptc, and Pass\is{passive}. Instead of assuming a head movement, we analyse them as hierarchically organized. The hierarchy is the same as the linear order in \citegen{HolmbergEtAl1993} constituent structure:

\ea
\textit{{Hierarchy of Fsm-Categories}}\\
{AgrS > T > Ptc > Pass}
\z

The finite\is{finiteness} sentence \isi{morphological} categories select the finite\is{finiteness} sentence lexical categories in a strict hierarchical order. The status of a \isi{morphological} category in the hierarchy of fsm-categories determines its place in the picking order. The highest \isi{morphological} category has the right to select the most desired lexical categories. The desirability of lexical categories is determined in another hierarchy.

In the hierarchy of fsm-categories, there is one difference compared to \citet{HolmbergEtAl1993}, namely that \isi{passive} is part of the hierarchy but it is not is not a functional head in the theory of Holmberg et al.  As we have abandoned the functional constituent heads, we can – or should – treat \isi{passive} in the same way as the rest of the finite\is{finiteness} sentence \isi{morphology}. As already pointed out above, the \ili{Finnish} \isi{passive} has two parts: the \isi{passive} marker (\textit{ttA})\footnote{The suffix \textit{ttA} (in which A indicates \textit{a} or \textit{ä} depending on the vowel harmony) is sensitive to its morpho-phonological context and may appear as \textit{tta}, \textit{ttä}, \textit{ta}, \textit{tä}, \textit{tt, t, la, lä, ra,} or \textit{rä}. This alternation is, however, beyond the scope of this article.} next to the verb stem and the \isi{passive} personal ending (\textit{Vn})\footnote{The V in the suffix \textit{Vn} indicates vowel lengthening: V appears as the lengthening of the preceding vowel.}  in the position of AgrS-endings:

\ea%13
\label{ex:nikanne:13} 
\textit{istu-}\textbf{\textit{ta}}\textit{{}-}\textbf{\textit{an}} [sit-\textbf{\textsc{pass}}{}-\textbf{\textsc{pass}}] ‘it is sat’\\
\textit{istu-}\textbf{\textit{tt}}\textit{{}-i-}\textbf{\textit{in}} [sit-\textbf{\textsc{pass}}{}-\textsc{past}-\textbf{\textsc{pass}}] ‘it was sat’\\
\textit{istu-}\textbf{\textit{tta}}\textit{{}-isi-}\textbf{\textit{in}} [sit-\textbf{\textsc{pass}}{}-\textsc{cond}-\textbf{\textsc{pass}}] `it would be sat’\\
\textit{istu-}\textbf{\textit{tta}}\textit{{}-ne-}\textbf{\textit{en}}  [sit-\textbf{\textsc{pass}}{}-\textsc{pot}-\textbf{\textsc{pass}}] `it probably will be sat’\\
\z

In the perfect and pluperfect \isi{tense}, the \isi{passive} marker is in the participial:

\ea%12
\label{ex:nikanne:12} 
\textit{on istu-}\textbf{\textit{tt}}\textit{{}-u}   [be.\oldstylenums{3}\textsc{sg} sit-\textbf{\textsc{pass}}{}-\textsc{past}\textsc{ptc}]  ‘it has been sat’\\
\textit{oli istu-}\textbf{\textit{tt}}\textit{{}-u}   [be.\textsc{past}.\oldstylenums{3}\textsc{sg} sit-\textbf{\textsc{pass}}{}-\textsc{past}\textsc{ptc}] ‘it had been sat’
\z

The \isi{person} of the auxiliary\is{Auxiliary} \textit{olla} ‘be’ is traditionally analyzed as \oldstylenums{3}\textsc{sg} as \oldstylenums{3}\textsc{sg} is the neutral or default \isi{person} in the \ili{Finnish} grammar. In colloquial \ili{Finnish}, the perfect and pluperfect \isi{tenses} are not always following the pattern given above: it is common to double the \isi{passive} \isi{morphology} in the auxiliary\is{Auxiliary}: \textit{ol-}\textbf{\textit{la}}\textit{{}-}\textbf{\textit{an}} \textit{istu-}\textbf{\textit{tt}}\textit{{}-u} [be-\textsc{pass}-\textsc{pass} sit-\textsc{pass}-\textsc{past}\textsc{ptc}] `it has been done’ instead of \textit{on istu-}\textbf{\textit{tt}}\textit{{}-u} [be.\oldstylenums{3}\textsc{sg} sit-\textbf{\textsc{pass}}{}-\textsc{past}\textsc{ptc}] ‘it has been sat.’

As mentioned above, the lexical categories of the finite\is{finiteness} sentence form a hierarchy. I use the term \textbf{finite\is{finiteness} sentence} \textbf{lexical categories} (\textbf{fsl-categories}) for the lexical categories that are characteristic for the finite\is{finiteness} sentence, i.e. Neg\is{negation}, Aux\is{Auxiliary}, and V. The hierarchy of these categories is as follows: 

\ea
\textit{{Hierarchy of Finite\is{finiteness} Sentence Lexical Categories}}\\
Neg\is{negation} > Aux\is{Auxiliary} > V
\z

The higher the lexical category is in the hierarchy the more valuable it is from the point of view of \isi{morphological} categories. Just as was the case with \isi{morphological} categories, the hierarchy of lexical categories corresponds to their order in the syntactic tree in the constituent analysis by \citet{HolmbergEtAl1993}. 

The \isi{morphological} form follows from general principles. 

\ea
\textit{{Linking between Finite\is{finiteness} Sentence \isi{Morphological}\is{morphology/morphological} and Lexical Categories}}
\begin{enumerate}
 \item[A.] Each fsl-category must always be selected by at least one fsm-category.
 \item[B.] Fsm-categories select a maximal \isi{number} of fs-\isi{morphological} categories from left to right following the lexical and \isi{morphological} hierarchies, with exceptions (i) and (ii).
\begin{enumerate}
 \item[(i)]  Neg\is{negation} can only be selected by AgrS.
 \item[(ii)] Ptc can only select V.
\end{enumerate}
\end{enumerate}
\z

The principles of selection based on these hierarchies correspond to the head movement in \citet{HolmbergEtAl1993}.

The \isi{morphological} categories must have values. These values are called $\varphi $-features in generative grammar. I will call them \textbf{finite\is{finiteness} features}. Traditionally, the finite\is{finiteness} sentence \isi{morphology} has been divided into such categories as voice, \isi{tense}, mood, \isi{person} and \isi{number}. Thus, the \isi{morphological} category T may carry such finite\is{finiteness} features as present \isi{tense}, past \isi{tense}, conditional mood, imperative mood, or potential mood. AgrS may carry \isi{person} and \isi{number} features, or the \isi{passive} feature.  

The finite\is{finiteness} features are organized in a constituent structure as in \figref{fig:nikanne:FinFeatures} (\textsc{pass} = \isi{passive}, \textsc{pres} = present \isi{tense}, \textsc{cond} = conditional mood, \textsc{imp} = imperative mood, \textsc{pot} = potential mood). Arrow indicates selection, i.e. dependency relation).

\begin{figure}[p]
\caption{The hierarchy of finite features and linking between finite features, fsm-categories and fsl-categories.}
\label{fig:nikanne:FinFeatures}
\begin{forest}
[m-root,name=mroot [\textsc{voice} [\textsc{active} [\textsc{person} [1,tier=word,name=One] [2] [3]] [\textsc{number} [\textsc{sg}] [\textsc{pl}]] ] [\textsc{passive} [\textsc{pass},tier=word] ] ] [\textsc{tense} [\textsc{past},tier=word] [\textsc{pres},tier=word] ] [\textsc{mood} [\textsc{cond},tier=word] [\textsc{imp},tier=word] [\textsc{pot},tier=word,name=pot] ] ]
\node[above=2\baselineskip of mroot] (fsm) {fsm-category (AgrS > T > Ptc > Pass\is{passive})};
\node[above=2\baselineskip of fsm] (fsl) {fsl-category (Neg\is{negation} > Aux\is{Auxiliary} > V)};
\draw [-{Stealth[]}] (mroot) -- (fsm);
\draw [-{Stealth[]}] (fsm) -- (fsl);
\path let \p1 = ($ (fsm) !.5! (mroot) $), \p2=(One.west) in coordinate (left) at (\x2,\y1);
\path let \p1 = ($ (fsm) !.5! (mroot) $), \p2=( $ (pot.east) +(2cm,0cm)$ ) in coordinate (right) at (\x2,\y1);
\draw[dashed] (left) -- (right);
\path let \p1 = ($ (fsm) !.5! (fsl) $), \p2=(One.west) in coordinate (left2) at (\x2,\y1);
\path let \p1 = ($ (fsm) !.5! (fsl) $), \p2=( $ (pot.east) +(2cm,0cm)$ ) in coordinate (right2) at (\x2,\y1);
\draw[dashed] (left2) -- (right2);
\path let \p1 = ($ (fsl) +(0cm,.5cm) $), \p2=(One.west) in coordinate (left3) at (\x2,\y1);
\path let \p1 = ($ (fsl) +(0cm,.5cm) $), \p2=( $ (pot.east) +(2cm,0cm)$ ) in coordinate (right3) at (\x2,\y1);
\draw[dashed] (left3) -- (right3);
\node [below= .3cm of right.east,anchor=east] {\sffamily\bfseries Finite\is{finiteness} features};
\node [above= .3cm of right.east,anchor=east] {\sffamily\bfseries Finite\is{finiteness} sentence \isi{morphological} categories};
\node [above= .3cm of right2.east,anchor=east] {\sffamily\bfseries Finite\is{finiteness} sentence lexical categories};
\end{forest}
\end{figure}

All the finite\is{finiteness} features of a finite\is{finiteness} sentence are “collected” by constituency to the same m-root. The m-root represents the whole set of features, and it selects (arrow indicates selection) a finite\is{finiteness} sentence \isi{morphological} category. The organization of the finite\is{finiteness} features is very much the same as in traditional grammars. The new idea is the linking between finite\is{finiteness} features and finite\is{finiteness} sentence \isi{morphology}. The feature system above is based on the grammar of \ili{Finnish}, but most parts of it are similar to other languages. The finite\is{finiteness} sentence morphemes (fsm-categories) and the linking between the \isi{morphological} categories and the features is more language specific. 

The fsm-categories can carry different finite\is{finiteness} features, i.e. they can carry a part of the constituent tree in \figref{fig:nikanne:FinFeatures}.  The restrictions – and possibilities – for categories of AgrS, T, and Ptc to carry finite\is{finiteness} features are given in \REF{ex:nikanne:14}. The circles show the range of finite\is{finiteness} features each fsm-category may carry in \ili{Finnish}.

\ea%14
\label{ex:nikanne:14}
The possible finite\is{finiteness} features of the fsm-categories AgrS, T, Ptc, and Pass\is{passive} in \ili{Finnish}.\\
\resizebox{.8\textwidth}{!}{\begin{forest}
[m-root,name=mroot [\textsc{voice},name=voice [\textsc{active} [\textsc{person} [1,tier=word,name=One] [2] [3]] [\textsc{\isi{number}},name=number [\textsc{sg}] [\textsc{pl}]] ] [\textsc{passive} [\textsc{pass},tier=word,name=pass] ] ] [\textsc{tense},name=tense,s sep=1mm,before drawing tree={x+=15mm}  [\textsc{past},tier=word,before drawing tree={x+=15mm},name=past] [\textsc{pres},tier=word,before drawing tree={x+=15mm}] ] [\textsc{mood}, before drawing tree={x+=15mm}, s sep=1mm [\textsc{cond},tier=word,before drawing tree={x+=15mm}] [\textsc{imp},tier=word,before drawing tree={x+=15mm}] [\textsc{pot},tier=word,name=pot,before drawing tree={x+=15mm}] ] ]
\node [above=\baselineskip of mroot] (fsm) {fsm-category};
\draw [-{Stealth[]}] (mroot) -- (fsm);
\node [fit = (tense) (pot),draw,loosely dashed,ellipse,inner sep=2pt,"T" {below right,draw,thick}] {};
\node [fit = (voice) (pass) (One),draw,densely dashed,ellipse,inner sep=2pt,"AgrS" {below left,draw,thick}] {};
\node [fit = (number) (past),draw,dotted,inner sep=-3pt,"Ptc" {below right,draw,thick}] {};
\node [fit = (pass) (pass),draw,solid,ellipse,inner sep=2pt,"Pass\is{passive}" {below left,draw,thick}] {};
\end{forest}}
\z\vspace{\baselineskip}\clearpage
% \todo[inline]{We ommited the ``pass'' group}

The categories AgrS and T do not share any values but Ptc can carry a part of the values of AgrS, namely \isi{number} and \isi{passive}, as well as a part of the values of T, namely past or present. The \isi{person} values can only be carried by AgrS and the mood values only by T. The main principle is that the branches in the fs-\isi{morphological} constituent structure must be interpreted as “pick only one,” with the excpetion that \textsc{voice} can co-occur wth \textsc{tense} and \textsc{mood} and \textsc{person} can co-occur with \textsc{number}. We can formulate this into a principle of \ili{Finnish} grammar:

\ea
\textit{{Co-occurrence of finite\is{finiteness} features in Finnish}}\\
The sisters of the finite\is{finiteness} feature constituent structure cannot co-occur in the same finite\is{finiteness} sentence, except (i) and (ii).
\begin{enumerate}
 \item[(i)] \textsc{voice}, \textsc{tense}, and \textsc{mood} can co-occur with each other.
 \item[(ii)] \textsc{person} can co-occur with \textsc{number}.
\end{enumerate}
\z

We should keep in mind that the finite\is{finiteness} sentence is a whole, and one finite\is{finiteness} sentence can only express one set of finite\is{finiteness} features (as represented in figure 3).   There is only one set of fsm-categories per finite\is{finiteness} sentence, and therefore we can say that the finiteness of the finite\is{finiteness} sentence is based on the \isi{morphology}. The only exception of this is \textsc{tense}, which we will discuss shortly. As a reminder of this, I mark the set of fsm-categories that belong to the same finite\is{finiteness} sentence with brackets and subscript index \textbf{fs} (“finite\is{finiteness} sentence”).

The principle of the unity of the finite\is{finiteness} sentence can be formulated as follows:
\ea
\textit{{Finite\is{finiteness} sentence as a unit (in \ili{Finnish})}}\\
In a finite\is{finiteness} sentence, there cannot be more than one instance of each finite\is{finiteness} feature. The only exception is \textsc{tense} of which there can be two instances.
\z

The perfect and pluperfect \isi{tense} can be described as having two instances of \isi{tense}: one in auxiliary\is{Auxiliary} and another one in the \isi{participle}. This is an exception to the main principle that each finite\is{finiteness} feature cannot be expressed more than once.

In theory by Holmberg \& al. the functional heads AgrS and T were supposed to always be present in a finite\is{finiteness} sentence of \ili{Finnish}. We must add this principle in the present system. In addition, the lexical category V is always present in a finite\is{finiteness} sentence. The principle is as follows:

\ea
\textit{{Obligatory categories in \ili{Finnish} finite\is{finiteness} sentence}}\\\relax
{The fsm-categories AgrS and T as well as the fsl-category V are obligatory in a finite\is{finiteness}   sentence.}
\z

Here is an example of the system. The example in \REF{ex:nikanne:15} \textit{istuisimme} [eat.\textsc{cond}.\oldstylenums{1}\textsc{pl}] ‘we would eat’ can be analyzed as follows and \textit{istumme} [eat.\oldstylenums{1}\textsc{pl}] ‘we sit’:\newpage

\begin{multicols}{2}
\ea%15
    \label{ex:nikanne:15}
\ea
   
    \gll istuisimme\\
	eat-\textsc{cond}-\oldstylenums{1}\textsc{pl}\\
    \glt‘we would eat’ 
  
\begin{forest}
for tree={grow=north}
[V [AgrS{]},edge={Stealth[]}- [root,edge={Stealth[]}- [\textsc{voice} [\textsc{active} [\textsc{number} [\textsc{pl}] ] [\textsc{person} [1,tier=word] ] ] ] ] ] [{[}\textsubscript{fs} T,edge={Stealth[]}- [root,edge={Stealth[]}- [\textsc{mood} [\textsc{cond},tier=word] ] ] ] ]
\end{forest}

\ex
\gll istumme\\
    sit-\oldstylenums{1}\textsc{pl}\\
  
\begin{forest}
for tree={grow=north}
[V [AgrS{]},edge={Stealth[]}- [root,edge={Stealth[]}- [\textsc{voice} [\textsc{active} [\textsc{number} [\textsc{pl}] ] [\textsc{person} [1,tier=word] ] ] ] ] ] [{[}\textsubscript{fs} T,edge={Stealth[]}- [root,edge={Stealth[]}- [\textsc{tense} [\textsc{pres},tier=word] ] ] ] ]
\end{forest}
 \z
\z
\end{multicols}

The perfect and pluperfect \isi{tenses} can be analyzed as follows:

\begin{multicols}{2}
\ea%16
\label{ex:nikanne:16}
\ea
\gll   olemme istuneet\\
             be-\oldstylenums{1}\textsc{pl} sit-\textsc{past}\textsc{ptc}-\textsc{pl}\\
  
\resizebox{.35\textwidth}{!}{\begin{forest}
for tree={grow=north}
[Aux\is{Auxiliary},name=Aux [T,name=T,edge={Stealth[]}- [m-root,edge={Stealth[]}- [\textsc{tense},name=tense [\textsc{pres},tier=word,name=pres] ] ] ] [{[}\textsubscript{fs} AgrS,edge={Stealth[]}- [m-root,name=mroot,edge={Stealth[]}- [\textsc{voice} [\textsc{active} [\textsc{number},name=number [\textsc{pl},tier=word ]] [\textsc{person} [1,tier=word]] ] ] ] ] ] ] ]
\node[baseline=-.75ex,right=1em of tense] (tense2) {\textsc{tense}};
\path let \p1=(tense2), \p2=(pres) in node at (\x1,\y2) (past) {\textsc{past}};
\path let \p1=(tense2), \p2=(Aux) in node at (\x1,\y2) (V) {V};
\path let \p1=(tense2), \p2=(T) in node at (\x1,\y2) (ptc) {Ptc]};
\path let \p1=(tense2), \p2=(mroot) in node at (\x1,\y2) (mroot2) {m-root};
\draw (past) -- (tense2); \draw (tense2) -- (mroot2);
\draw [-{Stealth[]}] (mroot2) -- (ptc);\draw [-{Stealth[]}] (ptc) -- (V);
\draw[dashed] (number.south) -- (mroot2.north);
\end{forest}}
 
\ex
\gll   olimme istuneet\\
                be-\textsc{past}-\oldstylenums{1}\textsc{pl} sit-\textsc{past}\textsc{ptc}-\textsc{pl}\\


\resizebox{.35\textwidth}{!}{\begin{forest}
for tree={grow=north}
[Aux\is{Auxiliary},name=Aux [T,name=T,edge={Stealth[]}- [m-root,edge={Stealth[]}- [\textsc{tense},name=tense [\textsc{past},tier=word,name=pres] ] ] ] [{[}\textsubscript{fs} AgrS,edge={Stealth[]}- [m-root,name=mroot,edge={Stealth[]}- [\textsc{voice} [\textsc{active} [\textsc{number},name=number [\textsc{pl},tier=word ]] [\textsc{person} [1,tier=word]] ] ] ] ] ] ] ]
\node[baseline=-.75ex,right=1em of tense] (tense2) {\textsc{tense}};
\path let \p1=(tense2), \p2=(pres) in node at (\x1,\y2) (past) {\textsc{past}};
\path let \p1=(tense2), \p2=(Aux) in node at (\x1,\y2) (V) {V};
\path let \p1=(tense2), \p2=(T) in node at (\x1,\y2) (ptc) {Ptc]};
\path let \p1=(tense2), \p2=(mroot) in node at (\x1,\y2) (mroot2) {m-root};
\draw (past) -- (tense2); \draw (tense2) -- (mroot2);
\draw [-{Stealth[]}] (mroot2) -- (ptc);\draw [-{Stealth[]}] (ptc) -- (V);
\draw[dashed] (number.south) -- (mroot2.north);
\end{forest}}
 \z
\z
\end{multicols}

The participles used in the finite\is{finiteness} sentence are in the same \isi{number} (\textsc{sg} or \textsc{pl}) as the whole finite\is{finiteness} sentence \isi{morphology}. This \isi{agreement} can be described as feature spreading. The feature spreading is marked by dashed constituent lines. The spreading lines are linked directly to the m-root: the participial forms are just agreeing\is{agreement} with the \isi{number} of the finite\is{finiteness} sentence. The primary expression of \isi{number} is in the \isi{morphological} form of the \isi{negation} word.

As discussed earlier, the \ili{Finnish} morpho-syntax is peculiar in the way that the finite\is{finiteness} features \isi{tense} and mood cannot co-occur in the same finite\is{finiteness} verb: the fsm-category T expresses either mood or \isi{tense} but not both. There is however, a solution: \isi{participle}. If \isi{tense} and mood co-occur, the mood is expressed by fsm-category T that selects Aux\is{Auxiliary} and \isi{tense} by fsm-category Ptc that selects V, for instance in \REF{ex:nikanne:17}.\largerpage


\noindent\parbox{\textwidth}{\ea%17
    \label{ex:nikanne:17}
    \gll	  {\upshape olisimme} {\upshape istuneet}\\
  be-\textsc{cond}-\oldstylenums{1}\textsc{pl} sit-\textsc{past}\textsc{ptc}-\textsc{pl}\\
\glt  ‘we would have sat’

\resizebox{.3\textwidth}{!}{\begin{forest}
for tree={grow=north}
[Aux\is{Auxiliary},name=Aux [T,name=T,edge={Stealth[]}- [m-root,edge={Stealth[]}- [\textsc{mood},name=tense [\textsc{cond},tier=word,name=pres] ] ] ] [{[}\textsubscript{fs} AgrS,edge={Stealth[]}- [m-root,name=mroot,edge={Stealth[]}- [\textsc{voice} [\textsc{active} [\textsc{number},name=number [\textsc{pl},tier=word ]] [\textsc{person} [1,tier=word]] ] ] ] ] ] ] ]
\node[baseline=-.75ex,right=1em of tense] (tense2) {\textsc{tense}};
\path let \p1=(tense2), \p2=(pres) in node at (\x1,\y2) (past) {\textsc{past}};
\path let \p1=(tense2), \p2=(Aux) in node at (\x1,\y2) (V) {V};
\path let \p1=(tense2), \p2=(T) in node at (\x1,\y2) (ptc) {Ptc]};
\path let \p1=(tense2), \p2=(mroot) in node at (\x1,\y2) (mroot2) {m-root};
\draw (past) -- (tense2); \draw (tense2) -- (mroot2);
\draw [-{Stealth[]}] (mroot2) -- (ptc);\draw [-{Stealth[]}] (ptc) -- (V);
\draw[dashed] (number.south) -- (mroot2.north);
\end{forest}}
 \z}

In this way, both \isi{tense} and mood can be expressed in the same finite\is{finiteness} sentence even though they do not have room in the same verb form. 

Another \isi{morphological} peculiarity in \ili{Finnish} is that \isi{passive} is expressed by two endings: the \isi{passive} marker and the \isi{passive} personal suffix. For instance:


\ea%18
    \label{ex:nikanne:18}
   \gll 	   istu-tt-i-in\\
      sit-\textsc{pass}-\textsc{past}-\textsc{pass}\\
\glt  ‘it was sat’
\z
\noindent I suggest the following solution:\largerpage

\ea%19
\label{ex:nikanne:19}
\resizebox{.3\textwidth}{!}{\begin{forest}
for tree={grow=north}
[V [AgrS{]},edge={Stealth[]}- [m-root,name=mroot3,edge={Stealth[]}-]] [T,edge={Stealth[]}- [m-root,edge={Stealth[]}- [\textsc{tense} [\textsc{past},tier=word]]]] [{[}\textsubscript{fs} Pass\is{passive}, edge={Stealth[]}- [m-root,edge={Stealth[]}- [\textsc{voice},name=voice [\textsc{passive} [\textsc{pass},tier=word,baseline]] ] ] ] ]
\draw (voice.south) -- (mroot3.north);
\end{forest}}
\z 

The solution in \REF{ex:nikanne:20} is rather obvious in a “nonlinear” \isi{morphology} as the present approach. The node \textsc{voice} is shared by the m-roots of the fsm-categories Pass\is{passive} and AgrS.  \citet{HakulinenKarlsson1979} suggest that \isi{passive} is “the fourth \isi{person}.” The motivation is that the \isi{passive} form has a \isi{person} suffix, i.e. AgrS. In the present approach, it is not necessary to assume a fourth \isi{person}. The \isi{passive} is just linked to the fsm-category AgrS, in addition to the fsm-category Pass\is{passive}, and the \isi{passive} voice is expressed in two fsm-categories.

The \isi{negation} word \textit{e(i)} 'not' is selected by the fsm-category AgrS. The mood or the \isi{tense} are then expressed by the verb:\largerpage[3]

\noindent\parbox{\textwidth}{\begin{multicols}{2}
\ea%20
    \label{ex:nikanne:20}
   
\ea
\gll  emme istuisi\\
not-\oldstylenums{1}\textsc{pl} sit-\textsc{cond}\\
\glt ‘we would not eat’
\resizebox{.3\textwidth}{!}{\begin{forest}
for tree={grow=north}
[Neg\is{negation},name=Neg [{[}\textsubscript{fs} AgrS,name=AgrS,edge={Stealth[]}- [root,name=root,edge={Stealth[]}- [\textsc{voice},name=voice [\textsc{active} [\textsc{number},name=number [\textsc{pl},tier=word,name=pl ]] [\textsc{person} [1,tier=word]] ] ] ] ] ] ] ]
\node[baseline=-.75ex,right=2em of voice] (mood) {\textsc{mood}};
\path let \p1=(mood), \p2=(pl) in node at (\x1,\y2) (cond) {\textsc{cond}};
\path let \p1=(mood), \p2=(root) in node at (\x1,\y2) (root2) {root};
\path let \p1=(mood), \p2=(AgrS) in node at (\x1,\y2) (T) {T]};
\path let \p1=(mood), \p2=(Neg) in node at (\x1,\y2) (V) {V};
\draw (cond) -- (mood); \draw (mood) -- (root2);
\draw [-{Stealth[]}] (root2) -- (T);\draw [-{Stealth[]}] (T) -- (V);
\end{forest}}
 
\ex
\gll emme istu\\
       not-\oldstylenums{1}\textsc{pl} sit\\
\glt      ‘we do not sit’  
\resizebox{.3\textwidth}{!}{\begin{forest}
for tree={grow=north}
[Neg\is{negation},name=Neg [{[}\textsubscript{fs} AgrS,name=AgrS,edge={Stealth[]}- [root,name=root,edge={Stealth[]}- [\textsc{voice},name=voice [\textsc{active} [\textsc{number},name=number [\textsc{pl},tier=word,name=pl ]] [\textsc{person} [1,tier=word]] ] ] ] ] ] ] ]
\node[baseline=-.75ex,right=2.5em of voice] (mood) {\textsc{tense}};
\path let \p1=(mood), \p2=(pl) in node at (\x1,\y2) (cond) {\textsc{present}};
\path let \p1=(mood), \p2=(root) in node at (\x1,\y2) (root2) {root};
\path let \p1=(mood), \p2=(AgrS) in node at (\x1,\y2) (T) {T]};
\path let \p1=(mood), \p2=(Neg) in node at (\x1,\y2) (V) {V};
\draw (cond) -- (mood); \draw (mood) -- (root2);
\draw [-{Stealth[]}] (root2) -- (T);\draw [-{Stealth[]}] (T) -- (V);
\end{forest}}
\z
\z
\end{multicols}}

We need to stipulate an exception to the system. But this stipulation is something that any system – known so far – must do. There is a third peculiarity in \ili{Finnish}: if the finite\is{finiteness} sentence has a \isi{negation} word, the verb or the auxiliary\is{Auxiliary} appears in the participial form in the past \isi{tense}. The auxiliary\is{Auxiliary} is in the participial form in the pluperfect as the pluperfect (normally) consists of an auxiliary\is{Auxiliary} in the past \isi{tense} plus the verb in a past participial form:

\ea%21
    \label{ex:nikanne:21}
\ea\label{ex:nikanne:21a}
\gll   Tyttö ei istunut tuolissa.            \\
girl.\textsc{nom} not.\oldstylenums{3}\textsc{sg} sit.\textsc{past}\textsc{ptc} chair.\textsc{ine}  \\
\glt                   ‘The girl did not sit on the chair.’    

                   (Negation and past \isi{tense}: V in a participial form.)          

\ex\label{ex:nikanne:21b}
\gll        Tytöt eivät istuneet tuolissa.            \\
                    girl.\textsc{pl}.\textsc{nom} not.\oldstylenums{3}\textsc{pl} sit.\textsc{past}\textsc{ptc} chair.\textsc{ine}  \\
\glt                   ‘The girls did not sit on the chair.’    

     (Negation and perfect \isi{tense}: Aux\is{Auxiliary} appears in a participial form.)

\ex\label{ex:nikanne:21c}
\gll      Tytöt eivät ole istuneet tuolissa.\\
                 girl.\textsc{pl.nom} not.\oldstylenums{3}\textsc{pl} be sit.\textsc{past}\textsc{ptc} chair.\textsc{ine} \\
\glt                   ‘The girl did have not sat on the chair.’    

 (Negation and pluperfect \isi{tense}: Aux\is{Auxiliary} appears in a participial form.)
\z
\z


This exception can tentatively be formalized as follows (i.e. the past \isi{tense} \isi{morphology} of Aux\is{Auxiliary} or V is replaced by the participial form in the presence of Neg\is{negation}).

\ea
\textit{{Past \isi{tense} in the negative sentence in Finnish}}\\
{If the fs-\isi{morphological} category AgrS selects Neg\is{negation}, then the value \textsc{past} of the fs-  \isi{morphological} category T appears as past \isi{participle}.} 
\z

Thus, the fsm-category that selects verb in \REF{ex:nikanne:21a} and the auxiliary\is{Auxiliary} in (\ref{ex:nikanne:21b}, \ref{ex:nikanne:21c}) is functionally T but it appears as a participial. The participial form is able to express \isi{tense}, and the \ili{Finnish} grammar takes an advantage of this property in the perfect and the pluperfect \isi{tenses}. Why the (simple) past \isi{tense} is expressed using a participial form in negative sentence, seems to be just a strange detail of the \ili{Finnish} grammar. What we know, however, that this is made possible by the facts (i) that a particple \textit{can} be selected by \isi{tense} and that (ii) the \isi{person} and \isi{number} select the \isi{negation} word when it is possible.

For instance the finite\is{finiteness} forms in \REF{ex:nikanne:21} can be described as in \REF{ex:nikanne:22}:

\begin{exe}%22
% \begin{multicols}{2}
\ex\label{ex:nikanne:22}
\begin{multicols}{2}
\begin{xlista}
\ex \gll   ei istunut            \\
     not-\oldstylenums{3}\textsc{sg} sit-\textsc{past}\textsc{ptc}  \\
\glt        ‘(she/he/it) did not sit’  

\begin{forest}
for tree={grow=north}
[Neg\is{negation},name=Neg [{[}\textsubscript{fs} AgrS,name=AgrS,edge={Stealth[]}- [m-root,name=root,edge={Stealth[]}- [\textsc{voice},name=voice [\textsc{active} [\textsc{number},name=number [\textsc{sg},tier=word,name=pl ]] [\textsc{person} [3,tier=word]] ] ] ] ] ] ] ]
\node[baseline=-.75ex,right=2em of voice] (mood) {\textsc{tense}};
\path let \p1=(mood), \p2=(pl) in node at (\x1,\y2) (cond) {\textsc{past}};
\path let \p1=(mood), \p2=(root) in node at (\x1,\y2) (root2) {m-root};
\path let \p1=(mood), \p2=(AgrS) in node at (\x1,\y2) (T) {T\slash Ptc]};
\path let \p1=(mood), \p2=(Neg) in node at (\x1,\y2) (V) {V};
\draw (cond) -- (mood); \draw (mood) -- (root2);
\draw [-{Stealth[]}] (root2) -- (T);\draw [-{Stealth[]}] (T) -- (V);
\draw[dashed] (number.south) -- (root2.north);
\end{forest}
 
\ex
\gll  eivät istuneet     \\
not-\oldstylenums{3}\textsc{pl} sit-\textsc{past}\textsc{ptc}-\textsc{pl}  \\
\glt ’(they) did not sit’

\begin{forest}
for tree={grow=north}
[Neg\is{negation},name=Neg [{[}\textsubscript{fs} AgrS,name=AgrS,edge={Stealth[]}- [m-root,name=root,edge={Stealth[]}- [\textsc{voice},name=voice [\textsc{active} [\textsc{number},name=number [\textsc{pl},tier=word,name=pl ]] [\textsc{person} [3,tier=word]] ] ] ] ] ] ] ]
\node[baseline=-.75ex,right=2em of voice] (mood) {\textsc{tense}};
\path let \p1=(mood), \p2=(pl) in node at (\x1,\y2) (cond) {\textsc{past}};
\path let \p1=(mood), \p2=(root) in node at (\x1,\y2) (root2) {m-root};
\path let \p1=(mood), \p2=(AgrS) in node at (\x1,\y2) (T) {T\slash Ptc]};
\path let \p1=(mood), \p2=(Neg) in node at (\x1,\y2) (V) {V};
\draw (cond) -- (mood); \draw (mood) -- (root2);
\draw [-{Stealth[]}] (root2) -- (T);\draw [-{Stealth[]}] (T) -- (V);
\draw[dashed] (number.south) -- (root2.north);
\end{forest}
\end{xlista}
\end{multicols}
\vspace{2\baselineskip}

\begin{multicols}{2}
\begin{xlista}\setcounter{xnumii}{2}
\ex
\parbox[t]{\textwidth}{%
\gll    eivät ole istuneet\\
       not-\textsc{pl} be  sit-\textsc{past}\textsc{ptc}-\textsc{pl} \\
\glt    ‘(they) have not sat’

\resizebox{.35\textwidth}{!}{%
\begin{forest} for tree={grow=north}
[Neg\is{negation},name=Neg [{[}\textsubscript{fs} AgrS,name=AgrS,edge={Stealth[]}- [m-root,name=root,edge={Stealth[]}- [\textsc{voice},name=voice [\textsc{active} [\textsc{number},name=number [\textsc{pl},tier=word,name=pl ]] [\textsc{person} [3,tier=word]] ] ] ] ] ] ] ]
\node[baseline=-.75ex,right=2em of voice] (mood) {\textsc{tense}};
\path let \p1=(mood), \p2=(pl) in node at (\x1,\y2) (cond) {\textsc{pres}};
\path let \p1=(mood), \p2=(root) in node at (\x1,\y2) (root2) {m-root};
\path let \p1=(mood), \p2=(AgrS) in node at (\x1,\y2) (T) {T\slash Ptc};
\path let \p1=(mood), \p2=(Neg) in node at (\x1,\y2) (V) {Aux};
\draw (cond) -- (mood); \draw (mood) -- (root2);
\draw [-{Stealth[]}] (root2) -- (T);\draw [-{Stealth[]}] (T) -- (V);
\draw[dashed] (number.south) -- (root2.north);
\node[baseline=-.75ex,right=1.5em of mood] (mood2) {\textsc{tense}};
\path let \p1=(mood2), \p2=(cond) in node at (\x1,\y2) (cond2) {\textsc{past}};
\path let \p1=(mood2), \p2=(root2) in node at (\x1,\y2) (root3) {m-root};
\path let \p1=(mood2), \p2=(T) in node at (\x1,\y2) (T2) {Ptc]};
\path let \p1=(mood2), \p2=(Aux) in node at (\x1,\y2) (V2) {V};
\draw (cond2) -- (mood2); \draw (mood2) -- (root3);
\draw [-{Stealth[]}] (root3) -- (T2);\draw [-{Stealth[]}] (T2) -- (V2);
\draw[dashed] (number.south) -- (root3.north);
\end{forest}
}}
 
\ex
\parbox[t]{\textwidth}{%
\gll   eivät olleet istuneet\\
    not-\textsc{pl}3 be-\textsc{past}\textsc{ptc}-\textsc{pl} sit-\textsc{past}\textsc{ptc}-\textsc{pl}\\
\glt    ‘(they) had not sat’

\resizebox{.35\textwidth}{!}{%
\begin{forest} for tree={grow=north}
[Neg\is{negation},name=Neg [{[}\textsubscript{fs} AgrS,name=AgrS,edge={Stealth[]}- [m-root,name=root,edge={Stealth[]}- [\textsc{voice},name=voice [\textsc{active} [\textsc{number},name=number [\textsc{pl},tier=word,name=pl ]] [\textsc{person} [3,tier=word]] ] ] ] ] ] ] ]
\node[baseline=-.75ex,right=2em of voice] (mood) {\textsc{tense}};
\path let \p1=(mood), \p2=(pl) in node at (\x1,\y2) (cond) {\textsc{past}};
\path let \p1=(mood), \p2=(root) in node at (\x1,\y2) (root2) {m-root};
\path let \p1=(mood), \p2=(AgrS) in node at (\x1,\y2) (T) {T\slash Ptc};
\path let \p1=(mood), \p2=(Neg) in node at (\x1,\y2) (V) {Aux};
\draw (cond) -- (mood); \draw (mood) -- (root2);
\draw [-{Stealth[]}] (root2) -- (T);\draw [-{Stealth[]}] (T) -- (V);
\draw[dashed] (number.south) -- (root2.north);
\node[baseline=-.75ex,right=1.5em of mood] (mood2) {\textsc{tense}};
\path let \p1=(mood2), \p2=(cond) in node at (\x1,\y2) (cond2) {\textsc{past}};
\path let \p1=(mood2), \p2=(root2) in node at (\x1,\y2) (root3) {m-root};
\path let \p1=(mood2), \p2=(T) in node at (\x1,\y2) (T2) {Ptc]};
\path let \p1=(mood2), \p2=(Aux) in node at (\x1,\y2) (V2) {V};
\draw (cond2) -- (mood2); \draw (mood2) -- (root3);
\draw [-{Stealth[]}] (root3) -- (T2);\draw [-{Stealth[]}] (T2) -- (V2);
\draw[dashed] (number.south) -- (root3.north);
\end{forest}}
}
\end{xlista}
\end{multicols}
\end{exe}

The fact that \isi{tense} is expressed by a \isi{participle} in negative sentences is marked as T/Ptc, which should be understood like T that is replaced by Ptc.

When it comes to the ability of the \isi{participle} form to express \isi{tense}, the \ili{Finnish} grammar uses it also when the T is selected by \textsc{mood}. Here for instance is the past \isi{tense} (or perfect \isi{tense}, if you like) of the conditional mood (the example \REF{ex:nikanne:17} is repeated here as \REF{ex:nikanne:23}): 


\ea%23
    \label{ex:nikanne:23}
  
\gll	   {\upshape olisimme} {\upshape istuneet}\\
              be-\textsc{cond}-\oldstylenums{3}\textsc{pl} sit-\textsc{past}\textsc{ptc}-\oldstylenums{3}\textsc{pl}\\
\glt             ‘we would have sat’

\begin{forest}
for tree={grow=north}
[Aux\is{Auxiliary},name=Aux [T,name=T,edge={Stealth[]}- [m-root,edge={Stealth[]}- [\textsc{mood},name=tense [\textsc{cond},tier=word,name=pres] ] ] ] [{[}\textsubscript{fs} AgrS,edge={Stealth[]}- [m-root,name=mroot,edge={Stealth[]}- [\textsc{voice} [\textsc{active} [\textsc{number},name=number [\textsc{pl},tier=word ]] [\textsc{person} [1,tier=word]] ] ] ] ] ] ] ]
\node[baseline=-.75ex,right=1em of tense] (tense2) {\textsc{tense}};
\path let \p1=(tense2), \p2=(pres) in node at (\x1,\y2) (past) {\textsc{past}};
\path let \p1=(tense2), \p2=(Aux) in node at (\x1,\y2) (V) {V};
\path let \p1=(tense2), \p2=(T) in node at (\x1,\y2) (ptc) {Ptc]};
\path let \p1=(tense2), \p2=(mroot) in node at (\x1,\y2) (mroot2) {m-root};
\draw (past) -- (tense2); \draw (tense2) -- (mroot2);
\draw [-{Stealth[]}] (mroot2) -- (ptc);\draw [-{Stealth[]}] (ptc) -- (V);
\draw[dashed] (number.south) -- (mroot2.north);
\end{forest}

\z 

The pluperfect of negative sentences in Standard \ili{Finnish} has \isi{passive} marker only in the participial form of the V \REF{ex:nikanne:26a}. In colloquial \ili{Finnish}, it is, however, very common to have the \isi{passive} marker both in the participial form of the auxiliary\is{Auxiliary} and the participial form of the verb. (School teachers tend to have a hard time trying to make children use the Standard \ili{Finnish} form, and even educated adult writers often use the colloquial form.) 

\ea%24
    \label{ex:nikanne:24}
  \ea\label{ex:nikanne:24a}
\gll   ei ollut istuttu\\
    not be-\textsc{past}\textsc{ptc} sit-\textsc{pass}-\textsc{past}\textsc{ptc}\\
\glt    ‘it had not been sat’

\begin{tikzpicture}[
baseline=0pt,
column sep=.5cm,
row sep=.5cm,
every node/.style={anchor=base,
text height=.8em,text depth=.2em}
]    
\matrix (ex24) [matrix of nodes,nodes in empty cells]{
\textsc{pass} &[1cm] \textsc{past} & \textsc{past}\\
\textsc{passive} & \textsc{tense} & \textsc{tense}\\
\textsc{voice} & & \\
{m-root} & {m-root} & {m-root}\\
{[\textsubscript{fs} AgrS} & {T\slash Ptc} & {Ptc]}\\
{Neg} & {Aux} & {V}\\
};
\foreach \y in {1,...,3} \draw (ex24-1-\y) -> (ex24-2-\y);
\foreach \y in {1,...,3} \draw [-{Stealth[]}] (ex24-4-\y) -- (ex24-5-\y);
\foreach \y in {1,...,3} \draw [-{Stealth[]}] (ex24-5-\y) -- (ex24-6-\y);
\draw (ex24-2-2) -- (ex24-4-2);\draw (ex24-2-3) -- (ex24-4-3);\draw (ex24-3-1) -- (ex24-4-1);\draw (ex24-2-1) -- (ex24-3-1); \draw (ex24-2-1.south) -- (ex24-4-3.north);
\end{tikzpicture}

 \ex\label{ex:nikanne:24b}
\gll   ei oltu istuttu (colloquial)\\
       not be-\textsc{pass}-\textsc{past}\textsc{ptc} sit-\textsc{pass}-\textsc{ptc}\\
\glt      ‘it had not been sat’

\begin{tikzpicture}[
baseline=0pt,
column sep=.5cm,
row sep=.5cm,
every node/.style={anchor=base,
text height=.8em,text depth=.2em}
]    
\matrix (ex24) [matrix of nodes,nodes in empty cells]{
\textsc{pass} &[1cm] \textsc{past} & \textsc{past}\\
\textsc{passive} & \textsc{tense} & \textsc{tense}\\
\textsc{voice} & & \\
{m-root} & {m-root} & {m-root}\\
{[\textsubscript{fs} AgrS} & {T\slash Ptc} & {Ptc]}\\
{Neg} & {Aux} & {V}\\
};
\foreach \y in {1,...,3} \draw (ex24-1-\y) -> (ex24-2-\y);
\foreach \y in {1,...,3} \draw [-{Stealth[]}] (ex24-4-\y) -- (ex24-5-\y);
\foreach \y in {1,...,3} \draw [-{Stealth[]}] (ex24-5-\y) -- (ex24-6-\y);
\draw (ex24-2-2) -- (ex24-4-2);\draw (ex24-2-3) -- (ex24-4-3);\draw (ex24-3-1) -- (ex24-4-1);\draw (ex24-2-1) -- (ex24-3-1); \draw (ex24-2-1.south) -- (ex24-4-3.north);\draw(ex24-2-1.south)--(ex24-4-2.north);
\end{tikzpicture}
\z
\z 

In the light of the present nonlinear micro-modular approach, it is easy to understand why the colloquial form \REF{ex:nikanne:24b} is so appealing: the \isi{passive} is spread across the whole finite\is{finiteness} sentence, just like in the present the simple past \isi{tense}. The Standard \ili{Finnish} form must be learned separately as skipping the category T/Ptc is somewhat unnatural.

\subsection{Word order and information structure in tiers}

The system suggested above covers the finite\is{finiteness} features, as well as the finite\is{finiteness} sentence \isi{morphology}. One thing that is not covered is word order. Word order is a linear system, and I suggest that the system, the word order tier, is simply a linear order of word order positions, and the lexical, \isi{morphological}, syntactic, and other elements are linked to these positions. Here is a suggestion for the word order tier:

\ea
\textbf{Word order tier}

\begin{tikzpicture}
\node[anchor=base] (0) {0};
\node[anchor=base,right=2em of 0] (1) {1};
\node[anchor=base,right=2em of 1] (2) {2};
\node[anchor=base,right=2em of 2] (3) {3};
\node[anchor=base,right=2em of 3] (4) {4};
\node[anchor=base,right=2em of 4] (5) {5};
\node[anchor=base,right=2em of 5] (6) {...};
\foreach \x [remember=\x as \lastx (initially 0)] in {1,...,6} \draw[dash dot,-{Kite[]}] (\lastx) -> (\x);
\end{tikzpicture}
\z

The notation \begin{tikzpicture}[baseline] \node[anchor=base] (X) {X}; \node[anchor=base,right=2em of X] (Y) {Y}; \draw[dash dot, -{Kite[]}] (X) -> (Y);  \end{tikzpicture} indicates ‘X immediately precedes Y in linear order.’ Linear order is an asymmetric relation: if A precedes B, then B does not precede A. It has a direction: if A precedes B and B precedes C, then A precedes C. The notation indicates this asymmetry and direction.


The importance of the linear order in a modular model of grammar has lately been emphasized by \citet[111--113]{Sadock2012}. The technical difference between the word-order tier suggested above and Sadock’s model is that Sadock suggests that linear order is a uniformly either left- or right-branching tree structure, which leads to an unambiguous linear order of the terminal nodes. The word-order tier above is an assumption that is one step simpler, as the relation “precede” simply states the linear order. We will see how far we can get with the null-hypothesis.

The information structure categories \textsc{topic} and \textsc{focus}\oldstylenums{1} (a.k.a. “contrastive focus”) have their designated positions Spec(CP\is{complementizer}) and Spec(AgrSP). According to \citet{Vilkuna1989}, the word order of \ili{Finnish} is based on categories such as \textsc{contrast} (our \textsc{focus}\oldstylenums{1}) and \textsc{topic}, so in \ili{Finnish}, the information structure must be linked to the word order tier. The word inflected in the AgrS-\isi{morphology} sits in the AgrS-position. Compared to the functional constituent tree of \citet{HolmbergEtAl1993}, the word order position 0 corresponds to Spec(CP\is{complementizer}), 1 corresponds to C, 2 to Spec(AgrSP), and 3 to AgrS. 

The designated positions of the information structure categories and the AgrS can be found in the word order tier:

\ea
\textit{{Fixed links between information structure and word-order in Finnish}}\\\relax
{\textsc{focus}\oldstylenums{1} always selects word order position 0 and \textsc{topic} always selects position 2.} 
\z

The position of the word inflected in the AgrS-\isi{morphology} can be formulated as follows:
\ea
\textit{{Fixed link between fsm-category AgrS and word-order in Finnish}}\\\relax
{AgrS always selects word order position 3.} 
\z

The \isi{question} words such as \textit{mikä} `what’, \textit{kuka} `who’, \textit{miten} `how’, \textit{millloin} `when’, etc. are linked to position 0. One interpretation is, naturally, that a \isi{question} word has a contrastive focus as it represents the missing piece of information that is in the focus of the \isi{question} sentence.  

Position 1 is needed for an \isi{expletive} in certain structures, when the information structure must be made visible \citep{Nikannefc}. 


\begin{figure}\label{ex:nikanne:25}
\caption{The fixed links between the word order tier, information structure, and finite sentence morphology in Finnish.
% \todo[inline]{This is now a float}
} 
\begin{tikzpicture}
\node at (0,0) (0) {0}; \node[anchor=base,right=2em of 0] (1) {1}; \node[anchor=base,right=2em of 1] (2) {2}; \node[anchor=base,right=2em of 2] (3) {3}; \node[anchor=base,right=2em of 3] (4) {4}; \node[anchor=base,right=2em of 4] (5) {5}; \node[anchor=base,right=2em of 5] (6) {...};
\foreach \x [remember=\x as \lastx (initially 0)] in {1,...,6} \draw[dash dot,-{Kite[]}] (\lastx) -> (\x);
\node[above=2\baselineskip of 0,anchor=base] (Focus1) {\textsc{focus\oldstylenums{1}}};
\node[below=2\baselineskip of 3,anchor=base] (0,0) (AgrS) {AgrS};
\node[above=2\baselineskip of 2,anchor=base] (topic) {\textsc{topic}};
\draw [-{Stealth[]}] (Focus1) -- (0); \draw [-{Stealth[]}] (AgrS) -- (3); \draw [-{Stealth[]}] (topic) -- (2);
\path let \p1 = ($ (0) !.5! (Focus1) $), \p2=(Focus1.west) in coordinate (left) at (\x2,\y1);
\path let \p1 = ($ (0) !.5! (Focus1) $), \p2=( $ (6.east) +(2cm,0cm)$ ) in coordinate (right) at (\x2,\y1);
\draw[dashed] (left) -- (right);
\path let \p1 = ($ (0) !.5! (AgrS) $), \p2=(Focus1.west) in coordinate (left2) at (\x2,\y1);
\path let \p1 = ($ (0) !.5! (AgrS) $), \p2=( $ (6.east) +(2cm,0cm)$ ) in coordinate (right2) at (\x2,\y1);
\draw[dashed] (left2) -- (right2);
\node [above= .3cm of right.east,anchor=east] {\sffamily\bfseries Information structure};
\node [above= .3cm of right2.east,anchor=east] {\sffamily\bfseries Word order};
\node [below= .3cm of right2.east,anchor=east] {\sffamily\bfseries Finite\is{finiteness} sentence morphology};
\end{tikzpicture}
\end{figure}

The \isi{morphological} category AgrS is fixed to position 3. The \isi{morphological} categories have right to pick the higher position according to the \isi{morphological} hierarchy.  If Neg\is{negation} and Aux\is{Auxiliary} are not present, T and Agr\is{agreement} appear as inflectional categories of V, and they are located in position 3 as in \REF{ex:nikanne:26}. Ptc has the right for position 4 if Neg\is{negation} is not present and T is in position 3 together with AgrS, as illustrated in \REF{ex:nikanne:26}. If all the fs-lexical categories (Neg\is{negation}, Aux\is{Auxiliary} and V) are present, T (here: conditional mood) is in position 4 and Ptc (here past participial plural) in position 5. The active voice sentences given in (\ref{ex:nikanne:paradigm1}--\ref{ex:nikanne:paradigm8}) are repeated below as \REF{ex:nikanne:26}. The word-order position is marked above each sentence, and the categories of other tiers fixed to that position (\textsc{focus}\oldstylenums{1}, \textsc{topic}, and AgrS) are marked above the word order position:



\ea\label{ex:nikanne:26} 
  \ea\label{ex:nikanne:26a}
  \gllll \textsc{focus}\oldstylenums{1}           \textsc{topic}              AgrS\\
  0    2   3         4  5\\
    millä\textsubscript{i}               tytöt   eivät              olisi   istuneet                   t\textsubscript{i}?’\\
  what.\textsc{ade}   girl.\textsc{pl}(\textsc{nom}) not.\oldstylenums{3}\textsc{pl}         be.\textsc{cond}    sit.\textsc{ptc}.\textsc{pl}\\
  \glt  ‘What would the girls not have sat on?’ \\
  (WH-WORD AS ADVERBIAL; \textsc{topic} ‘girls’)
  
  \ex\label{ex:nikanne:26b}
  \gllll \textsc{focus}\oldstylenums{1}           \textsc{topic}              AgrS\\
    0     2   3                     4  5\\
    ketkä\textsubscript{i}     t\textsubscript{i}   eivät   olisi   istuneet   tuolilla?\\
    who.\textsc{pl}(\textsc{nom}) \textit{t}   not.\oldstylenums{3}\textsc{pl}   be.\textsc{cond}   sit.\textsc{ptc}.\textsc{pl}    chair.\textsc{ade}\\
  \glt   ‘Who(\textsc{pl}) would not have sat on the chair?’ \\
  (WH-WORD AS SUBJECT; \textsc{topic} ‘who’)
  
  \ex\label{ex:nikanne:26c}
  \gllll  \textsc{focus}\oldstylenums{1}           \textsc{topic}              AgrS\\
	  0     2   3                     4  5\\
  tuolilla\textsubscript{i}  tytöt   eivät   olisi   istuneet   t\textsubscript{i}.\\
  chair.\textsc{ade}  girl.\textsc{pl}(\textsc{nom}) not.\oldstylenums{3}\textsc{pl} be.\textsc{cond} sit.\textsc{ptc}.\textsc{pl} \textit{t}\\
  \glt ‘It is the chair that the girls would not have sat on.’ \\
  (FOCUS ON ‘on the chair’, \textsc{topic}: ‘girls’)

  \ex\label{ex:nikanne:26d}
  \gllll \textsc{focus}\oldstylenums{1}           \textsc{topic}              AgrS {} {} {}\\
    0     2   3                     4  5 {}\\
     eivät\textsubscript{i}  tytöt     t\textsubscript{i}  olisi   istuneet   tuolilla.\\
    not.\oldstylenums{3}\textsc{pl}  girl.\textsc{pl}(\textsc{nom})  \textit{t}   be.\textsc{cond}   sit.\textsc{ptc}.\textsc{pl} chair.\textsc{ade}\\
  \glt ‘It is not the case that the girls would have sat on the chair.’ \\
  (FOCUS ON NEGATION; \textsc{topic}: ‘girls’)
  \z
\z

\section{Conclusion} %6

In this article, the finite\is{finiteness} sentence of \ili{Finnish} is analyzed as a unit that consists of three tiers: finite\is{finiteness} sentence features, finite\is{finiteness} sentence \isi{morphology}, and finite\is{finiteness} sentence lexical categories. In addition, I have suggested that word order is based on a simple one dimensional tier that takes care of the linear order of different tiers.  The suggested system allows us to give up abstract syntax when it comes to constituent structure with functional categories and head movement. The Tiernet-model of finite\is{finiteness} sentence resembles traditional grammars: 

\begin{enumerate}
 \item[(i)] \isi{Morphological}\is{morphology/morphological} and lexical categories are kept apart.
 \item[(ii)] Finite\is{finiteness} features are very close to those assumed traditionally.
\end{enumerate}

Grammars of languages may differ at least in the following ways:

\begin{enumerate}
\item[(i)] The inventory of the finite\is{finiteness} features, lexical categories, and \isi{morphological} categories may be different in different languages.
\item[(ii)] The links between the features and the \isi{morphological} and lexical features  may differ in different languages.
\end{enumerate}

 
\section*{Abbreviations}
\largerpage Abbreviations used in this article follow the Leipzig Glossing Rules’ instructions for word-by-word transcription, available at: \url{https://www.eva.mpg.de/lingua/pdf/Glossing-Rules.pdf}. Additionally used:\largerpage
\begin{multicols}{2}
\begin{tabbing}
\textsc{ade} \= the adessive case ‘on’ / ‘at’\kill
 \textsc{ade} \> the adessive case ‘on’ / ‘at’\\
 \textsc{ptc} \> \isi{participle}
\end{tabbing}
\end{multicols}

\section*{Acknowledgements}
I would like to thank the editors Laura Bailey and Michelle Sheehan as well as the two anonymous reviewers for their valuable help with the article. The remaining mistakes are my own. The theory presented in this article would not have been possible without my good friend Anders Holmberg, with whom I have had the pleasure to work on the structure of the finite sentence of {Finnish}.

\printbibliography[heading=subbibliography,notkeyword=this]
\end{document}