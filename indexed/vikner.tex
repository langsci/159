\documentclass[output=paper]{LSP/langsci} 
\author{Sten Vikner\and Ken Ramshøj Christensen\lastand Anne Mette Nyvad\affiliation{Dept. of English, Aarhus University, Denmark}
}
\title{V2 and cP/CP}
\abstract{As in \citet{NyvadEtAl2016}, we will explore a particular derivation of (embedded) V2, in terms of a cP/CP-distinction, which may be seen as a version of the CP-recursion analysis (\citealt{deHaanWeerman1986,Vikner1995} and many others). The idea is that because embedded V2 clauses do not allow extraction, whereas other types of CP-recursion clauses do \citep{ChristensenEtAl2013escape,ChristensenEtAl2013processing,ChristensenNyvad2014}, CP-recursion in embedded V2 is assumed to be fundamentally different from other kinds of CP-recursion, in that main clause V2 and embedded V2 involve a CP (``big CP''), whereas other clausal projections above IP are instances of cP (``little cP'').}

\ChapterDOI{10.5281/zenodo.1117724}
\maketitle

\begin{document}



% [Warning: Draw object ignored][Warning: Draw object ignored][Warning: Draw object ignored][Warning: Draw object ignored][Warning: Draw object ignored][Warning: Draw object ignored][Warning: Draw object ignored]\title{V2\is{verb second} and \textit{c}P/CP}

% \textit{http://au.dk/en/sten.vikner@cc,}

% \textit{http://au.dk/en/krc@cc \& { http://au.dk/en/amn@cc}} 


% \section{Keywords}CP-recursion, embedded verb second (V2), extraction, islands, complementiser stacking

\section{ Introduction}

Verb second (V2\is{verb second}) has long been and continues to be a fascinating topic, as witnessed by articles and books all the way back to \citet{Wackernagel1892} and \citet{Fourquet1938} and up to \citet{Holmberg2015verbsecond}. 

This paper will briefly present an analysis of the CP-level in embedded clauses, including what is often seen as \isi{CP-recursion} in cases of embedded V2\is{verb second}. The analysis is discussed in much more detail in \citet{NyvadEtAl2016}.

We follow the suggestion in \citet{Chomsky2000} that syntactic derivation proceeds in phases and that the syntactic categories \textit{v}P and CP\is{complementizer} are phases. We also follow \citet{Chomsky2005,Chomsky2006} in taking Internal \isi{Merge} operations such as \isi{A-bar movement} to be triggered by an edge feature on the \isi{phase} head (in \citealt{Chomsky2000}, this feature is called a P(eripheral)-feature, in \citealt{Chomsky2001} a generalised \isi{EPP-feature}\is{EPP}). Below, this feature will be referred to as an OCC (``occurrence'') feature (following \citealt[18]{Chomsky2005}), which provides an extra specifier position that does not require feature matching. OCC offers an escape hatch allowing an element to escape an embedded clause. 

The availability of this generic edge feature OCC together with the availability of multiple specifier positions, however, in principle permits any element from within the \isi{phase} domain to move across a \isi{phase} edge, and so \isi{island} effects should not exist (as also observed by  \citealt[60--61]{Boeckx2012}).

If instead of multiple specifiers, \isi{CP-recursion} is possible, the \ili{Danish} data presented in the present paper may be captured in a uniform manner. We will explore a particular derivation of (embedded) V2\is{verb second}, in terms of a \textit{c}P/CP-distinction, which may be seen as a version of the \isi{CP-recursion} analysis (\citealt{deHaanWeerman1986,Vikner1995,Bayer2002,Walkden2016}, and many others). Because embedded V2\is{verb second} clauses do not allow \isi{extraction}, whereas other types of \isi{CP-recursion} clauses do \citep{ChristensenEtAl2013escape,ChristensenEtAl2013processing,ChristensenNyvad2014}, \isi{CP-recursion} in embedded V2\is{verb second} is assumed to be fundamentally different from other kinds of \isi{CP-recursion}: 

\ea%1
    \label{ex:vikner:1}
\gll { a CP\is{complementizer} with }  { V2\is{verb second} }  { (headed by a finite\is{finiteness} verb) }  { = }  { \textbf{CP} }  { (``big CP\is{complementizer}'')}\\
{ a CP\is{complementizer} without }  { V2\is{verb second} }  { (headed by a functional element)~ }  { = }  { \textbf{{c}}\textbf{P} }  { (``little {c}P'')}\\
\z
	



The idea is to attempt a distinction parallel to the \textit{v}P-VP distinction \citep[347]{Chomsky1995}, with \textit{c}P being above CP\is{complementizer} (cf. \citealt[148]{Koizumi1995} who posits a CP-PolP corresponding to our \textit{c}P-CP\is{complementizer}, and \citegen{deCuba2007} independent proposal that non-\isi{factive} verbs select a non-recursive\is{recursion} cP headed by a semantic\is{semantics} operator removing the responsibility for the truth of the embedded clause from the speaker).

\textit{c°} like \textit{v°} is a functional head, whereas C° like V° should be a lexical head. The latter admittedly only works partially, in that C° is only lexical to the extent that it must be occupied by a lexical category, i.e. a finite\is{finiteness} verb.

\section{C°}

\ea%2
    \label{ex:vikner:2}
    \begin{forest} for tree={nice empty nodes}
     [\textit{c}P
      [] [\textit{c}'
	  [\textit{c}°\\\textit{at}, base=top, align=left] [CP\is{complementizer}
	    [topic] [C'
	      [C°\\verb\textsubscript{[fin]},base=top,align=left] [IP [,name=empty1] [,name=empty2]
	      ]
	  ]	  
	  ]
     ]
    ]
    \draw (empty1) -- (empty2);
     \end{forest}
\z

\noindent Although CP-spec is the specifier position that attracts topics, also in embedded clauses, its associated head, C°, does not have a topic-feature ``in the ordinary way'', because \isi{verb movement} into C° would then erase that feature. The fact that C°'s topic feature is thus different from e.g. the way \textit{c}° can have a feature like \textit{wh} should be related to the fact that topicalisations are never selected for, i.e. there are verbs that select only embedded questions, but there are no verbs that select only embedded topicalisations (maybe not being selected is what allows \isi{verb movement} into C°, whereas being selected prevents movement into \textit{c°}\textsubscript{[}\textsc{\textsubscript{wh}}\textsubscript{]}). The closest we get are verbs that allow embedded topicalisations, but even such verbs never require them, e.g. \textit{vide} `know', \textit{tro} `think', etc.

Where we thus say that the C° associated with the specifier that attracts topics is deficient/unusual in not really having a topic-feature, e.g. \citet[146]{Julien2015} argues that the topic head is a normal head that may contain other things than finite\is{finiteness} verbs, e.g. \textit{så} `then' in contrastive left dislocations, (\ref{ex:vikner:3}a):


\ea%3
    \label{ex:vikner:3}
   
\ili{Danish}\\
 
    \glll a.  [\textsubscript{Topic-sp}   Hvis  man  ikke  kan  sige  noget   pænt,   { ] }  [\textsubscript{\isi{Topic}°}   så  { ] }   [\textsubscript{ForceP}   ~   [\textsubscript{\isi{Force}°}   skal  { ] }  man  tie  stille.]]\\  
	 b.   [{\textsubscript{c}}\textsubscript{P-spec}   Hvis  man  ikke  kan  sige  noget  pænt,  { ] }  ~  ~  ~   [\textsubscript{CP}   så   [\textsubscript{C°}  skal  { ] }  man  tie   stille.]]\\
  ~  {} {If}   one  not  can  say {smth.}  nice  ~  ~  {(then)}  ~  ~   {(then)}   ~  shall  ~  one  keep   {quiet}\\
\z

We take it that the fact that \textit{så} also occurs in the first position in V2\is{verb second} clauses with no dislocation means that it is a rather unlikely head element. We also hesitate to draw conclusions about the syntax of embedded V2\is{verb second} from contrastive left dislocations, as they are also perfectly possible in non-V2\is{verb second} embedded clauses (although we have no account for why this is strongly degraded in \ili{Swedish} and \ili{Norwegian}, cf. \citealt[407]{Johannesen2014}):

\ea%4
    \label{ex:vikner:4}
    \ili{Danish}\\
    \gll { Det }  { er }  { en }  { skam }  { at }  { den her   artikel }  { den }  { aldrig }  { er }  { blevet }  { udgivet.}\\
	 { {It} }  { {is} }  { {a} }  { {shame} }  { {that} }  { {this here article} }  { {it} }  { {never} }  { {is} }  { {been} }  { {published}}\\ 
    \z

	  

As topicalisations are never selected for, it follows that a topicalisation-CP\is{complementizer} (i.e. with a topic in CP-spec and with a verb moving into C°) cannot be the highest level of an embedded clause (in most Germanic languages, e.g. \ili{Danish} or English). Another level is necessary above CP\is{complementizer}, viz. a \textit{c}P with \textit{at/that} in \textit{c}° (though see the discussion at the end of section 4 below). It is this higher \textit{at/that} which prevents \isi{extraction} from CP-spec (as a kind of \textit{that-}\isi{trace} violation, perhaps derived in terms of anti-locality as in \citealt{Douglas2015concordia}), i.e. (\ref{ex:vikner:5}d):

  \largerpage 
  
\ea%5
\let\eachwordone=\small 
\let\eachwordtwo=\small
\small
    \label{ex:vikner:5}
\ili{Danish}\\
    \glllll  a. * ~ Sagde Andrea ~  Lego-filmen havde Kaj allerede set {\longrule}?\\
 b. ~ ~           Sagde Andrea at Lego-filmen havde Kaj allerede set {\longrule}?\\
 c. * Lego-filmen sagde Andrea ~  {\underline{\hspaceThis{Lego-filme}}} havde Kaj allerede set {\longrule}.\\
 d. * Lego-filmen sagde Andrea at {\underline{\hspaceThis{Lego-filmen}}} havde Kaj allerede set {\longrule}.\\
 ~ ~  (Lego-film.the) said Andrea (that) (Lego-film.the) had Kaj already seen\\ 
\z  
    
   
(Notice that (\ref{ex:vikner:5}c) is ungrammatical for the same reason as (\ref{ex:vikner:5}a): topicalisations cannot be selected, they must be inside a \textit{c}P.)

This is supported by \ili{German}, which for some reason allows embedded topicalisation without this higher \textit{that}, (\ref{ex:vikner:6}a), and which allows \isi{extraction} via CP-spec, (\ref{ex:vikner:6}c):

\ea%6 
  \let\eachwordone=\small 
  \let\eachwordtwo=\small
  \let\eachwordthree=\small
  \let\eachwordfour=\small
  \let\eachwordfive=\small
    \label{ex:vikner:6}
    \ili{German}\\     
    \glllll a. ~  ~   ~ Hat  Andrea  gesagt,  ~  {den Lego-Film}  hat  Kai  schon  {\longrule}  gesehen?\\
	  b. *  ~  ~ Hat  Andrea  gesagt,  dass  {den Lego-Film}  hat  Kai  schon  {\longrule}  gesehen?\\
	  c. ~  Den Lego-Film  hat  Andrea  gesagt,  ~  \underline{\hspaceThis{den Lego-film}}  hat  Kai  schon  {\longrule}  gesehen.\\ 
	  d.  *  Den Lego-Film  hat  Andrea  gesagt,  dass  \underline{\hspaceThis{den Lego-film}}  hat  Kai  schon  {\longrule}  gesehen.\\
	  e. ~  ({The} {Lego-film})  {has}  {Andrea}  {said}  ({that})  ({the Lego-film})  {has}  {Kai}  {already}  ~  {seen}\\
    \z 


CP\is{complementizer} may thus be a \isi{phase} in \ili{German}, and also in \ili{Danish} and English (where extractions via spec-CP\is{complementizer} are \textit{that}{}-\isi{trace} violations). From this, it would follow that CPs are strong \isi{islands} (cf. \citealt[111]{Holmberg1986}; \citealt[493ff]{MüllerSternefeld1993}; \citealt{SheehanHinzen2011}), provided there is no OCC escape hatch in CP\is{complementizer}, like the one suggested for \textit{c}P in \sectref{sec:vikner:3} below: 

\ea%7
    \label{ex:vikner:7}  
\let\eachwordone=\small 
\let\eachwordtwo=\small
\let\eachwordthree=\small
\small
    \ili{Danish}\\
    \glll a. ~   ~               Sagde Andrea at måske havde Kaj allerede set Lego-filmen?\\
	  b. *   Lego-filmen      sagde Andrea at måske havde Kaj allerede set {\longrule}?\\
	  {} ~  ({Lego-film.the}) {said} {Andrea} {that} {maybe} {had} {Kaj} {already} {seen} ({Lego-film.the})\\
    \z

	


\ea%8
    \label{ex:vikner:8}    
    \let\eachwordone=\small 
\let\eachwordtwo=\small
\let\eachwordthree=\small
\small
    \ili{German}\\
    \glll  a. ~   ~  Hat Andrea gesagt, vielleicht hat Kai {den Lego-Film} schon gesehen?\\
	  b. * {Den Lego-Film} hat Andrea gesagt, vielleicht hat Kai {\longrule}{\longrule} schon gesehen. \\
 ~ {} ({The Lego-film}) {has} {Andrea} {said} {maybe} {has} {Kai} ({the Lego-film}) {already} {seen}\\
    \z

	

A different approach that might explain the absence of an escape hatch could be to say that embedded V2\is{verb second} clauses are not really embedded at all, but instead there is a radical break/restart at the beginning of an embedded V2\is{verb second} clause, similar to what happens at the beginning of a new main clause (as argued e.g. by \citealt{Petersson2014}). Then \isi{extraction} out of an embedded V2\is{verb second} clause like (\ref{ex:vikner:7}b/\ref{ex:vikner:8}b) would correctly be ruled out, but this would also incorrectly rule out all other potential links across the edge of embedded V2\is{verb second} clauses (see also \citealt[157]{Julien2015}-159), so that e.g. the following c-command difference should not exist, as co-\isi{reference} should incorrectly be ruled out in both (\ref{ex:vikner:9}a) and (\ref{ex:vikner:9}b):

\ea%9
    \label{ex:vikner:9}
    
    \ili{Danish}\\

    \glll a. *  Han\textsubscript{1}   ~   sagde  at  [\textsubscript{CP}  den  her  bog  ville  Lars\textsubscript{1}  aldrig  læse.  ]\\
	b.  ~   Hans\textsubscript{1}  mor  sagde  at  [\textsubscript{CP}  den  her  bog  ville  Lars\textsubscript{1}  aldrig  læse.  ]\\
  ~ ~   {He/His}  {mum}  {said}  {that}   ~   {this}  {here}  {book}  {would}  {Lars}  {never}  {read}  \\
    \z

	  

Both (\ref{ex:vikner:9}a,b) would be expected to be just as impossible as such links across a main clause boundary:

\ea%10
\let\eachwordone=\small 
\let\eachwordtwo=\small
\let\eachwordthree=\small
\small
    \label{ex:vikner:10}  
     \ili{Danish}\\
    \glll a. *  {I går}  mødte  jeg  ham\textsubscript{1}   ~   i  bussen.   ~   [\textsubscript{CP}  Lars\textsubscript{1}  var  lige  blevet  forfremmet.  ]\\
	  b. *  {I går}  mødte  jeg  hans\textsubscript{1}  mor  i  bussen.   ~   [\textsubscript{CP}  Lars\textsubscript{1}  var  lige  blevet  forfremmet.  ]\\
  ~ ~   {Yesterday}  {met}  {I}  {him/his}  {mum}  {in}  {bus-the}   ~    ~   {Lars}  {had}  {just}  {been}  {promoted}  \\
    \z



\section{ \textit{c}° with OCC}
   \label{sec:vikner:3}
\ea%11
    \label{ex:vikner:11}
    \begin{forest} for tree={nice empty nodes}
     [\textit{c}P
     [t] [\textit{c}'
      [\textit{c}°\textsubscript{[OCC]}] [\textit{c}P\slash CP\is{complementizer}\slash IP
	[,name=e1] [,name=e2]
      ]
     ]
     ]
     \draw (e1) -- (e2);
    \end{forest}
\z

\textit{c}° can have a feature that may cause movement to \textit{c}P-spec, and such a feature can either be a so-called occurrence-feature or a slightly more standard type feature as e.g. a \textit{wh}{}-feature. (As mentioned above, for some reason C° cannot have an \isi{OCC-feature}.)

\citet[18--19]{Chomsky2005} suggests an OCC (``occurrence'') feature, which provides an extra specifier position ``without feature matching", i.e. the XP moves into the specifier of \textit{c°}\textsubscript{[}\textsc{\textsubscript{occ}}\textsubscript{]} without itself having an \isi{OCC-feature}. A \textit{c°}\textsubscript{[}\textsc{\textsubscript{occ}}\textsubscript{]} thus offers an escape hatch which allows an XP to escape an embedded clause. In fact only those XPs that move into a \textit{c}P-spec because of OCC will be able to move on, because they are the only XPs whose feature make up has not been altered/valued/checked as a result of the movement into \textit{c}P-spec.

\textit{c°}\textsubscript{[}\textsc{\textsubscript{occ}}\textsubscript{]} may be above another \textit{c}P, and then the \textit{c}P-layer headed by a \textit{c}° carrying an \isi{OCC-feature} is transparent to selection in the same way as e.g. NegP is in constituent \isi{negation} (e.g., \textit{she ate not the bread but the cake}) or quantificational layers (as in \textit{she ate all/half the cake}), cf. the notion of extended projections \citep{Grimshaw2005}. However, \textit{c°}\textsubscript{[}\textsc{\textsubscript{occ}}\textsubscript{]} may also be inside another \textit{c}P, in which case nothing further needs to be said.

\section{\textit{c°} with other features, e.g.\textit{ wh}}
% \begin{multicols}{2}
\vspace*{-2\baselineskip}\noindent\parbox{\textwidth}{\ea%12
    \label{ex:vikner:12}
    \begin{multicols}{2}
\ea \label{ex:vikner:12a}
\begin{forest} for tree={nice empty nodes}
[\textit{c}P 
  [\textit{wh}] [\textit{c}'
    [\textit{c}°\textsubscript{[WH]}] [\textit{c}P\slash CP\is{complementizer}\slash IP
      [,name=e1] [,name=e2]
  ]
]
] \draw (e1) -- (e2);
\end{forest}\\
\ex \label{ex:vikner:12b}
\begin{forest} for tree={nice empty nodes}
[\textit{c}P
  [OP] [\textit{c}'
    [\textit{c}°\textsubscript{[OP]}] [\textit{c}P\slash CP\is{complementizer}\slash IP
      [,name=e1] [,name=e2]
  ]
]
] \draw (e1) -- (e2);
\end{forest}
\z
\end{multicols}
\z}

We take the basic distinction between CP\is{complementizer} and \textit{c}P to be whether or not there is \isi{verb movement} into the head, but we want this to go hand in hand with other basic distinctions between the two, e.g. that C° is the potential host of the topic feature, whereas \textit{c}° is the relevant/necessary head for the outside context, e.g. as the highest head of embedded questions or of \isi{relative} clauses (= in the terms of \citealt[283]{Rizzi1997}, \textit{c}P is `facing the outside' whereas CP\is{complementizer} is `facing the inside').

In other words, we want to link the difference \textit{c}°/C° not just to individual features (much like the difference between different heads in the C-domain is linked to features in the cartographic approach, \citealt{Rizzi1997,WiklundEtAl2007,Julien2015,Holmberg2015verbsecond}...) – but we also want to link the difference to whether or not the head is the landing site of \isi{verb movement}. 

Spec-\textit{c}P\textsubscript{[}\textsc{\textsubscript{wh}}\textsubscript{]} in \REF{ex:vikner:12a} is where the \textit{wh}{}-phrase in an embedded \isi{question} occurs, and spec-\textit{c}P\textsubscript{[OP]} in \REF{ex:vikner:12b} is where we find the empty operator that may occur in e.g. \textit{som}{}-\isi{relative} clauses in \ili{Danish} (and in \textit{that}{}-\isi{relative} clauses in English).

It appears that a \textit{wh}{}-element that has moved into such a specifier cannot move on from here: 

\ea%13
    \label{ex:vikner:13}
    
    \ili{Danish}

    \glll a. {  }  {  }  {  }  { Spurgte }  { Andrea }  { [{\textsubscript{c}}\textsubscript{P} hvilken }  { film }  { {c°}\textsubscript{[}\textsc{\textsubscript{wh}}\textsubscript{]} }  { Kaj }  { allerede }  { havde }  { set]?}\\
	 b. {  * }  { Hvilken }  { film }  { spurgte }  { Andrea }  { [{\textsubscript{c}}\textsubscript{P} {\longrule}{\longrule}{\longrule} }  { {\longrule}\_ }  { {c°}\textsubscript{[}\textsc{\textsubscript{wh}}\textsubscript{]} }  { Kaj }  { allerede }  { havde }  { set]?}\\
     {} {  }  { ({Which} }  { {film}) }  { {asked} }  { {Andrea} }  {  ({which~~} }  { {film}) }  {  }  { {Kaj} }  { {already} }  { {had} }  { {seen}}\\
    \z

	 

This may be because the embedded clause in (\ref{ex:vikner:13}b) with an empty specifier and an empty \textit{c° }can no longer be identified as a \textit{wh}{}-clause, as is required of an object clause of the verb \textit{ask} (cf. clausal typing, \citealt{Cheng1991}). 

Following  { \citet[20]{RizziRoberts1996}, \citet[50]{Vikner1995}, \citet[412]{Grimshaw1997}, the reason why there can be no \isi{verb movement} into \textit{c°}\textsubscript{[}\textsc{\textsubscript{wh}}\textsubscript{]} is that this would change the properties of the selected head (i.e. \textit{c°}\textsubscript{[}\textsc{\textsubscript{wh}}\textsubscript{]}), and therefore this head would no longer satisfy the requirements of the selecting matrix expression. In fact, according to \citet[103]{McCloskey2006}, a head modified in this way (by movement into it) is not an item that could possibly be selected by a higher lexical head (it is not part of the ``syntactic \isi{lexicon}''), which would lead to the prediction that there could not be movement into heads of complements of lexical heads (which may very well be too strong, cf. that it would have consequences for many other cases, e.g. N°{}-to-D° movement in \ili{Scandinavian} would have to be something like N°{}-to-Num\is{number}° movement).} 

If, on the other hand, there is a \textit{c}P (with the declarative Complementizer\is{complementizer} \textit{at} in \textit{c}°) above the CP\is{complementizer} in which V2\is{verb second} takes place, then this problem does not arise. The selected clause is a \textit{c}P, its head is a \textit{c}° containing a complementiser, and the C° into which there is \isi{verb movement} is situated lower down inside the \textit{c}P. 

(Embedded topicalisations in \ili{German}, embedded questions in \ili{Afrikaans}, and embedded questions in some variants of English might be exceptions to the above in that they seem to have embedded V2\is{verb second} into the highest selected complementiser head. In such cases, an {\textquotedbl}invisible{\textquotedbl} \textit{c}P above the embedded V2\is{verb second} CP\is{complementizer} have been suggested, e.g. in \citet[101]{McCloskey2006} and in \citet[12--13]{Biberauer2015}. In fact, being inside such an {\textquotedbl}invisible{\textquotedbl} \textit{c}P might even be a possible analysis for those \ili{Danish} examples with embedded V2\is{verb second} but not preceded by \textit{at}, which do occur sometimes, e.g. (ii) in { \citet[55]{JensenChristensen2013}, although we find such examples ungrammatical.)} 

\section{ \textit{c}° without features}

\ea%14
    \label{ex:vikner:14}
\begin{forest} for tree={nice empty nodes}
 [\textit{c}P
  [] [\textit{c}' 
      [\textit{c}°\\\textit{at},base=top,align=left] [CP\is{complementizer}\slash IP
	[,name=e1] [,name=e2]
      ]
    ]
]\draw (e1) -- (e2);
\end{forest}

    
\z    

It is also possible for a \textit{c°} not to have any features, in which case no movement will take place into spec-\textit{c}P. This is possible both when such a \textit{c°} is the sister of an IP and the sister of a CP\is{complementizer}. 

\ea%15
    \label{ex:vikner:15}
    
\ili{Danish}\\
    \glll a.  ~   { Sagde }  { Andrea }  { at }   ~    ~   { Kaj }  { allerede }  { havde }  { set }  { Lego-filmen?}\\
	 b. ~   { Sagde }  { Andrea }  { at }  { Lego-filmen }  { havde }  { Kaj }  { allerede }   ~   { set? }  { }\\
  {} ~   { {Said} }  { {Andrea} }  { {that} }  { {(Lego-film.the)} }  { {(had)} }  { {Kaj} }  { {already} }  { {(had)} }  { {seen} }  { {(Lego-film.the)}}\\
    \z

	  

Because such an \textit{at/that} has no special features, it may also occur below other complementisers, when these are selected from above, e.g. below a \textit{wh}{}- or a \isi{relative} \textit{c}P-layer. As an extra complementiser, \textit{at} is preferred over other complementisers, which have more content:

\ea%16
    \label{ex:vikner:16}
  \ili{Danish} (Tom Kristensen, \textit{Livets Arabesk} (novel), 1921, cited in \citealt[III: 388]{Hansen1967}; in \citealt[122,  (149c)]{Vikner1995}; and in \citealt[368, (10)]{Nyvad2016}).\\  
    \gll { ... }  { hvis }  { at }  { det }  { ikke }  { havde }  { været }  { så }  { sørgeligt. }\\
	 {  }  { {if } }  { {that } }  { {it } }  { {not } }  { {had} }  { {been} }  { {so} }  { {sad}}\\     
    \z

\section{  Predictions concerning extraction}

The above suggestions (especially the OCC escape hatch in \textit{c}P) make the prediction that \isi{extraction} is possible almost everywhere (i.e. except topic \isi{islands}), which is much more general than usually assumed (including in \citealt{Vikner1995}). However, it turns out that such unexpectedly acceptable examples are fairly widespread, including extractions from \isi{relative} clauses:

\ea%17
    \label{ex:vikner:17}
    
     \ili{Danish} (\citealt[35]{ChristensenNyvad2014}, (13c,d))\\
\ea
    \gll  ~   { Pia }  { har }  { engang }  { mødt }  { en }  { pensionist }  { som }  { havde }  { sådan }  { en }  { hund.}\\
	 ~   { {Pia} }  { {has} }  { {once} }  { {met} }  { {a} }  { {pensioner} }  { {that} }  { {had} }  { {such} }  { {a} }  { {dog}}\\
\ex	 
\gll	 ~   { Sådan en hund\textsubscript{1} }  { har }  { Pia }  { engang }  { mødt }  { [\textsubscript{DP} }  { en }  { [\textsubscript{NP} }  { pensionist }  { ] }  { [{\textsubscript{c}}\textsubscript{P} {\longrule}\textsubscript{1}  {c°}\textsubscript{[}\textsc{\textsubscript{occ}}\textsubscript{]} }  { [{\textsubscript{c}}\textsubscript{P } }  { OP\textsubscript{2} }  { [{\textsubscript{c°}} }  { som }  { ] }  { [\textsubscript{IP} }  { {\longrule}\textsubscript{2} }  { havde }  { {\longrule}\textsubscript{1}.]]]]}\\
 ~   { {Such a dog} }  { {has} }  { {Pia} }  { {once} }  { {met} }   ~   { {a} }   ~   { {pensioner} }   ~    ~    ~    ~    ~   { {that} }   ~    ~    ~   { {had} }  { }\\

    \z
\z


... and extractions from embedded questions (\textit{wh}{}-\isi{islands}):

\ea%18
    \label{ex:vikner:18}
     \ili{Danish}  \citep[63]{ChristensenEtAl2013escape}\\
  \ea 
  \gll  {  }  { Hvilken båd\textsubscript{1} }  { foreslog }  { naboen }  { [{\textsubscript{c}}\textsubscript{P} {\longrule}\textsubscript{1}  {c°}\textsubscript{[}\textsc{\textsubscript{occ}}\textsubscript{]} }  { [{\textsubscript{c}}\textsubscript{P } }  { hvor billigt\textsubscript{2} }  { {c°}\textsubscript{[}\textsc{\textsubscript{wh}}\textsubscript{]} }  { [\textsubscript{IP} }  { vi }  { skulle }  { sælge }  { {\longrule}\textsubscript{1} }  { {\longrule}\textsubscript{2}.]]]}\\
{  }  { {Which   boat} }  { {suggested} }  { {neighbour.the} }  {  }  {  }  { {how cheaply} }  {  }  {  }  { {we} }  { {should} }  { {sell} }  {  }  { }\\
  \glt

  \ex
  \gll {  }  { Hvor billigt\textsubscript{2} }  { foreslog }  { naboen }  { [{\textsubscript{c}}\textsubscript{P} {\longrule}\textsubscript{2}  {c°}\textsubscript{[}\textsc{\textsubscript{occ}}\textsubscript{]} }  { [{\textsubscript{c}}\textsubscript{P } }  { hvilken båd\textsubscript{1} }  { {c°}\textsubscript{[}\textsc{\textsubscript{wh}}\textsubscript{]} }  { [\textsubscript{IP} }  { vi }  { skulle }  { sælge }  { {\longrule}\textsubscript{1} }  { {\longrule}\textsubscript{2}.]]]}\\
 {  }  { {How cheaply} }  { {suggested} }  { {neighbour.the} }  {  }  {  }  { {which   boat} }  {  }  {  }  { {we} }  { {should} }  { {sell} }  {  }  { }\\
  \z
\z
 
 


\ea%19
    \label{ex:vikner:19}
    
    \ili{Danish} (\url{http://ordnet.dk/ddo/ordbog?query=stads}, \citealt{HjorthKristensen20032005})\\

    \gll  {  }  { Om }  { morgenen }  { skulle }  { jeg }  { give }  { dem }  { medicinen, }  { noget }  { brunt }  { stads, }  { [{\textsubscript{c}}\textsubscript{P} OP\textsubscript{1} }  { som }  { [\textsubscript{IP} }  { jeg }  { ikke }  { ved }  { [{\textsubscript{c}}\textsubscript{P} {\longrule}\textsubscript{1}  {c°}\textsubscript{[}\textsc{\textsubscript{occ}}\textsubscript{]} }  { [{\textsubscript{c}}\textsubscript{P } }  { hvad\textsubscript{2}  {c°}\textsubscript{[}\textsc{\textsubscript{wh}}\textsubscript{]} }  { [\textsubscript{IP} }  { {\longrule}\textsubscript{1} }  { var }  { {\longrule}\textsubscript{2}.]]]]]}\\
 {  }  { {In} }  { {morning-the} }  { {should} }  { {I} }  { {give} }  { {them} }  { {medicine-the,} }  { {some} }  { {brown} }  { {stuff,} }  {  }  { {that} }  {  }  { {I} }  { {not} }  { {know} }  {  }  {  }  { {what} }  {  }  {  }  { {was} }  { }\\
    \z

... as well as extractions from adverbial clauses:

\noindent\parbox{\textwidth}{\ea%20
    \label{ex:vikner:20}
\ili{Danish} (Knud Poulsen, 1918, cited in \citealt{Hansen1967}, I: 110)\\
    \gll {  }  { ... }  { men }  { det\textsubscript{1} }  { bliver }  { han }  { så }  { vred }  { [{\textsubscript{c}}\textsubscript{P} {\longrule}\textsubscript{1}  {c°}\textsubscript{[}\textsc{\textsubscript{occ}}\textsubscript{]} }  { [{\textsubscript{c}}\textsubscript{P } }  { OP }  { [{\textsubscript{c°}} }  { når] }  { [\textsubscript{IP} }  { man }  { siger }  { {\longrule}\textsubscript{1}.]]]}\\
	 {  }  {  }  { {but} }  { {that} }  { {becomes} }  { {he} }  { {so} }  { {angry} }  {  }  {  }  {  }  {  }  { {when} }  {  }  { {one} }  { {says} }  { }\\
    \glt
    \z
}

\section{Conclusion}

We have presented an analysis of the CP-level in embedded clauses, including what is often seen as \isi{CP-recursion} in cases of embedded V2\is{verb second}. The analysis, which is discussed in much more detail in \citet{NyvadEtAl2016}, attempts to unify a whole range of different phenomena related to \isi{extraction} and embedding, while acknowledging that \isi{extraction} in \ili{Danish} is considerably less restricted than has often been assumed.

The \isi{CP-recursion} that takes place in syntactic environments involving movement out of certain types of embedded clauses seems to be fundamentally different from that occurring in embedded V2\is{verb second} contexts, and hence, we propose a \textit{c}P/CP\is{complementizer} distinction: The \isi{CP-recursion} found in complementiser stacking and long extractions requiring an \isi{OCC-feature} involves a \isi{recursion} of \textit{c}P, \REF{ex:vikner:21a}, whereas the syntactic \isi{island} constituted by embedded V2\is{verb second} involves the presence of a CP\is{complementizer}, \REF{ex:vikner:21b}. 

\begin{multicols}{2}
\ea \label{ex:vikner:21} %21
\ea \label{ex:vikner:21a}
    \begin{forest} for tree={nice empty nodes}
     [\textit{c}P
      [t\textsubscript{WH}] [\textit{c}'
	[\textit{c}°\textsubscript{[OCC]}] [\textit{c}P
	  [\textit{wh}\slash OP] [\textit{c}'
	    [\textit{c}°\textsubscript{[WH]\slash [OP]}] [IP
	      [,name=e1] [,name=e2]
	  ]
	]
      ]
    ]
   ] \draw (e1) -- (e2);
  \end{forest}
  
\ex   \label{ex:vikner:21b}
  \begin{forest} for tree={nice empty nodes}
   [\textit{c}P
    [] [\textit{c}'
      [\textit{c}°\\\textit{at},base=top,align=left] [CP\is{complementizer}
	[topic] [C'
	  [C°\\verb\textsubscript{[fin]}] [IP
	    [,name=e1] [,name=e2]
	  ]
	]
      ]
    ]   
   ] \draw (e1) -- (e2);
  \end{forest}
\z \z \end{multicols}

The exact structure of \isi{CP-recursion} may be subject to parametric variation: \ili{German} does not seem to allow \isi{CP-recursion} given that \isi{extraction} from embedded \textit{wh}{}-questions is ungrammatical irrespective of which function the extracted element has (unless it moves via spec-CP\is{complementizer}, (\ref{ex:vikner:6}c)), and that embedded V2\is{verb second} is in complementary distribution with the presence of an overt complementiser in C°. 

Whether a cartographic approach to the structure of the CP-domain in the \ili{Scandinavian} languages will turn out to be more appropriate than a \isi{CP-recursion} analysis (e.g. \citealt{Rizzi1997,WiklundEtAl2007,Julien2015,Holmberg2015verbsecond}), we will leave for future research to decide. Until we have data that support a fine-grained left periphery in the relevant structures in \ili{Danish}, the version of \isi{CP-recursion} as argued for here would appear promising, as it captures the data presented here while making perhaps slightly fewer stipulations than e.g. the cartographic approach or the multiple specifier analysis.

\section{Acknowledgements}

We are grateful to Laura Bailey, Theresa Biberauer, Constantin Freitag, Hans-Martin Gärtner, Johannes Kizach, Doug Saddy, Johanna Wood, and two anonymous reviewers for helpful comments and suggestions as well as to participants in the Grammar in Focus workshop series at the University of Lund, participants in the SyntaxLab talk series at the University of Cambridge, and participants in the DGfS conference in Konstanz, February 2016. 

Special thanks to Anders Holmberg not just for being such a role model when it comes to creating worldwide interest in {Scandinavian} syntax, but also for his help and support from  Trondheim 1984 to this very day. 

The work presented here was partly supported by \textit{Forskningsrådet for Kultur og Kommunikation} ({Danish} Research Council for Culture and Communication).
 
 

\printbibliography[heading=subbibliography,notkeyword=this]
\end{document}