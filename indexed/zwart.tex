\documentclass[output=paper]{LSP/langsci} 
\author{Jan-Wouter Zwart	\affiliation{University of Groningen}
}
\title{An argument against the syntactic nature of verb movement} 
% \epigram{Change epigram}
\abstract{Recent research into the nature of periphrasis converges on the view that periphrastic forms occupy cells in morphological paradigms. This paper argues that the relative past (``perfect'') in Dutch should be understood as periphrastic in this sense. Adopting the current minimalist view on the relation between morphology and syntax, in which inflectional morphemes are not generated in syntax but realized postsyntactically in a morphological component, the analysis leads to the conclusion that the relative past’s auxiliary is not an element of narrow syntax either. The paper argues that this approach simplifies the syntactic analysis of Dutch verb clusters. The upshot of the analysis is that since auxiliaries undergo verb-second, verb movement must be a postsyntactic operation, as suggested by \citet{Chomsky2001}.}

\ChapterDOI{10.5281/zenodo.1117746}
\maketitle

\begin{document}

 
%%please move the includegraphics inside the {figure} environment
%%\includegraphics[width=\textwidth]{a2Zwart-img1}

 
%%please move the includegraphics inside the {figure} environment
%%\includegraphics[width=\textwidth]{a2Zwart-img2}

 
%%please move the includegraphics inside the {figure} environment
%%\includegraphics[width=\textwidth]{a2Zwart-img3}

 
%%please move the includegraphics inside the {figure} environment
%%\includegraphics[width=\textwidth]{a2Zwart-img4}

 
%%please move the includegraphics inside the {figure} environment
%%\includegraphics[width=\textwidth]{a2Zwart-img5}
 
\section{Introduction}\label{sec:zwart:1}

Like many languages, \ili{Dutch} has two ways of expressing that an event took place in the past, a simple (synthetic) past and a periphrastic past, often mistakenly associated with perfective aspect\is{Aspect}. These are illustrated in \REF{ex:zwart:1}, where it can be observed that the periphrastic \isi{tense} involves an auxiliary\is{Auxiliary} (\textit{hebben} ‘have’ or \textit{zijn} ‘be’) and a past \isi{participle}, typically marked by a prefix \textit{ge}-.\footnote{The examples in \REF{ex:zwart:1} and \REF{ex:zwart:2} are all third \isi{person} singular.}

\ea%1
\label{ex:zwart:1}
\begin{tabular}[t]{llllll}
   & \textsc{verb} & & \textsc{simple past} & \multicolumn{2}{l}{\textsc{periphrastic past}}\\
a. & wandel-t & ‘walk’  &  wandel-de  &  heeft   &  ge-wandel-d\\
   & walk-\textsc{\oldstylenums{3}sg}  &  &  walk-\textsc{past.sg} &   \textsc{aux.\oldstylenums{3}sg}   &    \textsc{ge}-walk-\textsc{part}\\
b. & loop-t  &  ‘walk’ &   liep   &   heeft  &  ge-lop-en\\
   & walk-\textsc{\oldstylenums{3}sg}   &  &    walk:\textsc{past.sg} &    \textsc{aux.\oldstylenums{3}sg} &   \textsc{ge}-walk-\textsc{part}\\
c. & gebeur-t &  ‘happen’ & gebeur-de &   is &   ge-beur-d\\
& & & & \textsc{aux.\oldstylenums{3}sg} & \\
d. & kom-t  &  ‘come’  &  kwam   &   is   & ge-kom-en\\
e. &  ontdek-t  & ‘discover’ & ontdek-te &   heeft  &  ontdek-t
\end{tabular}
\z

\noindent The periphrastic past’s auxiliary\is{Auxiliary} itself may express the present \isi{tense}, as in \REF{ex:zwart:1}, or the (simple) past, yielding the opposition in \REF{ex:zwart:2}.

\ea%2
\label{ex:zwart:2}
\begin{tabular}[t]{lllll}
& \multicolumn{2}{l}{\textsc{periphrastic past (present)}}   & \multicolumn{2}{l}{\textsc{periphrastic past (past)}}\\
a. &  heeft  &   ge-wandel-d &    had  &  ge-wandel-d\\
   &  \textsc{aux.\oldstylenums{3}sg} &  \textsc{ge}-walk-\textsc{part}   & \textsc{aux.past.\oldstylenums{3}sg} & \textsc{ge}{}-walk-\textsc{part}\\
b.   &  heeft   & ge-lop-en  &  had &    gelopen\\
c.   &  is      & ge-beur-d     &  was    & gebeurd   \\
& \textsc{aux.\oldstylenums{3}sg} & & \textsc{aux.past.\oldstylenums{3}sg} & \\
d.   &  is      & ge-kom-en     &  was    & gekomen   \\
e.   &  heeft   & ontdek-t   & had  &   ontdekt \\
\end{tabular}
\z

\noindent The periphrastic past \isi{tense} locates the event in the past \isi{relative} to a \isi{reference} point, which may be in the present or in the past, and the \isi{tense} of the auxiliary\is{Auxiliary} refers to the position of the \isi{reference} point on the time axis. The choice of the synthetic or periphrastic past is independent of the telicity or the progressive/completed nature of the verb/event, showing that the distinction is one of (\isi{relative}) \isi{tense}, not aspect\is{Aspect}.\footnote{For further discussion see \citet[89]{Vendryes1937}, \citet[117]{Kiparsky2002}, \citet[20f]{Verkuyl2008}, \citet[12]{Zwart2011}, and \citet[108]{BroekhuisEtAl2015}.}

The simple past must be used to express cotemporaneity with a \isi{reference} point in the past. Making a cotemporaneous \isi{reference} point in the past explicit forces the use of the simple past, to the exclusion of all other \isi{tenses}:

\ea%3
\label{ex:zwart:3}\settowidth\jamwidth{(\ili{Dutch})}\jambox{(\ili{Dutch})}\vspace{-\baselineskip} % A little messy, but it works...
\begin{tabular}[t]{llllll}
   [  & Toen & ik & binnen  kwam ]  ...  &          & \\
      & when & I  & in      come.\textsc{past.sg} & & \\
   a. &      &  ... & sliep  &  hij & \\
      &      &      & sleep.\textsc{past.sg} &  he & \\
   b. & *    &  ... & slaap-t & hij & \\
      &      &      & sleep.\textsc{\oldstylenums{3}sg} & he & \\
   c. & *    &  ... & heeft   & hij & ge-slap-en \\
      &      &      & \textsc{aux.\oldstylenums{3}sg} &  he & \textsc{ge}{}-sleep-\textsc{part}\\
      \\
      \multicolumn{6}{l}{‘When I came in, he was asleep.’}\\
\end{tabular}
\z

We want to recall this as a test for a) the syntactic presence of a feature \textsc{past} and b) a \isi{morphological} effect correlated with the presence of the feature \textsc{past}. Assuming, as is common in current minimalism, that the \isi{morphological} component is fed by the syntactic derivation, the \isi{morphological} effect can be described as the selection of a particular form from the relevant paradigm, based on the features associated with the root \textsc{sleep} (cf. \citealt[428]{Halle1997}). In this example, the root \textsc{sleep} activates the paradigm of the \ili{Dutch} verb \textit{slapen} ‘sleep’, and the features \textsc{past} and \textsc{singular} associated with the root \textsc{sleep} serve to select the unique form \textit{sliep} from that paradigm.

The question how \isi{tense} and \isi{agreement} features come to be associated with a verb root has been approached in various ways throughout the history of generative grammar. I will assume a simple, \isi{minimalist} approach in which a syntactic structure contains a range of \textit{controllers} sharing their feature values with the verb (or, more exactly, with their sister constituents dominating the verb), where the feature sharing process is taken to be a function of \isi{Merge} (as defined in \citealt[3]{Chomsky2001}). Relevant controllers include the subject (for \isi{person} and \isi{number} features) and the \isi{tense} operator (typically described as a functional head T, but I will remain agnostic as to its syntactic status). It helps to think of the control relation as c-command (where α c-commands β iff α is merged with β or a constituent dominating β), itself a function of \isi{Merge}.\footnote{\citet{Epstein1999}. Note that the concept of feature sharing between a controller and its sister is different from the probe-goal Agree\is{agreement} mechanism of \citet[122]{Chomsky2000}. See \citet{Zwart2006} for more discussion.}

The advantage of this approach to inflectional \isi{morphology} is that no special mechanisms, such as Affix Hopping, \isi{verb raising}, or the operations proposed in the context of Distributed Morphology \citep{HalleMarantz1993,Embick2000}, need to be invoked to get the functional features to be associated with the verb. Feature sharing \citep{Koster1987} as a function of asymmetric \isi{Merge} \citep{Zwart2005} is all that is needed.

None of this brings us any closer to an understanding of the nature of the periphrastic \isi{tense}. In particular, it is not clear what the syntactic status of the auxiliary\is{Auxiliary} is, and as far as I am aware, this question has not often been explicitly addressed, at least not for \ili{Dutch}. I think most analyses implicitly take the auxiliary\is{Auxiliary} to be a defective verb, generated inside a more layered VP (following the lead of \citealt[20]{AkmajianEtAl1979} for English). Let us call this the \textsc{syntactic} approach (which allows for a range of variants, most notably generating the auxiliary\is{Auxiliary} in a functional head position), in which the auxiliary\is{Auxiliary} is an independently merged member of the Numeration (the set of elements feeding the syntactic derivation).

I can see at least two potential alternative approaches, which we might call \textsc{presyntactic} and \textsc{postsyntactic}. In the presyntactic approach, the auxiliary\is{Auxiliary} and the \isi{participle} would be syntactically merged in a separate derivation, yielding a cluster to be inserted as a Root into the Numeration for another derivation (the derivation generating the clause in which the periphrastic \isi{tense} features). In the postsyntactic approach, the derivation would contain just a single verb Root, and the auxiliary\is{Auxiliary} will not appear in the syntactic derivation at all; rather, the cluster would happen to fill a cell in the \isi{morphological} paradigm, in opposition to other (synthetic or periphrastic) members of the paradigm.

The postsyntactic approach is supported by research of the past fifteen years on the relation between inflection and \isi{periphrasis}, as I will show below. What I want to argue here is that this postsyntactic approach is also supported in that it yields the simplest syntactic derivation, needing no ad hoc mechanisms to complicate the general \isi{minimalist} procedure sketched above.

If so, I submit that this state of affairs provides an argument in support of \citegen[37]{Chomsky2001} conjecture that “a substantial core of head-raising processes (...) may fall within the phonological component”. In particular, since auxiliaries undergo verb-second in \ili{Dutch}, and auxiliaries are only introduced in the \isi{morphological} component, verb-second must be a postsyntactic process as well.

\newpage The discussion is organized as follows. \sectref{sec:zwart:2} discusses recent research in theoretical \isi{morphology} on the status of periphrastic expressions. \sectref{sec:zwart:3} addresses the question of the division of labor between syntax and \isi{morphology} in \isi{periphrasis}. \sectref{sec:zwart:4} argues that deriving the periphrastic past in \isi{morphology} sheds new light on a range of syntactic problems associated with verb clusters in \ili{Dutch}. \sectref{sec:zwart:5} concludes.

\section{Periphrasis and postsyntactic morphology}\label{sec:zwart:2} 

Let us continue to assume that \isi{morphology} is postsyntactic, i.e. inflectional affixes do not exist in syntax (neither in functional heads nor on lexical roots). Inflected words exist only in \isi{morphological} paradigms, which are accessed postsyntactically to find a spell-out for a syntactic terminal. What exists in syntax are roots and grammatical features, the latter instrumental in picking the right form from the paradigm.

The question to be asked here is the following: given that affixes do not exist in syntax, what evidence is there that the auxiliaries featuring in periphrastic \isi{tenses} exist in syntax? Until recently, the fact that auxiliaries undergo movements like verb-second could be taken as evidence that auxiliaries are syntactically present (cf. \citealt[203]{Embick2000}, \citealt[132]{Kiparsky2005}). But since \isi{verb movement} is at issue here (it might also be postsyntactic), we need evidence of a different kind.\footnote{This is where Chomsky’s observation that verb movements do not seem to feed the postsyntactic component dealing with interpretation becomes relevant; I will not address this line of argumentation, but see \citet{Holmberg2015verbsecond} for discussion.}

Recent research into the nature of \isi{periphrasis} leans heavily towards the alternative position, in which periphrastic forms occupy cells in \isi{morphological} paradigms (see \citealt{Chumakina2013} and \citealt{SpencerPopova2015} for a survey). The construction of a periphrastic expression, on this view, is not a matter of syntactic derivation any more than the formation of inflected word forms.

In the three following subsections, we discuss the key issues figuring in the discussion of \isi{periphrasis} in theoretical \isi{morphology}: the status of \isi{periphrasis} vis-à-vis \isi{morphological} paradigms, the compositionality of periphrastic expressions, and the process of auxiliation.

\subsection{Periphrasis and paradigms}

The idea that paradigms may include periphrastic formations appears to have been commonplace in structuralist linguistics (see e.g. \citealt[124]{Robins1959}, \citealt[130]{Benveniste19651974}). The thinking here is that paradigms “represent interlocking systems of grammatical oppositions” and where periphrastic expressions are “comparable to single words in the corresponding places of a different paradigm they are obviously to be included in paradigms themselves” \citep[124]{Robins1959}.\footnote{Robins says ``syntactically comparable'', the context suggesting that he has syntactic category and syntactic dependency in mind rather than syntactic position.}

This position was (silently) abandoned in the weak-lexicalist approach of early generative grammar, where even inflectional affixes constituted independent syntactic elements, generated in functional heads (e.g. \citealt[52]{Chomsky1981}). But in a strong-lexicalist approach (as adopted in minimalism, where inflected words are not created in syntax but introduced pre- or postsyntactically in fully inflected form), the question of paradigm structure resurfaces, and the idea that \isi{periphrasis} is part of the inflectional paradigm can be entertained once more.

This is reflected in the survey article by \citet[202f]{SpencerPopova2015}, referring to recent work by \citet{BörjarsEtAl1997}, \citet{Stump2001}, and \citet{AckermanStump2004}, among others, in which we find versions of the original structuralist position again. As before, the central idea is that paradigms are structured by the intersection of features expressed in the forms (e.g. combinations of \isi{person}, \isi{number}, \isi{tense}, voice, etc.). Each feature intersection defines a cell in the paradigm, which may be filled by a specific inflectional form, or, in its absence, by a periphrastic expression \citep[14]{Stump2001}.

This may be exemplified by \ili{Latin}, where the features \isi{tense} and voice intersect to yield the paradigm in \tabref{tab:zwart:1}. As is well-known, the cell where perfect \isi{tense} and \isi{passive} voice intersect cannot be filled by a synthetic form (forms in third \isi{person} singular, from the verb \textit{laud\=are} ‘praise’).

\begin{table}
\begin{tabular}{lcc}
 \lsptoprule
\scshape \isi{tense} & \multicolumn{2}{c}{\scshape voice}\\\cmidrule(lr){2-3}
& \scshape  active & \scshape \isi{passive} \\
 \midrule
 \scshape present & laudat  & laud\=atur \\
 \scshape imperfect & laud\=abat & laudab\=atur \\
 \scshape perfect & laud\=avit & laud\=atus est\\
 \lspbottomrule
\end{tabular}  
\caption{Latin tense/voice paradigm}
\label{tab:zwart:1}
\end{table}

Another example is provided by Burushaski \citep[243f]{Lorimer1935}, where even a single \isi{tense} paradigm can show a mix of synthetic and periphrastic forms, involving the verb \textit{ɛtʌs} ‘to do, to make’ and a form of the copula (\tabref{tab:zwart:2}; see also \citealt[9]{Chumakina2013} and references cited there).

  \begin{table}
\begin{tabular}{lcc}
 \lsptoprule
 \scshape \isi{person} & \multicolumn{2}{c}{\scshape number}\\\cmidrule(lr){2-3}
 & \scshape singular & \scshape plural \\
 \midrule
 1. &   εča ba & εča ba:n \\
 2. & εča & εča:n \\
 3.  \scshape hum.m & εčaii  & εča:n \\
 3.  \scshape hum.f & εču bo & εča:n \\
 3.  \scshape animate & εči bi  & εčiε(n) \\
 3.  \scshape inanimate & εči bi:la / εči:la  &  εčitsʌn\\
\lspbottomrule
\end{tabular} 
\caption{Burushaski present tense paradigm \citep[245]{Lorimer1935}}
\label{tab:zwart:2}
\end{table}

\noindent If periphrastic expressions are not allowed to fill the relevant cells in the \ili{Latin} and Burushaski paradigms, these paradigms would be randomly defective. Moreover, we would have to explain why syntax provides a periphrastic construction precisely there where these gaps in the paradigm happen to exist. Assuming that \isi{periphrasis} is syntactic and inflectional \isi{morphology} postsyntactic would lead us to the conclusion that \isi{periphrasis} somehow causes the gaps in the paradigm observed in Tables \ref{tab:zwart:1}--\ref{tab:zwart:2} (i.e. \isi{periphrasis} blocks inflection), the converse of what we typically find in blocking relations \citep{Kiparsky2005}.\footnote{Kiparsky solves this problem by assuming \isi{morphology} before syntax.}

In \ili{Dutch}, the relevant features are \textsc{tense} and some feature responsible for the \isi{relative} \isi{tense} interpretation (anteriority). We may follow \citet[75]{Wiltschko2014} in identifying this feature as \textsc{point of view}. Wiltschko calls the \isi{tense} feature \textsc{anchoring} and locates both \textsc{anchoring} and \textsc{point of view} as particular areas in the clausal spine, comparable with \isi{TP} (IP) and \isi{AspP} in current \isi{minimalist} analyses. Using the terminology introduced above, we may say that both\textsc{ tense} and \textsc{point of view} (\textsc{pov}) are potential controllers that may share features with the verbal root. The paradigm, then, is as in \tabref{tab:zwart:3} (cf. \ref{ex:zwart:1}).

\begin{table}  
\caption{Dutch finite paradigm (\textsc{3sg})} \label{tab:zwart:3}
\begin{tabular}{lcc}
 \lsptoprule
 \scshape \isi{tense} & \multicolumn{2}{c}{\scshape pov}\\\cmidrule(lr){2-3}
 & \scshape unmarked & \scshape anterior \\
 \midrule
 \scshape present & wandelt & heeft gewandeld\\
 \scshape past & wandelde & had gewandeld \\
 \lspbottomrule
\end{tabular}
\end{table}

The ``anterior present'' is what we described above as the \isi{relative} past: it locates the event prior to the here and now. The ``anterior past'', marked by the past \isi{tense} on the auxiliary\is{Auxiliary}, locates the event prior to a \isi{reference} point in the past (i.e. a past-shifted \isi{relative} past). As can be seen, the periphrastic expressions fill the cells where the \isi{tense} feature interacts with the anterior point of view feature.

A simple way to describe the situation in \ili{Dutch} would be to say that the operators \textsc{tense} and \textsc{pov} control the corresponding features on the verb, assigning them certain values pointing to particular cells in the paradigm in \tabref{tab:zwart:3}. That some of these cells are filled by periphrastic expressions is not a matter of syntax, but of \isi{morphology}.

\subsection{Compositionality}

There is a long tradition, going back to at least \citet{Benveniste19651974}, that treats the Indo-European periphrastic perfect, exemplified here by \ili{Dutch}, as non-compositional, in the sense that “the construction as a whole might be associated with morphosyntactic properties that do not arise from any of the component parts” \citep[211]{SpencerPopova2015}. To \citet{AckermanStump2004}, this is one of the diagnostic criteria for \isi{periphrasis} (for which they refer to Mirra Gukhman).

As \citet[184]{Benveniste19651974} argues, the auxiliary-\isi{participle} construction shows a clear division of labor (the auxiliary\is{Auxiliary} carrying inflection and the \isi{participle} conveying lexical meaning), but the grammatical property of anteriority arises only as a function of the combination of the auxiliary\is{Auxiliary} and the \isi{participle}.

In contrast, \citet[123]{Kiparsky2005} argues that the periphrastic (\isi{relative}) past \textit{is} compositionally derived from the meaning of its parts. This assumes that the past \isi{participle} contributes the meaning \textsc{past} (i.e. anteriority), and the auxiliary\is{Auxiliary} (through its \isi{tense} features) the location of the \isi{reference} point \isi{relative} to which the anteriority is to be interpreted. 

I am not convinced that the \isi{participle} denotes the past, as Kiparsky contends. In many languages, the same \isi{participle} appears in the \isi{passive} (with a different auxiliary\is{Auxiliary}), without a hint of anteriority (cf. \citealt[288--289]{Wackernagel1920}). Moreover, Kiparsky’s suggestion that the periphrastic \isi{tense} is compositional fails to specify the contribution made by the (\isi{possessive}) auxiliary\is{Auxiliary}, since the \isi{reference} point \isi{relative} to which the anteriority is to be interpreted is not derived from the presence and nature of the auxiliary\is{Auxiliary}, but from the \isi{tense} feature of the clause (spelled out by the auxiliary\is{Auxiliary}’s \isi{tense} \isi{morphology}).

More seriously, we can show that any compositionality that may have existed originally in the formation of the periphrastic past is very often lost as the periphrastic past became enshrined in the temporal/aspectual system of the language. As a result, closely related languages like \ili{Dutch}, \ili{German} and English show subtle differences in the grammatical properties of the auxiliary-\isi{participle} combination. 

In English, unlike \ili{Dutch}, the ``perfect time span'' in which the event is situated is not fully anterior, running up to and including the here and now \citep{IatridouEtAl2001}. This can be seen from the incompatibility of the periphrastic past (``perfect'') with time adverbials locating the event squarely in the past, like \textit{yesterday} (cf. \citealt{Klein1992,Zwart2008}):

% \todo{jambox has to be fixed}
\settowidth\jamwidth{(\ili{Dutch})}
\ea%4
\label{ex:zwart:4}
\ea  John (*has) read the book yesterday.
\ex \gll  Jan  heeft    het  boek  gisteren    ge-lez-en. \\
          John \textsc{aux.\oldstylenums{3}sg} the  book  yesterday  \textsc{ge}{}-read-\textsc{part}\\\jambox{(\ili{Dutch})}
    \glt ‘John read the book yesterday.’
\z\z
    
\noindent It is not so clear where this subtle but high-impact distinction between \ili{Dutch} and English participles originates, or what this instance of variation tells us about the core meaning of the past \isi{participle}.

\ili{German} is like \ili{Dutch} in this respect, but in large parts of the \ili{German} speaking area, the periphrastic past \isi{tense} has completely replaced the simple past, so that it can now be used to express cotemporaneity with a \isi{reference} point in the past (``Präteritumschwund'', cf. \citealt{AbrahamConradie2001}). Compare \ili{German} \REF{ex:zwart:5} with \ili{Dutch} (\ref{ex:zwart:3}c):

\settowidth\jamwidth{(\ili{German})}
\ea%5
    \label{ex:zwart:5}
\gll Als  ich  herein  kam    hat  er  ge-schlaf-en.\\
     when  I  in  come.\textsc{past.sg}  \textsc{aux.\oldstylenums{3}sg}  he  \textsc{ge}{}-sleep-\textsc{part}\\\jambox{(\ili{German})}
\glt  ‘When I came in he was asleep.’
\z

\noindent This additional shift in interpretation indicates that anteriority is not an inherent or stable property of the periphrastic \isi{tense}, casting doubt on the suggestion that the particular \isi{semantics} of the \ili{Dutch} \isi{relative} past derive compositionally from its \isi{morphological} component parts.

Moreover, as already observed for English in \citet[8]{Hoffmann1966}, the forced anteriority reading of the periphrastic \isi{tense} disappears in \ili{Dutch} nonfinite clauses \citep{Zwart2014}. This can be seen in (\ref{ex:zwart:6}), applying the past \isi{tense} diagnostics of \REF{ex:zwart:3}.

\ea%6
    \label{ex:zwart:6}
    \gll Hij  beweer-t  ...\\
	  he  claim-\oldstylenums{3}\textsc{sg}\\
    \glt  ‘He claims ...
\begin{xlista} 
\ex[]{
\gll    ...  te  slap-en.\\
     {} \textsc{inf}  sleep-\textsc{inf}\\
\glt     ... to be asleep.’}

\ex[]{\label{ex:zwart:6b}
\gll    ...  te  heb-ben    ge-slap-en  toen  ik  binnen  kwam.\\
     {} \textsc{inf}  \textsc{aux-inf}    \textsc{ge}{}-sleep-\textsc{part}  when  I  in  come:\textsc{past.sg}\\
\glt ... to have been asleep when I came in.’}

\ex[*]{
\gll  ...  te  slap-en    toen  ik  binnen  kwam.\\
      {} \textsc{inf}  sleep-\textsc{inf}  when  I  in  come:\textsc{past.sg}\\}
\end{xlista}
\z

In \REF{ex:zwart:6b}, making the \isi{reference} point in the past explicit (by \textit{toen ik binnen kwam} ‘when I came in’) forces a shift from the unmarked \isi{infinitive} \textit{te slapen} ‘to sleep’ to an \isi{infinitive} marked for past \isi{tense} \textit{te hebben geslapen} ‘to have been asleep’. But since \ili{Dutch} lacks a synthetic past \isi{tense} \isi{infinitive}, once again the periphrastic expression appears. If the periphrastic past’s anteriority reading were compositional, \REF{ex:zwart:6b} should have a forced anteriority reading as well, contrary to fact.\footnote{The infinitival periphrastic construction can also be used to express present and past anteriority.}

We can now supplement \tabref{tab:zwart:3} with its nonfinite counterpart in \tabref{tab:zwart:4}.

\begin{table}
\begin{tabular}{lcc}
\lsptoprule
 \scshape \isi{tense} & \multicolumn{2}{c}{\scshape pov}\\\cmidrule(lr){2-3}
 & \scshape unmarked    &  \scshape anterior \\
 \midrule
\scshape present & te wandelen   & te hebben gewandeld   \\
\scshape past &  te hebben gewandeld  & te hebben gewandeld   \\
\lspbottomrule
\end{tabular}
\caption{Dutch nonfinite paradigm}
\label{tab:zwart:4}
\end{table}

\noindent Like in southern \ili{German}, the periphrastic expression encroaches on the synthetic form, apparently ignoring whatever compositionality  (if any) gave rise to its formation in the first place.

The non-compositionality of the periphrastic past in \ili{Dutch} is consistent with the idea that the periphrastic past is a \isi{morphological} rather than a syntactic creation.

\subsection{Auxiliation}

The development of the periphrastic past \isi{tense} of the type discussed here is a textbook example of the process of grammaticalization (e.g. \citealt[57]{HopperTraugott1993}, \citealt[182f]{HarrisCampbell1995}, \citealt[40f]{Kuteva2001}). In the course of this process, a lexical verb of possession becomes an auxiliary\is{Auxiliary}, and what was initially a secondary predicate is reanalyzed as a participial main verb. A detailed discussion of this development is beyond the scope of this article, so we will assume our understanding of it to be by and large correct, noting the important refinements by \citet[86f]{Benveniste1968}.

What is striking is that the same development took place in many languages, and that its distribution can certainly not be explained as contact-induced propagation \citep[87--88]{Vendryes1937}. Moreover, what we find repeatedly is a push chain effect, shaking up the temporal/aspectual system of the language. Thus, \citet[88]{Benveniste1968} notes that the development of the periphrastic perfect in \ili{Latin} leads to a reinterpretation of the original synthetic perfect as an aorist. In other languages, the synthetic perfect has disappeared completely (\citealt[90]{Vendryes1937}, \citealt[149f]{Meillet1921}).

What these changes seem to indicate is that \isi{periphrasis} and synthesis are competing for the same turf. This follows naturally if \isi{periphrasis} is \isi{morphological}, but is somewhat unexpected if both processes, \isi{periphrasis} and synthesis, are in the different leagues of syntax and \isi{morphology}.

\section{Division of labor between syntax and morphology in periphrasis}\label{sec:zwart:3} 

\isi{Periphrasis}\is{periphrasis} is analyzed as a mix of \isi{morphology} and syntax in \citet{BrownEtAl2012}, with the aim of identifying a set of criteria to be employed for the proper characterization of apparent periphrastic phenomena in (ideally) any language. Using these criteria, we may decide where the \ili{Dutch} periphrastic \isi{tense} may be located in this morphosyntactic spectrum.

Criteria favoring \isi{morphological} character are (i) obligatoriness: the inevitable need to use a particular form in a particular morphosyntactic environment, (ii) expression of contextual rather than inherent features, (iii) the creation of a word \textit{form }rather than a (new) lexeme, and (iv) the expression of a paradigmatic opposition. All these criteria apply to the periphrastic past \isi{tense} in \ili{Dutch}: it (i) must be used to express \isi{relative} past, (ii) expresses \isi{tense}, a clausal feature, (iii) creates a (periphrastic) form of a word rather than a new lexeme, which (iv) enters into paradigmatic oppositions (see Tables \ref{tab:zwart:3} and \ref{tab:zwart:4}).

A fifth criterium listed by \citet{BrownEtAl2012}, that of (v) being complex, applies to both \isi{morphological} and syntactic formations, and indeed to the \ili{Dutch} periphrastic past as well.

Criteria favoring syntactic character are (vi) word order flexibility and (vii) allowing inflected subparts. These both apply to the \ili{Dutch} periphrastic \isi{tense}: the auxiliary\is{Auxiliary} (vi) need not be adjacent to the \isi{participle}, and appears on either side of the \isi{participle} (see below), and (vii) carries the clausal \isi{tense} and \isi{agreement} inflections. There is, however, a problem with these two criteria, as they serve to demarcate syntax from word formation, but not (necessarily) syntax from the formation of \isi{periphrasis}. Being composed of more than one word is in the very nature of \isi{periphrasis}, and it is not clear what would block the component words from undergoing processes of postsyntactic movement or inflection marking.

It seems, then, that the set of criteria identified in \citet{BrownEtAl2012} overwhelmingly points to \isi{periphrasis} being \isi{morphological}. 

This is not to deny that \isi{periphrasis} is complex and structured. A useful starting point is to assume that anything complex and structured is derived by \isi{Merge}, i.e. syntactically. But many complex and structured  items are clearly \isi{morphological}, such as \isi{compounds}\is{compound} and the products of derivational \isi{morphology}. Evidently, the \isi{morphological} inventory contains elements that are produced syntactically (see \citealt{AckemaNeeleman2004}), just like lexical items (roots) can be produced syntactically \citep{HaleKeyser2002}. 

However, when we say that a periphrastic item [α β] is syntactic, as opposed to \isi{morphological}, we mean that α and β, along with a range of other elements \{γ, ..., ω\}, are members of a single Numeration feeding a single derivation that yields the sentence composed of α, β, γ, etc. and ω. In other words, there is no separate syntactic subderivation in which [α β] is created, either before syntax (feeding the Numeration) or after syntax (feeding the \isi{morphological} paradigms), but the periphrastic expression is created ``on the fly'', during the derivation that yields the clause in which it appears. When I deny the syntactic status of \isi{periphrasis}, it is in this particular sense, in which levels of derivation that should be kept apart have been mixed.

This approach to the division of labor between syntax and \isi{morphology} is close to that of \citet{BörjarsEtAl1997}, which drew a sharp critique in \citet[223--224]{Embick2000}. Embick’s point seems to be that if cells in the \isi{morphological} paradigm can be filled by phrases created in a separate derivation, no predictions can be made about the nature and structure of those phrases. His own proposal holds that both the synthetic and the periphrastic perfect of \ili{Latin} are created in clausal syntax (thus mixing the levels of syntax and word formation in the tradition of weak lexicalism and Distributed Morphology; cf. \citealt{HalleMarantz1993,Halle1997}). It seems to me that the interesting part of this analysis can be made compatible with the \citeauthor{BörjarsEtAl1997} approach quite easily, whereas the part involving the syntactic derivation is considerably less compelling,\largerpage as shown by \citet[129f]{Kiparsky2005}.\footnote{\citet{Embick2000} argues that the synthetic and the periphrastic perfect in \ili{Latin} involve the same sets of features, distributed across identical syntactic structures, and that word formation is a function of syntactic movement, which is blocked in the periphrastic present by an opacity factor. The blocking is stipulated, but let us assume it to be correct. As far as I can tell, the generalizations of the analysis are not lost if the syntactic structure in fact describes a subderivation feeding into the \isi{morphological} paradigm, separate from the sentential syntactic derivation. Since the elements in the \isi{morphological} paradigm serve to express the features in the (sentential) syntactic terminals, some parallelism between the sentential syntactic and morpho-syntactic derivations is to be expected.  This also answers Embick’s objection that in a \citeauthor{BörjarsEtAl1997} (\citeyear{BörjarsEtAl1997}) type approach, anything goes. Clearly, for a phrase to obtain a position in an inflectional paradigm, some commonality in morphosyntactic features has to exist, which arguably requires some structural parallelism between the phrasal and inflectional elements as well (perhaps along the lines of \citegen{Williams2003} \textit{shape conservation}). Embick’s analysis goes a long way towards bringing such parallelisms to light, strengthening rather than weakening the lexicalist approach.}

\section{Further arguments}\label{sec:zwart:4} 

So far we have seen that there are reasons to consider the periphrastic past as a \isi{morphological} phenomenon, occupying a cell in the \isi{morphological} paradigm. Assuming postsyntactic \isi{morphology}, this entails that the auxiliary\is{Auxiliary} is only introduced after the narrow syntactic derivation has run its course.

It is important to note that this conclusion does not necessarily carry over to the other types of verb clusters in \ili{Dutch}, involving \is{Modal}modal auxiliaries \REF{ex:zwart:7a} or lexical verbs selecting infinitival complements \REF{ex:zwart:7b}.

\ea%7
\label{ex:zwart:7}
\ea \label{ex:zwart:7a}
\is{Modal}modal auxiliaries\\
\gll
  ...  dat  Tasman    het  Zuidland  wil      ontdek-ken\\
  ~ \textsc{comp}  Tasman    \textsc{def.ntr}  South.Land  \textsc{aux.volition.sg}  discover-\textsc{inf}\\
\glt   ‘... that Tasman wants to discover the South Land.’

\ex \label{ex:zwart:7b}
infinitival complements\\
\gll  ...  dat  Tasman    het  probeer-t  te  ontdek-ken\\
~  \textsc{comp}  Tasman    it  try-\textsc{\oldstylenums{3}sg}    \textsc{inf}  discover-\textsc{inf}\\
\glt   ‘... that Tasman is trying to discover it.’
\z
\z


Clusters with \is{Modal}modal auxiliaries or infinitival complement taking verbs are straightforwardly compositional and cannot be analyzed as occupying a cell in an otherwise inflectional \isi{morphological} paradigm. These clusters, therefore, must be thought of as being created either in narrow syntax or before that (i.e. in a separate derivation feeding the Numeration rather than the \isi{morphological} paradigms).

With this out of the way, we can briefly discuss a {number} of additional observations supporting the \isi{morphological} (postsyntactic) nature of the periphrastic past in \ili{Dutch}.

\subsection{Variability}
The verb cluster expressing the periphrastic past (i.e. consisting of a temporal auxiliary\is{Auxiliary} \textit{hebben} ‘have’ or \textit{zijn} ‘be’ and a past \isi{participle}) shows a remarkable variability in the order of its elements, both across dialects and within the standard language. Marking the auxiliary\is{Auxiliary} 1 and the \isi{participle} 2, both ascending (\textit{1-2}, auxiliary\is{Auxiliary}—\isi{participle}) and descending (\textit{2-1}, \isi{participle}—auxiliary\is{Auxiliary}) orders occur. This is different from the clusters featuring \is{Modal}modal auxiliaries and infinitive-taking lexical verbs, which are predominantly ascending (\textit{1-2}) across dialects and almost invariably ascending (\textit{1-2}) in the standard language (see \citealt{Stroop1970,Zwart1996}, and more recently \citealt[14--25]{BarbiersEtAl2008}).

Deriving the variable orders in the cluster syntactically poses a range of problems, giving rise to a diversity of analyses too wide to discuss here (but see \citealt{Wurmbrand2005}). Suffice it to say that existing proposals often must resort to ad hoc devices, such as optional movement, rightward movement, movement of intermediate projections, verb incorporation (``\isi{verb raising}'') and excorporation, reanalysis, and roll-up movement. None of this is necessary if the periphrastic past is a product of postsyntactic \isi{morphology}.

More particularly, the fact that the verb cluster in the periphrastic past behaves differently from the verb clusters headed by \is{Modal}modal auxiliaries and lexical verbs taking infinitives can now be ascribed to the circumstance that the periphrastic past is created postsyntactically, and the other verb clusters are not.

To some extent, the problem of how to account for variability in the order of the auxiliary\is{Auxiliary} and the \isi{participle} remains, but a large part of that problem, namely to describe the phenomena in terms of syntactic processes, has disappeared. And perhaps we may even ascribe the variation in linear order to the externalization process (``spell out\is{Spell-Out}''), reducing the problem of cluster generation simply to the merger\is{Merge} of an auxiliary\is{Auxiliary} and a \isi{participle} in a separate derivation feeding \isi{morphological} paradigms.

\subsection{The IPP-effect}

In three-verb clusters, where the highest verb is a temporal auxiliary\is{Auxiliary}, the second verb is realized as an \isi{infinitive} instead of as a past \isi{participle} (the \isi{Infinitivus Pro Participio} or IPP\is{Infinitivus Pro Participio} effect; \citealt{Lange1981,Zwart2007,Schallert2014}, among many others):

\ea%8
\label{ex:zwart:8}
\ea \label{ex:zwart:8a}
two-verb cluster, auxiliary\is{Auxiliary} selects \isi{participle}\\
\gll     heeft  \{   ge-wil-d / *wil-len  \}\\
\textsc{aux.\oldstylenums{3}sg} {}  \textsc{ge}{}-want-\textsc{part} / want-\textsc{inf} {}\\
\glt ‘wanted’
\ex \label{ex:zwart:8b}
three-verb cluster, auxiliary\is{Auxiliary} selects \isi{infinitive}\\
\gll     heeft  \{  *ge-wil-d / wil-len  \}  ontdek-ken\\
\textsc{aux.\oldstylenums{3}sg}  {}  \textsc{ge}{}-want-\textsc{part} / want-\textsc{inf} {} discover-\textsc{inf}\\
\glt     ‘wanted to discover’
\z
\z

While much about the IPP-effect remains unclear, the present approach suggests a new angle. Recall that we assume that the periphrastic past is created postsyntactically: in syntax, \REF{ex:zwart:8a} is just the verb \textit{willen} ‘want’ with \isi{relative} past (present anterior) features. In \REF{ex:zwart:8b}, however, the syntactic element to be replaced in \isi{morphology} is not a single verb but a cluster \textit{willen ontdekken} ‘want discover’. The generalization, then, would be as in \REF{ex:zwart:9}:\footnote{I am assuming that the replacement of the participial ending by the infinitival ending is a secondary effect, see \citet[85]{Zwart2007} following \citet[128]{Paul1920}.}

\ea%9
\label{ex:zwart:9}
\textit{IPP-effect}\\
The \isi{relative} past of \textit{x} is marked with \textit{ge-} only if \textit{x} is not a verb cluster.
\z

\noindent This is a \isi{morphological} generalization, referring to the inventory of forms and the processes generating them, and not to syntax.

One generalization about the IPP-effect follows immediately, namely the generalization that the IPP-effect is absent in dialects not marking the \isi{relative} past with \textit{ge}{}- \citep{Hoeksema1980,Lange1981}. More problematic, however, is the generalization that the IPP-effect is sensitive to linear order, clusters with strictly descending orders (\textit{3-2-1}) typically not showing the IPP-effect (though exceptions do exist, cf. \citealt[78f]{Zwart2007}). The following example is from Achterhoeks \ili{Dutch} \citep[76]{BlomHoekstra1996}.

\settowidth\jamwidth{(Achterhoeks \ili{Dutch})}
\ea%10
    \label{ex:zwart:10}
    \gll 	  ...  dat  ik  schriev-m  e-wil-d    had    \\
                  {} \textsc{comp}  I  write-\textsc{inf}  \textsc{ge}{}-want-\textsc{part}  \textsc{aux.past.sg}\\\jambox{(Achterhoeks \ili{Dutch})}
\glt   ‘... that I had wanted to write.’
\z


Further refinement of \REF{ex:zwart:9}, then, would still be needed, but it is not clear that this would put the entire approach in any kind of jeopardy. One possibility would be that in (10), \textit{schrievm} ‘write’ and \textit{willn} ‘want’ are separate terminals in syntax (so there is no presyntactic cluster formation), with only \textit{willn} marked with the present anterior features triggering periphrastic \isi{tense} formation in \isi{morphology}.\footnote{The IPP-variant is optional in \REF{ex:zwart:10}. This variant would then differ in involving presyntactic cluster formation, so that \REF{ex:zwart:9} applies.}

\subsection{Mixed cluster orders}
It has been observed that not all mixed cluster orders (\textit{1-3-2}, \textit{2-1-3}, \textit{2-3-1}, \textit{3-1-2}) are equally frequent across Continental \ili{West Germanic} dialects, though all are attested (see \citealt{Zwart2007} and \citealt{Salzmann2016} on the rare \textit{2-1-3} type). The {question} is whether this is what we expect to find if periphrastic \isi{tense} formation is postsyntactic.

If the temporal auxiliary\is{Auxiliary} (\textit{have} or \textit{be}) is the highest verb in the cluster, there are two possibilities. First, the \textit{2-3} verbs form a cluster, created presyntactically; in that case the cluster is transfered as a single terminal after completion of the narrow syntactic derivation, which is turned into a three-verb cluster in \isi{morphology}. The IPP-effect (9) applies, and we expect the mixed orders \textit{1-3-2} and \textit{2-3-1} to occur (in addition to the consistent ascending \textit{1-2-3} and descending \textit{3-2-1} orders). The other possibility, hinted at in section 4.2, is that the \textit{2} and \textit{3} verbs are independent elements in the syntactic derivation, only one of which (the \textit{2} verb) is turned into a periphrastic past in \isi{morphology}. The IPP-effect does not apply, as the relevant verb is not a cluster, and we expect the orders \textit{2-1-3} and \textit{3-1-2} to occur (\textit{2-1}/\textit{1-2} being the periphrastic element produced by \isi{morphology}). 

This is indeed what we find. Interestingly, all attested \textit{2-1-3} cases show no IPP-effect \citep{Zwart2007,Salzmann2016}. An example is given in (11) from Luxemburgish \citep[95]{Bruch1973}.

\settowidth\jamwidth{(Luxemburgish)}
\ea%11 
\let\eachwordone=\small
\let\eachwordtwo=\small
\label{ex:zwart:11}
    \gll 	  ...  ob-s    de  hollänesch  ge-leier-t  hues    schwätz-en  \\
   {}  comp.\textsc{int-\oldstylenums{2}sg}  you  \ili{Dutch}    \textsc{ge}{}-learn-\textsc{part}  \textsc{aux.\oldstylenums{2}sg}    speak-\textsc{inf}\\\jambox{(Luxemburgish)}
\glt     ‘... whether you learned to speak \ili{Dutch}.’        
\z


On the other hand, the \textit{3-1-2} order does show the IPP-effect (example from Austrian Bavarian, \citealt[278]{Patocka1997}; the IPP-verb is \textit{soin}, the \is{Modal}modal \isi{infinitive}):

\settowidth\jamwidth{(Austrian Bavarian)}
\ea%12
    \label{ex:zwart:12}
    \gll	  ...  da  ma  wås    lean-a    hett-n    soi-n\\
                  {}  \textsc{comp}  we  something  learn-\textsc{inf}  \textsc{aux.past-pl}  \textsc{mod-inf}\\\jambox{(Austrian Bavarian)}
    ‘... that we should have learned something.’      
\z


On our approach, this can only be explained if the position of the \textit{3} verb \textit{leana} ‘learn’ is due to a postsyntactic leftward shift process, breaking up the cluster \textit{soin leana}.

  If the temporal auxiliary\is{Auxiliary} is the {number} \textit{2} verb, we again have to consider the two possibilities of presyntactic cluster formation, yielding a single terminal at the end of the narrow syntactic derivation, and the alternative derivation in which both verbs are independent syntactic elements. Let us assume that the {number} \textit{1} verb is a \is{Modal}modal auxiliary\is{Auxiliary} selecting an \isi{infinitive} (ultimately the {number} \textit{3} verb). Then assuming presyntactic cluster formation, the periphrastic past can only occur if the \isi{infinitive} has an independent \isi{tense} feature (present anterior), not a straightforward possibility, but let us proceed. This yields a cluster \textit{2-3} after syntax, which we predict to have to stay together, barring the postsyntactic leftward shift of the number \textit{3} verb needed for (12). This gives us \textit{1-3-2}, \textit{2-3-1} and \textit{3-1-2} (after leftward shift), but not \textit{2-1-3}. If the \is{Modal}modal and the \isi{infinitive} are both independently present in syntax, and the \isi{infinitive} gets an independent \isi{tense} feature (present anterior), the result is the same. 

  It is interesting, then, to note that all the \textit{2-1-3} orders I have seen in dialect descriptions and in the theoretical literature involve a number \textit{1} temporal auxiliary\is{Auxiliary}, none involving a number \textit{1} \is{Modal}modal auxiliary\is{Auxiliary} and a number \textit{2} temporal auxiliary\is{Auxiliary} (see also \citealt[271]{Schallert2014}). This follows from the analysis contemplated here, where the periphrastic past is produced in postsyntactic \isi{morphology}, assuming that the cluster thus created can only be broken up by postsyntactic leftward shift of the \isi{participle} (itself a generalization in need of explanation).\footnote{Three-verb clusters in which the number \textit{1} verb is not an auxiliary\is{Auxiliary} show the same pattern as the three-verb clusters with a \is{Modal}modal auxiliary in the number \textit{1} position, with \textit{3-1-2} allowed but \textit{2-1-3} excluded, and I would suggest an explanation along similar lines, except that the pattern also applies where the number \textit{2} verb is not a temporal but a \is{Modal}modal auxiliary. These patterns, then, do not bear directly on the proposed analysis of periphrastic \isi{tense}.}

\subsection{Auxiliary selection}

A well-known property of the periphrastic past is that the nature of the temporal auxiliary\is{Auxiliary} may vary, with verbs ``selecting'' as the auxiliary\is{Auxiliary} to be used either a copula (\textit{be}, \ili{Dutch} \textit{zijn}) or a \isi{possessive} verb (\textit{have}, \ili{Dutch} \textit{hebben}). Basically, auxiliary\is{Auxiliary} \textit{be} is selected by unaccusative and \isi{passive} verbs, and \textit{have }is selected by active transitive and unergative verbs. Some verbs may select both \textit{have} and \textit{be}, but as \citet{Hoekstra1984} has shown, the variation is not random, as these verbs can be construed in different ways, featuring either unaccusative or unergative syntax.\is{ergative}

  I am assuming here that the discussion in \citet{Hoekstra1999} is essentially correct, showing that auxiliary\is{Auxiliary} selection is not a function of a lexical mutativity feature \citep{Kern1912}, but of syntactic structure. But not assuming postsyntactic \isi{morphology}, Hoekstra describes the auxiliary\is{Auxiliary} \textit{have} as being created in syntax through movement of a functional head specific to transitive and unergative structures into Infl (see \citealt{Kayne1993} for a related proposal). 

  Assuming postsyntactic \isi{morphology}, this analysis can be simplified in that we may adopt the syntactic structures without the hypothesized movements. The more complex structure associated with transitivity (including unergative constructions) may involve controllers imparting additional features on the verb, to be spelled out in \isi{morphology}.

  The advantage of this approach would be that it leaves room for \isi{morphological} idiosyncrasies, which we know are quite frequent in this domain. For example, languages featuring the periphrastic past do not always show the same auxiliary\is{Auxiliary} selection pattern, as is immediately clear from the example of English (using \textit{have} systematically). Realization of the auxiliary\is{Auxiliary}, then, cannot be an automatic function of syntactic structure.

  This is nowhere more apparent than in the selection of the auxiliary\is{Auxiliary} for the copular verb itself. The syntactic analysis of \citet{Hoekstra1999} here predicts the auxiliary\is{Auxiliary} to have to be \textit{be}, as it is in \ili{Dutch}, but many languages employ \textit{have} instead. The crosslinguistic pattern has been studied in \citet{Postma1993}, who derives the surprising generalization that selection of the auxiliary\is{Auxiliary} \textit{be} in this domain is determined by the presence of suppletive \isi{morphology} in the participial form of the copula. This is exemplified in \tabref{tab:zwart:5} for the closely related languages \ili{Dutch} and Frisian.\footnote{Postma bases his generalization on 19 Indo-European languages.}

\begin{table} 
\begin{tabularx}{\textwidth}{lllll}
\lsptoprule
\scshape language & \scshape copula (\isi{infinitive}) & \scshape periphrastic \isi{tense} & \scshape auxiliary\is{Auxiliary} & \scshape suppletion\\
\midrule
\ili{Dutch}   & zij-n   & is ge-wees-t  & be     &  + \\
Frisian & wêz-e   & ha wes-t      & have   &  --   \\
\lspbottomrule
\end{tabularx}
\caption{Auxiliary selection}
\label{tab:zwart:5}
\end{table}

Postma proposes a syntactic analysis of this generalization, which space does not permit me to discuss more fully here. But if suppletion, a hallmark of inflectional \isi{morphology}, determines auxiliary\is{Auxiliary} selection, then auxiliary\is{Auxiliary} selection must be \isi{morphological} too. And if \isi{morphology} is postsyntactic, as we have been assuming throughout, then auxiliary\is{Auxiliary} selection must be postsyntactic, too.

\section{Conclusion}\label{sec:zwart:5} 

In this article, I have argued that if \isi{morphology} is postsyntactic, as in current minimalism, the periphrastic \isi{tense} must be thought of as a product of \isi{morphology} rather than syntax.

  We have seen that the periphrastic past in \ili{Dutch} occupies a cell in the verbal paradigm, as illustrated in \tabref{tab:zwart:3} for the finite\is{finiteness} paradigm and in \tabref{tab:zwart:4} for the nonfinite paradigm. The ``meaning'' of the periphrastic \isi{tense} is not compositionally derived from its component parts (following \citealt{Benveniste19651974}, \textit{pace} \citealt{Kiparsky2005}), witness the shifts in interpretation that the periphrastic \isi{tense} displays across languages and dialects, sometimes even replacing the simple past, as in Southern \ili{German} dialects and in nonfinite contexts more generally. While the auxiliation process giving rise to the periphrastic \isi{tense} has been described as syntactic reanalysis, the fact that the periphrastic \isi{tense} effectuates a reorganization of a language’s temporal/aspectual system shows that the process is really \isi{morphological}.

  We have applied the diagnostic criteria of \citet{BrownEtAl2012} to show that the periphrastic \isi{tense} of \ili{Dutch} is a \isi{morphological}, rather than a syntactic phenomenon. This is not to deny that the fine structure of the periphrastic \isi{tense} formation may parallel the structure of the clause in which the periphrastic \isi{tense} appears, as observed by \citet{Embick2000}, but rather than taking the (imperfect) parallel as evidence for a syntactic derivation of the periphrastic \isi{tense}, the similarity must be ascribed to the need for the products of \isi{morphology} to externalize the features accrued in the syntactic derivation.

  We have shown that taking this perspective on the periphrastic \isi{tense} casts new light on several curious aspects of \ili{Dutch} verbal syntax, including the IPP-effect, generalizations about the order of elements in the verb clusters, and auxiliary\is{Auxiliary} selection.

  The upshot of the discussion is this. If the periphrastic \isi{tense} is a product of postsyntactic \isi{morphology}, the auxiliary\is{Auxiliary} that we observe in the periphrastic \isi{tense} does not exist in syntax. Yet the auxiliary\is{Auxiliary}, when finite\is{finiteness}, invariably undergoes \isi{verb movement} (‘verb-second’) to the position to the immediate right of the first clausal constituent in \ili{Dutch} main clauses. This movement, then, cannot take place in narrow syntax, but must be postsyntactic. Since all finite\is{finiteness} verbs in main clauses are subject to the same linearization restriction, all of verb-second must be postsyntactic. And since verb-second represents a core case of head movement, a case can be made for the postsyntactic nature of head movement more generally.
 
  

\printbibliography[heading=subbibliography,notkeyword=this]
\end{document}