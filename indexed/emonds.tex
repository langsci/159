\documentclass[output=paper]{LSP/langsci} 
\author{Joseph Emonds\affiliation{Palacky University, Olomouc}}
\title{Theoretical limits on borrowing through contact; not everything goes} 
% \epigram{Change epigram}
\abstract{The traditional derivation of \ili{Middle English} (\textsc{me}) from Old English (\textsc{oe}) is highly problematic.

\begin{itemize}
\item Essentially no \ili{Scandinavian} \isi{borrowing} in \textsc{oe}  (\citealt{Baugh2002}; \citealt{Strang1970}).
\end{itemize}

\begin{itemize}
\item In a short period (1130--1200) when English wasn’t written, most \textsc{oe} vocabulary was lost. 
\item The earliest \textsc{me} texts (from 1200, \textit{Ormulum}) are the first to contain numerous daily life \ili{Scandinavian} “borrowings”.
\item Roughly \textit{half of the \textsc{me} grammatical lexicon} is cognate with Old \ili{Norse}.
\item \textsc{me} syntax shares with North Germanic (\textsc{ng}) over 20 syntactic properties that \ili{West Germanic} (\textsc{wg}) lacks  (\citealt{EmondsFaarlund2014}).
\end{itemize}

From these facts, these authors conclude that \textsc{me} is an \textsc{ng} language “Anglicized \ili{Norse}” (\textsc{an}) with many \textsc{oe} borrowings rather than the other way around. This paper proposes to strengthen this hypothesis by arguing on theoretical grounds that \textit{several \textsc{ng} syntactic properties of \textsc{me} could not have been \isi{borrowed} from \textsc{ng}}. They must have resulted from internal developments in \textsc{an}. That is, the paper justifies a hypothesis that limits the t\textit{ype} of morpho-syntax that can be \isi{borrowed} via language contact.}

\ChapterDOI{10.5281/zenodo.1117698}
\maketitle

\begin{document}

\section{Introduction}\label{sec:emonds:1} 



% \textbf{Yellow highlight, added for review A}\textbf{;} \textbf{blue added for review B}\textbf{,} \textbf{green for review C}\textbf{”}

The traditional derivation of \ili{Middle English} (\textsc{me}) from Old English (\textsc{oe}) is highly problematic.


\begin{itemize}
\item Essentially no \ili{Scandinavian} \isi{borrowing} in \textsc{oe}  (\citealt{Baugh2002}; \citealt{Strang1970}).
\end{itemize}

\begin{itemize}
\item In a short period (1130--1200) when English wasn’t written, most \textsc{oe} vocabulary was lost. 
\item The earliest \textsc{me} texts (from 1200, \textit{Ormulum}) are the first to contain numerous daily life \ili{Scandinavian} “borrowings”.
\item Roughly \textit{half of the \textsc{me} grammatical lexicon} is cognate with Old \ili{Norse}.
\item \textsc{me} syntax shares with North Germanic (\textsc{ng}) over 20 syntactic properties that \ili{West Germanic} (\textsc{wg}) lacks  (\citealt{EmondsFaarlund2014}).
\end{itemize}

From these facts, these authors conclude that \textsc{me} is an \textsc{ng} language “Anglicized \ili{Norse}” (\textsc{an}) with many \textsc{oe} borrowings rather than the other way around. This paper proposes to strengthen this hypothesis by arguing on theoretical grounds that \textit{several \textsc{ng} syntactic properties of \textsc{me} could not have been \isi{borrowed} from \textsc{ng}}. They must have resulted from internal developments in \textsc{an}. That is, the paper justifies a hypothesis that limits the t\textit{ype} of morpho-syntax that can be \isi{borrowed} via language contact. 

\citet{EmondsFaarlund2014} discuss several \textsc{ng} constructions in \textsc{me} that are not instances of single lexical items. The first four below are well attested in earliest Mainland \textsc{ng} and \textsc{me}.


\begin{enumerate}[label={(\alph*)}]
\item A full system of \textit{post-verbal} \textit{\isi{directional} and aspectual particles}, contrasted with \textsc{wg} pre-verbal separable prefixes.
\item \textit{Preposition stranding}, including in sluicing (\textit{who with, what about}, etc.)
\item Unmarked \textit{head-initial word order in \textsc{vp}s}, in both main and dependent clauses.
\item \textit{Subject and object raising}, absent in in both \textsc{oe} and \textsc{wg} generally.
\item \textit{Parasitic gaps}, freely formed only in \textsc{ng}; restricted or absent in \textsc{wg}. 
\item \textit{Tag questions}, based on syntactic copies of Subjects and Tense\is{tense} in \textsc{ng} but not in \textsc{wg}.
\end{enumerate}

It is difficult to see how these properties, taken as changes from \textsc{oe} to \textsc{me}, could be “borrowings” of lexical items via language contact of \textsc{oe} with \ili{Norse} in England. Whole classes of verbs and prepositions would have to be \isi{borrowed} \textit{en masse}. The properties in italics satisfy no independent sense of “lexical item.” Though these all are properties particular to \textsc{ng} (none seem to be Indo-European), there is no clear evidence that any have ever been \isi{borrowed} by contact into or from neighbouring Germanic, \ili{Romance}, or \ili{Slavic} languages. At one time in the past, these constructions have developed, \textit{internal to \textsc{ng}}. Consequently, properties (a-f) could not have entered \textsc{me} through simple contact with \ili{Scandinavian} even in a rapidly evolving \textsc{oe} language. Properties (a-f) rather testify to an unchanging \textsc{ng} character of \textsc{me}. 


\section{A-theoretical perspectives on language contact}\label{sec:emonds:2}

\ili{Middle English} (\textsc{me}), in contrast to Old English (\textsc{oe}), has many words of \ili{Scandinavian} origin, conventionally attributed to language contact and \isi{borrowing}. However, recourse to such an account doesn’t stand up to even moderate scrutiny. Traditional scholarship, e.g. \citet{Campbell1959} and \citet{Strang1970}, locates the great bulk of this \isi{borrowing} from c. 1170 onwards, starting with the \textsc{me} period, while in the \textsc{oe} period, hardly any \ili{Scandinavian} words were \isi{borrowed} \citep[99]{Baugh2002}. Yet, the \ili{Scandinavian} language in England, of which there are no records in the \textsc{oe} period, is taken to have died out by 1150 (\citealt{Thomason2016}; \citealt{Baugh2002}: 96), before serious \isi{borrowing} from it even started. Hence this \isi{borrowing} can’t be ascribed to contact, at least contact with the living.

This inconsistent dating has been a fertile source for creative sociolinguistic scenarios, although no facts actually confirm these speculations in the contact literature.\footnote{One variant (\citealt[286--287]{ThomasonKaufman1988}) proposes that when in the 10\textsuperscript{th} c. \ili{Norse} died out in various areas, Anglo Saxon speakers introduced numerous aspects of \ili{Norse} grammar and phonology into (unattested) \textsc{oe} dialects such as ``East Mercian'' and ``East Saxon''. These ``packets'' of dozens of ``\ili{Norse} grammatical elements'' then spread southward and westward to the whole country, ultimately resulting in ``Norsified English'' (i.e. Early \textsc{me}). The authors devise this \textit{sui generis} scenario of contact to account for why, a century or more later, these diverse Norsification features first appear as a group in written \textsc{me}: “These features of Simplification and Norsification … did not appear gradually; \textit{they appear in the earliest \ili{Middle English} documents of the \isi{Danelaw}.”} (op. cit. 278–279, my emphasis).   In place of elaborate contact scenarios,  \citet{EmondsFaarlund2014} claim that \textsc{me} but not \textsc{oe} was a development of North Germanic \ili{Norse}, eventually learned as a second and then first language by all Anglo-Saxons. Consequently, much \textsc{me} grammatical vocabulary and morphophonology was \ili{Norse}.} A centre piece in such thinking is often some kind of ``spoken Old English'' (there are no texts) which must have \isi{borrowed} extensively from the English variant of \ili{Scandinavian} before the latter died out. Subsequently, these extensive borrowings, including much daily life vocabulary, suddenly came to light in written Early \textsc{me}. 

Moreover, this (allegedly \isi{borrowed}) \ili{Scandinavian} vocabulary was not limited to content words, counter to normal contact situations, as was the massive influx of \ili{French} words into Late \textsc{me} when \ili{French} speakers in England all switched to English as their first language (14\textsuperscript{th} c.). The fact is, not only did hundreds of daily life terms in \textsc{me} have a  known \ili{Scandinavian} origin, so also did roughly half of its grammatical \isi{lexicon} (For more on how this component differs from the open class \isi{lexicon}, consult \citealt{Ouhalla1991} and \citealt[Chs. 3 and 4]{Emonds2000}). Something other than ``\isi{borrowing} through contact'' must have transpired, not only because of dating but also because of the types of ``\isi{borrowed}'' words and morphemes.

\section{The importance of Middle English syntax}\label{sec:emonds:3} 
  Another discrepancy between \textsc{oe} and \textsc{me} is the key to understanding all these puzzles. If one assembles the data patterns of \textsc{me} syntax, the language groups typologically with North Germanic (\textsc{ng}), while \textsc{oe} has unmistakable \ili{West Germanic} (\textsc{wg}) syntax (\citealt[133]{GianolloEtAl2008}). On the basis of such patterns, \citet{EmondsFaarlund2014} argue that \textsc{me} shares with \ili{Mainland Scandinavian} more than 20 syntactic properties that \textsc{oe} and other \textsc{wg} languages lack. They conclude \REF{ex:emonds:1}:

\ea \label{ex:emonds:1}%bkm:Ref376546782
 

      \textbf{English as North Germanic.} \ili{Middle English} was a direct descendent of the Mainland \ili{Norse} spoken by \ili{Scandinavian} settlers in England.
    \z

The presence in \textsc{me} of \ili{Norse} morphophonology and daily life and grammatical vocabulary is thus explained. The familiar facts that this hypothesis now makes strange are the daily life and grammatical vocabulary of \textsc{oe} found in \textsc{me}; because of this factor, Emonds and Faarlund call Early \textsc{me} by the name “Anglicized \ili{Norse}”. Under this view, in the realm of syntax there is basically nothing to explain, since, they argue, \textsc{me} \textit{shares no syntax or morpho-syntax with \textsc{oe}} that is not common to Germanic languages in general.

According to this study’s guiding hypothesis \REF{ex:emonds:1}, \textsc{me} continues Mainland \ili{Norse} and not \ili{Icelandic}. I continue to follow \citet[Ch.1, p.127]{EmondsFaarlund2014}, who exclude any particularities of \ili{Icelandic}, because all known or plausible \ili{Scandinavian} immigration into the \isi{Danelaw} was directly from Denmark and Norway, and also because \ili{Icelandic} centrally differs from both \ili{Mainland Scandinavian} and \textsc{me} in maintaining productive proto-Germanic \isi{morphological} case.\footnote{A reviewer objects that because the latter’s texts are older, arguments about \textsc{me} must be primarily based on \ili{Icelandic}. This logic comes down to a version of \textit{post hoc ergo propter hoc}: ``if the texts of X are older than the texts of a related language Y, the a third related language Z must descend from X not Y.'' To see the fallacy, take X to be the oldest \ili{Italian} texts, Y to be medieval \ili{French}, and Z to be any dialects of Quebec \ili{French}.} 

In the new perspective that \textsc{me} is basically the written down form of Anglicized \ili{Norse}, there remains no reason why the earliest \textsc{me} texts must follow the last \textsc{oe} texts. Indeed, a British Library webpage concludes that the first text in \textsc{me}, a translation of a \ili{Latin} homily, dates from c. 1150.\footnote{\url{http://www.bl.uk/learning/timeline/item126539.html}} This dating is suspiciously late (possibly to reconcile its language with the notion that it must post-date \textsc{oe} texts), since Ralph d’Escures’s original must have pre-dated his debilitating stroke in 1119 (d. 1122).\footnote{\url{https://en.wikipedia.org/wiki/Ralph_d'Escures}} Though traditional histories of English don’t acknowledge any dating overlap, works arguably in \textsc{oe}, e.g. the poem \textit{The Owl and the Nightingale,} were written at least until close to 1200.\footnote{This poem is said to be in the \textsc{me} ``southern dialect'', again to preserve the idea that \textsc{oe} ``changed'' diachronically into \textsc{me} in a short period around 1150. Historians of English allude to a long hiatus in written English to allow for this ``development'', but the hiatus was no more than 50 years; the two different written languages may have even co-existed for a short time.} 

When Anglicized \ili{Norse} began to be written extensively around 1200 (e.g. the text \textit{Ormulum}), it was considered to be a version of English, what is now called the East Midlands dialect of \textsc{me}. Uncontroversially this dialect is the forerunner of Modern English, which therefore descends from \ili{Norse}, not from \textsc{oe}. The latter became the Southern and Western dialects of \textsc{me}, which eventually ceased being written or spoken. 

  This paper proposes to strengthen the hypothesis \REF{ex:emonds:1} by arguing on theoretical grounds that \textit{several \textsc{ng} syntactic properties of \textsc{me} could not have entered \textsc{me} by borrowing}. They must have resulted from internal developments in Anglicized \ili{Norse}. That is, the paper puts forward a claim that limits the \textit{type} of morpho-syntax that can be \isi{borrowed} via language contact. It is of course not disputed that changes are sometimes simply internal to a language (e.g. in Modern English, the sharply differing syntax of modals and lexical verbs, the development of the progressive). My proposal here is that several Early \textsc{me} syntactic properties must have developed this way. If these properties are moreover typical of \textsc{ng}, it must be that Early \textsc{me} is also.\footnote{Rephrasing, if Early \textsc{me} simply continues \textsc{wg} \textsc{oe}, and yet displays the \textsc{ng} properties discussed in \sectref{sec:emonds:4}, \textsc{me} would have had to acquire them by \isi{borrowing}. I will argue that certain such borrowings are impossible (contrary to the literature of language contact, which essentially holds that under contact ``anything can happen''.)}

  Before continuing, I should acknowledge an extensive attempted refutation of hypothesis \REF{ex:emonds:1}, the review of \citet{Bech2015}. About a quarter of it consists of a section on method,\footnote{That section takes issue with the decision of \citet{EmondsFaarlund2014} to argue on the basis of syntax and morpho-syntax, leaving aside studies of \textsc{dna}, whether the bilingualism of the time was social or individual, sound change, etc.} and the rest proposes different interpretations of mostly well-known patterns. Space limitation obviously precludes an evaluation of their 20 page work in this study, which focuses on presenting another type of argument for \REF{ex:emonds:1}. I can indicate, however, that one syntactic argument which they stress (their Sect. 3.3.1) concerns their claim that \textsc{me} and \ili{Scandinavian} ``verb-second'' systems are different, and that the former continues at least one salient \textsc{oe} pattern, allowing scene{}-setting \textsc{pp}s in pre-subject position. \citet{Emonds2016} argues that a better account of this \textsc{me} construction is available in terms of \isi{Universal Grammar}, and is unrelated to a language-particular continuation of \textsc{oe}.

\section{Borrowable syntax: The lexical entries of Borer’s Conjecture}\label{sec:emonds:4}

\citet{EmondsFaarlund2014} discuss over 20 morpho-syntactic properties that indicate that \textsc{me} has \textsc{ng} syntax. Some 15 of these can be formally expressed, without much difficulty, as single entries in its Grammatical \isi{Lexicon}\is{lexicon}. According to Borer’s Conjecture \citep[29]{Borer1984}, now widely adopted in generative studies, such entries are the essence of language-particular grammars.

As a result, it is possible in principle that a rapidly evolving \textsc{oe} could have \isi{borrowed} (or lost) such properties/entries through contact with \ili{Scandinavian} speakers. Nonetheless, as Emonds and Faarlund argue, the sheer number of these entries and the short interval in which they were \isi{borrowed} or disappeared (leaving aside speculations about a distinct ``spoken \textsc{oe}''), constitutes a strong argument in favour of the Anglicized \ili{Norse} Hypothesis \REF{ex:emonds:1}.\footnote{The traditional view (\textsc{oe} → \textsc{me}) must locate this avalanche of changes and several others inside a single century.} Here is a list of the changes that can be associated with individual grammatical morphemes: The last two (\ref{ex:emonds:2}p--q) have been brought to my attention after publication of \citet{EmondsFaarlund2014}.

\ea \label{ex:emonds:2}%bkm:Ref442612570
     

      \textbf{Single entries in the \textsc{me} Grammatical Lexicon} (attributable to \ili{Norse})\footnote{These constructions are all discussed separately in \citet[Chs 3-6]{EmondsFaarlund2014}. I don’t formalize these entries here because specific notations might be controversial and/or difficult to grasp at a glance. It is unfortunate that three decades of lip service to Borer’s Conjecture have produced so few actual proposals for formalizing these entries, now the sine qua non for truly generative grammars. Cf \citet{Emonds2000} for extensive arguments favouring syntax-based formalized lexical entries.} 

    \z

\begin{enumerate}[label={(\alph*)}]
\item As in \textsc{ng}, certain modals start to express the future \isi{tense} in \textsc{me}. In contrast, \textsc{oe} uses present \isi{tense} and Modern \textsc{wg} uses non-\is{Modal}modal (agreeing\is{agreement}) auxiliary\is{Auxiliary} verbs. 
\item The \textsc{me} infinitival \textit{to} is a free morpheme like Old \ili{Norse} \textit{at}; both can be split from V. \textsc{wg}  uses only bound prefixes (\ili{Dutch} \textit{te,} \ili{German} \textit{zu}), and this includes \textsc{oe} \textit{to} (Susan Pintzuk, pers. comm.). 
\item No \isi{passive}/past \isi{participle} prefix is the general rule in Old \ili{Norse} and \textsc{me}. But in \textsc{oe} and \ili{German} the prefix \textit{ge-/y-} is frequent and sometimes obligatory. (\textit{Ge-} was also lost in some \textsc{wg} dialects bordering on \textsc{ng} areas, but these were not sources of \textsc{me}.)
\item The \textsc{ng} languages including \textsc{me} have a perfect \isi{infinitive} \textit{to have V+en}. \textsc{oe} does not \citep[336--337]{Fischer1992}.\footnote{A reviewer observes that the modern ``\ili{West Germanic} languages \ili{Dutch} and \ili{German} also have a perfect \isi{infinitive}.'' But even so, no one suggests that these languages influenced the change from \textsc{oe} to \textsc{me}, so this observation is beside the point. The only issue here is then whether the \textsc{oe} lack of a perfect \isi{infinitive} was typical of \textsc{wg} or was somewhat special.}
\item Like \textsc{ng}, \textsc{me} expresses sentence \isi{negation} with free morphemes that are initial in \textsc{vp} (\ili{Norse} \textit{ikke,} \textsc{me} \textit{naht}). \textsc{oe} uses a pre-verbal bound morpheme \textit{ne-}.\footnote{Early \textsc{me} puts together the \textsc{oe} prefix \textit{ne-} and the post-verbal free morpheme \textit{noht} (i.e. ‘double \isi{negation}’)\textit{,} but eventually drops \textit{ne-}, and so syntactically adopts the \textsc{ng} pattern. Three reviewers bring up ``Jespersen’s cycle'', a descriptive name for the preceding process, combined with the possibility that free \isi{negation} morphemes can also become bound (Modern English \textit{not} → \textit{n’t}). As always, the issue is, can we explain such replacements? \citet[Sect. 7.2.6]{EmondsFaarlund2014} argue that the free morpheme in \textsc{me} is simply a part of Anglicized \ili{Norse} replacing West Saxon. One reviewer, after commenting that this ``similarity of \textsc{me} and \textsc{ng} is indeed striking,'' cites \citet{Breitbarth2009} to the effect that \isi{negation} developed similarly in \ili{Dutch}, \ili{German} and Frisian. If so, the {question} is again, which of \ili{Norse} and these \textsc{wg} languages were in contact with Early \textsc{me}, so that one of them could have either influenced \textsc{oe} (the traditional view) or replaced it (the view of this essay). Under either view, the historical facts point unambiguously to \ili{Norse}.}
\item \begin{stylerootsmmxivArticleText}
As in Old \ili{Norse}, \textsc{me} \textit{that} appears in complex subordinators like \textit{now that},\textit{ if that},\textit{ before that}, \textit{in that, etc.} while \textsc{oe} and \textsc{wg} typically don’t use general subordinators (\textit{þe, dass}) in this way (\citealt[143--144]{EmondsFaarlund2014}; \citealt[ 295]{Fischer1992}). 
\end{stylerootsmmxivArticleText}\item The \textsc{oe} “correlative adverbs” \textit{swa…swa, tha…tha,} etc. are unknown in Mainland Scandinavia and disappear in \textsc{me}.
\item Early \textsc{me} loses \textsc{oe} \isi{relative} pronouns that display case or \isi{gender} (\textit{se ƿe};  \citealt{Mitchell1992}). As in Old \ili{Scandinavian}, Early \textsc{me} relativizers are invariant. 
\item \textsc{me}, like \textsc{ng}, grades long adjectives analytically (\textit{more, most)}. \textsc{oe} does not.
\item As in \ili{Mainland Scandinavian}, the \textsc{me} subjunctive is no longer used to mark indirect speech, as it could in \textsc{oe} \citep[314]{Fischer1992}. 
\item Subject pronouns could be pro-clitics on second position Vs in \textsc{oe}, but not in \textsc{ng} languages. \textsc{me} came to resemble \textsc{ng} in this regard (\citealt{KrochEtAl2000}).
\item The copula selects \isi{infinitive} complements in both \textsc{me} and \textsc{ng}, but not in \textsc{oe} \citep[336--337]{Fischer1992}. 
\item The derived nominal suffix in \textsc{oe} is \textit{–ung} (like \ili{German})\textit{.} \textsc{ng} also allows \textit{–ing,} and in \textsc{me} the latter form is the only possibility.
\item The general patronymic suffix in \textsc{ng} for new families is \textit{–son;} it replaces \textsc{oe}\textit{ –ing} throughout England in c. 1200 \citep{Strang1970}.
\item \textsc{oe} genitive case appears on the head \textit{and} pre-modifiers in a \isi{possessive} phrase. But in both \textsc{me} and \textsc{ng} a single enclitic \textit{–s} follows a \isi{possessive} phrase, whether its head is final or not.
\item In \textsc{wg} (\textsc{oe} and \ili{German}), some intransitive verbs, with meanings “of movement and change of state” form the perfect with \textit{be} \citep{Denison1993}\textit{.} But in \textit{the earliest \ili{Norse} texts,} the \textit{perfect auxiliary} was uniformly \textit{hafa} ‘have’ (J. T. Faarlund, pers. comm.) Similarly in England, the change to uniform use of \textit{have} (\textsc{me}) rather than \textit{be} (\textsc{oe}) with motion verbs begins just after the Conquest \citep[350--355]{Denison1993}. 
\item \ili{Mainland Scandinavian} and Modern English (bur not \ili{Icelandic}) share the possibility of a null Complementizer\is{complementizer} in a range of finite\is{finiteness} clauses \citep{Holmberg2016norse}, It is difficult to search for null items in corpora, so this may be a medieval \textsc{np} property or a more recent shared innovation.
\end{enumerate}

  On point (o) of \REF{ex:emonds:2}, a reviewer asks if it isn’t ``very unlikely that the version of \ili{Norse} that was spoke in the British Isles only marked genitive by adding \textit{–s} at the end of the entire \textsc{np}''? As there are no texts prior to \textsc{me}, one can only note that this device, unknown in \textsc{oe} and \textsc{wg} generally, is restricted to and yet general in \ili{Mainland Scandinavian} \textit{throughout the \textsc{me} period}.  \citet{ChalupováInPreparation} give numerous examples, including these from \citet{Kroch2000}.


\ea \label{ex:emonds:3} 
\ili{Middle English}

\gll ani ancre Iesu cristes spuse\\ 
   \textsc{q}~ hermit Jesus Criste-s spouse \\\jambox{(\citetitle{Ancrene}, II.98.1173)   (1215--1222)}

\gll  sein gregories wordes\\
 Saint George-s word.\textsc{pl} \\\jambox{(\citetitle{Ancrene}, II.61.632)   (1215--1222)}

\gll te holy gostes helpe\\
  the Holy Ghost-s help \\\jambox{(\citetitle{Ayenbite}, 98.1923)   (1250--1350)}

\gll þe dome of godes spelle \\
  the judgment of god-s story \\ \jambox{(\citetitle{Ayenbite}, 11.125)  (1250--1350)}
  \z


  These authors further point out what most corpus dating obscures: mature authors are presumably using grammar acquired as children, often easily thirty years earlier than the date of a text. In particular, they report the last two examples as penned by a 70-year old born between 1180 and 1280. Hence these examples of phrasal \textit{{}-s} no doubt reflect the spoken language of 1185--1285. 

The {question} that emerges from such data is: where else could this highly unusual type of case-marking (a phrasal suffix in an otherwise analytic head-initial language) have come from, if not from a \ili{Norse} continuously spoken in England throughout this period? We expect odd borrowings only into contracting or dying languages (the reviewer exemplifies with Prince Edward Island \ili{French}), but \textsc{me} was not dying out.

For purposes of discussion I grant that, at least singly, the \ili{Norse} features of Early \textsc{me} in \REF{ex:emonds:2} could have been \isi{borrowed} through contact. Individually they all conform to Borer’s Conjecture, and plausibly, changes in particular grammars involve \isi{borrowing} or deleting single entries in Grammatical Lexicons.

Thus, looking through the list \REF{ex:emonds:2} one by one, \textit{no single one of these properties is in itself implausible} as “contact \isi{borrowing} in syntax” from \ili{Norse} into an evolving \textsc{oe}.\footnote{But taken together, as noted above, it is completely implausible that the long list in \REF{ex:emonds:2} could be borrowings effected within a century. And sociologically, why would monolingual Anglo-Saxon speakers \isi{borrow} so copiously from the supposedly dying language of their former adversaries?}  A rather transparent interpretation of Borer’s Conjecture is thus that changes in particular grammars are simply changes in the lexical entries of individual functional category morphemes, and as such, they can be \isi{borrowed} through contact.\footnote{One reviewer provides a sequence of alternative scenarios for many of the points in \REF{ex:emonds:2}. There is no unified pattern in these disparate and sometimes complex suggestions; it is simply a list of separate diachronic events which must be postulated to counter the unified explanation of \REF{ex:emonds:1}, namely that in the syntactic development of early \textsc{me}, nothing happened. In a few instances, the reviewer supports suggestions with evidence. Thus, there are \textsc{me} remnants \textit{y-} of the \textsc{wg} participial prefix \textit{ge-.} I suggest this –\textit{y} was due to bilingual Saxons, and predictably disappeared after a few generations. In other instances, data seems misinterpreted. E.g. \textsc{me} and \ili{Norse} both uniformly use \textit{have} in the perfect \isi{tense}, unlike the \textsc{wg} and \ili{Romance} languages cited by the reviewer, where \textit{have} and\textit{ be} alternate.}


\section{Language-particular architecture: Syntax which cannot be borrowed}\label{sec:emonds:5}

In addition to the constructions in \REF{ex:emonds:2},  \citet{EmondsFaarlund2014} discuss six \textsc{ng} constructions in \textsc{me} that \textit{cannot be expressed as lexical entries for single morphemes}. That is, these constructions are generalizations that lexical entries may reflect, but the entries themselves do not suffice for expressing them in single statements. Because of their more general nature, I call them \textsc{architectural} rather than lexical properties. The first four in the list \REF{ex:emonds:4} are well attested in earliest Mainland \textsc{ng} as well as \textsc{me}; the last two are easily found only in the modern period. Discussion and references for each property are given in the sections indicated below from \citet{EmondsFaarlund2014}.\largerpage[2]

These generalizations that describe these language-particular configurations cannot be adequately expressed formally by single lexical entries, e.g. \isi{P-stranding} is not a property that different Ps accidentally have in common. To maintain an adequate model, either Borer’s Conjecture or the notion of lexical entry will have to be modified in some way. However this is to be accomplished, the content in \REF{ex:emonds:5} below of “lexical specifications of only single functional category items" should remain unchanged.\largerpage[2]

\ea \label{ex:emonds:4}%bkm:Ref442694699
     

      \textbf{North Germanic architectural properties of \textsc{me},} not part of \textsc{oe}:


\begin{enumerate}[label={(\alph*)}]
\item \textit{Head-initial word order within \textsc{vp}s} is unmarked, in both main and dependent clauses \citep[3.1]{EmondsFaarlund2014}.\footnote{A reviewer feels that \ili{Norse} could have changed English word order by contact. Keeping in mind that \ili{Norse} was dying out under this traditional scenario, this is as likely as \ili{French} shifting to \textsc{vso} order in the face of Breton dying out. The reviewer also claims that the \textsc{ie} shift away from verb-final orders was ``arrested'' in Indo-Aryan by contact with \ili{Turkic} and Dravidian. But arrested change is no change and requires no convoluted contact explanations.  This reviewer also repeats a widespread assumption that ``Southern \ili{Semitic} languages have shifted to head-final orders … due to contact with Nilotic'' [sic; presumably Cushitic, JE]. For a carefully argued alternative to this scenario in terms of diglossia, see \citet{Ouhalla2015}.} 
\item A system of \textit{post-verbal \isi{directional} and aspectual free morpheme particles}, contrasted with \textsc{wg} systems of pre-verbal separable prefixes (3.2). This change may be facilitated by (\ref{ex:emonds:4}a), but the two properties are definitely not the same \citep{Emonds2016}.
\item \textit{Preposition stranding}, at first in \isi{relative} clauses and eventually even in sluicing constructions (\textit{who with, what for,} etc. \citealt[3.7--3.8]{EmondsFaarlund2014}.)
\item \textit{Subject raising,} both into subject and object position after epistemic verbs. These are absent in both \textsc{oe} and \textsc{wg} generally (\citealt[221]{Denison1993}; \citealt[82]{Hawkins1986}; \citealt[3.3--3.4]{EmondsFaarlund2014}).
\item Freely formed \textit{parasitic gaps;} these appear freely only in \textsc{ng}, and are restricted or absent in \textsc{wg} languages (for \ili{German}, see \citealt{Kathol2001}; for \ili{Dutch}, see \citealt{Bennis1985}; \citealt[6.4]{EmondsFaarlund2014} ).\footnote{As a reviewer notes, these sources indicate that the basic \textsc{ov} character of \ili{West Germanic} seems to preclude many parasitic gaps that exist in English. A more complete future understanding of them may lead to deriving property (\ref{ex:emonds:4}e) from (\ref{ex:emonds:4}a). This does not affect this study’s conclusion; it simply would mean that of the six architectural properties listed here, only five are independent.} 
\item \textit{Tag questions} based on syntactic copies of the Subject and Tense\is{tense} in \textsc{ng} but not in \textsc{wg} \citep[6.5]{EmondsFaarlund2014}.
\end{enumerate}
\z

These constructions all seem to be language-particular properties of \textsc{ng} languages. For example, there is no widely accepted evidence that any of them are Indo-European. This suggests that all must have developed \textit{internally} in \textsc{ng} languages during the Germanic {phase} of their history.  There is no evidence outside the issue at hand (the relation of \textsc{oe} to \textsc{me}) that any of the six constructions in \REF{ex:emonds:4} have ever been \isi{borrowed} by contact \textit{either into or from} neighbouring \ili{West Germanic}, Celtic or \ili{Slavic} languages.

There are 72 ways one of these six properties could have been \isi{borrowed} from one of these four language families into another.\footnote{Given four families A, B, C, D, possible borrowings for each property are A\textrightarrow B, B\textrightarrow A, A\textrightarrow C, C\textrightarrow A, A\textrightarrow D, D\textrightarrow A, B\textrightarrow C, C\textrightarrow B, B\textrightarrow D, D\textrightarrow B, C\textrightarrow D, D\textrightarrow C, 12 total. There are six properties, so the total conceivable borrowings are 72 in all. Even if the source family for these properties is taken as certain, there are still 18 possible but unattested borrowings.} While I cannot categorically state that none have ever occurred, the number of such borrowings is minuscule compared to the implication of traditional histories of English. These contend that contact with dying \ili{Norse} (or pure accident) caused Middle and Modern English to acquire all six \textsc{ng} architectural properties, four in the space of at most 200 years.\footnote{There is one problematic instance that might require some revision in \REF{ex:emonds:5}. Current research may point to a relation between Celtic and \textsc{ng} languages in the syntax of tag questions. When the import of this research becomes clearer, we can re-assess its relation to \REF{ex:emonds:5}.}

Given these considerations, I wish to strengthen the hypothesis that \textsc{me} is Anglicized \ili{Norse} \REF{ex:emonds:1}, by proposing that \textit{the \textsc{ng} syntactic properties in} \REF{ex:emonds:4}, or at least most of them, could not in principle have been \isi{borrowed} into late \textsc{oe} from \textsc{ng}. They must have resulted from internal developments in \ili{Norse}, most of which we know predated or were simultaneous with \ili{Scandinavian} settlement in England. That is, I suggest a restrictive and historically justified hypothesis that limits what aspects of morpho-syntax can be \isi{borrowed} via language contact. Under this hypothesis, the \textsc{ng} syntax of \textsc{me} cannot in principle be due to “\textsc{oe} + contact with \ili{Norse} speakers”.

To formulate this hypothesis, the morpho-syntactic properties that distinguish \textsc{me} from \textsc{oe} (and group it with \textsc{ng}) have been divided into two types. As explained above, I coin the contrasting labels “lexical” and “architectural” for language-particular properties, according to which the second group has properties that cannot be reduced in a trivial way to the first, i.e. to the format required by Borer’s Conjecture.

\ea \label{ex:emonds:5}%bkm:Ref442699320
 \textbf{Restricted \isi{Borrowing} Hypothesis.} 
 
 

Under language contact, a living language L\textsubscript{1} can \isi{borrow} from L\textsubscript{2} lexical specifications of \textit{only single functional category items}. 
 \z
 
That is, lexical properties can be (sparingly) \isi{borrowed} under contact, but architectural properties cannot be. 

It follows that properties (\ref{ex:emonds:4}a--f) could not have entered \textsc{me} even in a rapidly evolving \textsc{oe} (a fortiori, essentially simultaneously) through language contact of \textsc{oe} with \ili{Scandinavian}. The \textsc{me} properties (\ref{ex:emonds:4}a--f) testify rather to \textit{an unchanging \textsc{ng} character of \textsc{me}}. That is, changes in patterns that we here call the syntactic “architecture” of a particular language can only arise through internal developments, not through simple contact of adjacent languages. 

I note in conclusion how strongly this view contrasts with the traditional claim that \textsc{oe} → \textsc{me}, and hence that all the properties in \REF{ex:emonds:4} developed in a very short time, at least four of them in the 12\textsuperscript{th} and 13\textsuperscript{th} c., via language contact with a language which had already died out (c. 1150) before the evidence of \isi{borrowing} is attested (after 1170). The traditional view of the genesis of Early \textsc{me}, when one reflects on its actual claims, violates not only the canons of a restrictive diachronic theory, but also those of common sense.\pagebreak
 
\section*{Abbreviations}
\begin{multicols}{2}
\begin{tabbing}
DNA \= Derived Nominal Agreement\kill
\textsc{an}  \> Anglicized \ili{Norse}\\
\textsc{dna} \> Derived Nominal Agreement\\
\textsc{ie}  \> Indo-European\\
\textsc{me}  \> \ili{Middle English}\\
\textsc{ng}  \> North Germanic\\
\textsc{np}  \> Noun Phrase\\
\textsc{oe}  \> Old English\\
\textsc{ov}  \> Object-Verb\\
\textsc{p}   \> Preposition\\
\textsc{pp}  \> Prepositional Phrase\\
\textsc{v}   \> Verb\\
\textsc{vp}  \> Verb Phrase\\
\textsc{vso} \> Verb-Subject-Object\\
\textsc{wg}  \> \ili{West Germanic}
\end{tabbing}
\end{multicols}


\section*{Acknowledgements}
I have enjoyed and greatly profited from many linguistic conversations over three de-\linebreak cades with Anders Holmberg, in many different venues throughout Europe. I am especially appreciative of his encouragement regarding this work relating Modern English to its syntactic and {morphological} {Scandinavian} roots, and for his suggestions of points to follow up on. 
I thank two reviewers and the editors for helpful suggestions and encouragement.

\printbibliography[heading=subbibliography,notkeyword=this]
\end{document}