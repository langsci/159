\documentclass[output=paper]{LSP/langsci}
\author{Mais Sulaiman\affiliation{Newcastle University}
}
\title{Verb second not verb second in Syrian Arabic}
\abstract{This paper discusses the obligatory verb second order in a non-verb second dialect, Syrian Arabic. It will be argued following \citet{Holmberg2014verbsecond} that the obligatory Wh-V-S order in Syrian Arabic is a consequence of a property on a functional head F in the left periphery similar to that on C in V2 languages. This head has another property that allows movement of one and only one constituent past its specifier. In cases where more than one XP precedes the verb, the first XP is externally merged. Unlike C, it does not have to attract a constituent to its Spec, so declarative clauses may have VS(O) order.}

\ChapterDOI{10.5281/zenodo.1117720}
\maketitle

\begin{document}


\section{Verb second and residual verb second}


In \isi{verb second} (V2\is{verb second}) languages, the finite\is{finiteness} verb must obligatorily be the second constituent in main clauses or in all finite\is{finiteness} clauses  \citep{denBesten1977,Rizzi1990speculations,Holmberg2014verbsecond}. Some languages manifest V2\is{verb second} in specific constructions, as is the case with wh-questions in languages such as English and \ili{Italian} illustrated in \REF{ex:sulaiman:1} and \REF{ex:sulaiman:2}. A subject cannot intervene between the wh-element and the auxiliary\is{Auxiliary} in main questions. These languages are classified as residual \isi{verb second}, a residue of a \isi{verb second} system.


\ea%1
\label{ex:sulaiman:1}
English \citep[63]{Rizzi1996}

\ea[]{ What has Mary said?}

\ex[*]{What    Mary    has    said?}
\z
\z

\ea%2
\label{ex:sulaiman:2}
\ili{Italian}

\ea[]{
\gll {Che cosa}    ha   detto    Maria?\\
 what                              has      said            Maria\\
\glt ‘What has Mary said?’}



\ex[*]{
\gll {Che    cosa}    ha    Maria detto?\\
what has  Maria    said\\}
\z
\z


Similarly, some languages derive \isi{verb second} order; however, they are not real V2\is{verb second} languages. This is the case in Standard \ili{Arabic}:


\ea%3
\label{ex:sulaiman:3}
Standard \ili{Arabic}  (\citealt[13]{Fassi1993})
\ea \label{ex:sulaiman:3a}
\gll Kataba                                                 Zayd-un        r-risaalat-a. \\
wrote.\textsc{\oldstylenums{3}sg.m}         Zayd-\textsc{nom}        the-letter-\textsc{acc}\\
\glt ‘Zaid wrote the letter.’

\ex \label{ex:sulaiman:3b}
\gll Zaydun    kataba r-risaalat-a.\\
Zayd-\textsc{nom}       wrote.\textsc{\oldstylenums{3}sg.m}              the-letter-\textsc{acc}\\
\glt ‘Zayd wrote the letter.’
\z
\z


Sentences like \REF{ex:sulaiman:3b} manifest a V2-like effect; however, Standard \ili{Arabic} is different from V2\is{verb second} languages in that it allows a V-initial order as the unmarked order see \citet[27ff]{Fassi1993}.

In this paper, I argue that Syrian \ili{Arabic} is not a residual V2\is{verb second} language, yet the subject cannot intervene between the wh-phrase and the verb in wh-questions. This order can be accounted for if we assume that there is a specific feature on a lower functional head in the left periphery that is in common with V2\is{verb second} languages.




\section{Syrian Arabic: A non-residual V2-language}


Syrian \ili{Arabic} employs the VSO order, as is the case in Standard \ili{Arabic}. V2\is{verb second} orders are forced in specific constructions, as in SVO declarative sentences and wh-questions. The Wh-V-S order is obligatory with most questions introduced by argumental wh-phrases.

\ea%4
\label{ex:sulaiman:4}
Syrian \ili{Arabic} \citep[32]{Sulaiman2016}

\ea[]{
\gll shw         ħaka                                               basem?              \\
 what said.\textsc{\oldstylenums{3}sg.m}     Basem           \\
\glt ‘What did Bassel say?’}


 \ex[*]{
 \gll shw  basem ħaka?\\
what       Basem said.\textsc{\oldstylenums{3}sg.m}\\}
\z
\z

\ea%5
\label{ex:sulaiman:5}
\ea[]{
\gll   miin        shaf                                                   Iyad?\\
who              saw.\textsc{\oldstylenums{3}sg.m}     Iyad\\
\glt ‘Who did Iyad see?’}


\ex[*]{
\gll miin Iyad shaf?\\
 who            Iyad      saw.\textsc{\oldstylenums{3}sg.m}\\}
 \z
 \z

However, the Wh-V-S can be optional with some questions introduced with certain adjuncts like \textit{lesh} ‘why’. Compare \REF{ex:sulaiman:6a} and \REF{ex:sulaiman:6b}:



\ea%6
\label{ex:sulaiman:6}
\ea \label{ex:sulaiman:6a}
\gll lesh     mary     tddayʔ-et?              \\
why Mary upsetted-\textsc{\oldstylenums{3}sg.f}          \\
\glt  ‘What did upset Mary?’

\ex \label{ex:sulaiman:6b}
\gll lesh        tddayʔ-et                                         mary?                            \\
 why      upsetted-\textsc{\oldstylenums{3}sg.f}    Mary                 \\
\glt ‘What did upset Mary?’
\z
\z

It is also possible to have a topic phrase preceding the wh-phrase in questions, as in \REF{ex:sulaiman:7}:


\ea%7
\label{ex:sulaiman:7}
\ea
\gll bassel        šw                      ħaka?\\
 Bassel      what      said\\
\glt ‘What did Bassel say?’

\ex
\gll mama      lesh        ʕam                tʕayeT?\\
 mom              why        \textsc{prog}     shouting\\
\glt  ‘Why is mom shouting?’
\z
\z

An adverbial phrase can intervene between the wh-phrase and the verb see \REF{ex:sulaiman:8}.



\ea%8
\label{ex:sulaiman:8}
\ea
\gll min          hallaʔ            ija?\\
 who        now                 come\\
\glt ‘Who has just arrived?’

\ex
\gll shw              issa      ʕam                  t-ʕml-i?\\
 what        still    \textsc{prog}        \textsc{pres.f}-doing-\textsc{\oldstylenums{2}sg.f}\\
\glt  ‘What are you still doing?’
\z
\z

In contrast, movement of either an auxiliary\is{Auxiliary} or the support \textit{do} to C is obligatory in English, leaving the adverb behind, as in (\ref{ex:sulaiman:9}a, b).


\ea%9
\label{ex:sulaiman:9}
\ea Who would you never offend with your actions?
\ex Which language does Pepita still study in her free time?
\z
\z

From what has been discussed, it can be concluded that Syrian \ili{Arabic} is not a V2\is{verb second} or a residual V2\is{verb second} language, yet it manifests a V2\is{verb second} order in specific constructions derived by V movement\is{verb movement} to a lower functional head in the left periphery (see \citealt{Benmamoun2000,AounEtAl2010,Sulaiman2016} for an overview). The obligatory restriction on the verb appearing in a second position in wh-questions can be explained following \citegen{Holmberg2014verbsecond} account of V2\is{verb second} languages.




\section{Movement condition}


\citet{Holmberg2014verbsecond} argues that V2\is{verb second} languages are characterised by two properties: There is a functional head in the left periphery, C1, which (a) attracts the finite\is{finiteness} verb, and (b) has an \isi{EPP} feature that requires movement of a constituent to the Spec of C1. C1 has a third property as well: it prevents movement of any other constituent across it, apart from the one attracted by its \isi{EPP} feature. The rationale for this property, in \citet{Holmberg2014verbsecond}, who follows \citet{Roberts2004}, is the following: the \isi{EPP} feature can attract any constituent (argument or adjunct or wh-phrase, with almost any features). This property blocks movement of any other category to a higher position than Spec of C1. This allows for the possibility, however, that categories are externally merged in the C-domain higher than Spec of C1. The two properties are independent, so in some languages C1 may have property (a) but not property (b), as is the case in certain VSO languages. It is also possible that a language may have a finiteness particle or a null C as C1 with the \isi{EPP} with no \isi{verb movement} to C1.

Following from these assumptions, it can be argued that in Syrian \ili{Arabic}, there is a functional head in the left periphery marked with a feature that is in common with that of C1 in V2\is{verb second} languages. This head allows movement of only one constituent past its specifier, assuming \citegen{Rizzi1997} fine structure of the left periphery. More than one constituent can appear before the verb if one of the constituents is externally merged.\textsc{} Unlike C1, it does not have to attract a constituent to its Spec, so declarative clauses may have the VS(O) order.

This analysis can thus explain sentences like \REF{ex:sulaiman:10}, where more than one XP can appear before the verb.


\ea%10
\label{ex:sulaiman:10}
\ea \label{ex:sulaiman:10a}
\gll basem          šw                      ħaka?\\
 Basem      what      said.\textsc{\oldstylenums{3}sg.m}\\
\glt  ‘What did Bassel say?’

\ex  \label{ex:sulaiman:10b}
\gll Mama      lesh        ʕam                t-ʕayeT?\\
 mom              why        \textsc{prog}     \textsc{pres}-shouting.\textsc{\oldstylenums{3}sg.f}\\
\glt  ‘Why is mom shouting?’
\z
\z

\textit{Basem} in \REF{ex:sulaiman:10a} and \textit{mama} in \REF{ex:sulaiman:10b} can be externally merged in the highest TopP position in the left periphery, only when the wh-phrase raises across C1.\footnote{For discussion on externally merged \textit{aboutness topics} see \citet{Reinhart1981,Lambrecht1994}, and  \citet{FrascarelliHinterhölzl2007}.}

This analysis can also explain sentences like \REF{ex:sulaiman:8}, in which an adverbial phrase intervenes between the verb and the wh-phrase, if we assume that the adverb is externally merged as an adjunct.

A subject can intervene between the wh-phrase and the verb in questions introduced with wh-phrases like \textit{lesh} ‘why’, as illustrated in \REF{ex:sulaiman:11}:


\ea%11
\label{ex:sulaiman:11}
\ea \label{ex:sulaiman:11a}
\gll lesh       mary              tddayʔ-et?\\
why          Mary       upset-3\textsc{sg.f}    \\
\glt ‘What upset Mary?’
\ex
\gll lesh              tddayʔ-et                                      mary?\\
 why         upset-3\textsc{sg.f}      Mary\\
\glt ‘What upset Mary?’
\z
\z

It is well known since \citet{Rizzi1991} that \textit{why} questions are distinctive. Rizzi noted that while other wh-questions require inversion in \ili{Italian}, this is not the case with \textit{perché} ‘why’.

\ea%12
\label{ex:sulaiman:12}

 \ili{Italian} \citep[273]{Rizzi2001}

\ea[]{
\gll Dove  {è andato} Gianni?\\
 where went       Gianni\\
\glt  ‘Where did Gianni go?’}

 \ex[*]{
 \gll Dove Gianni {è andato}?\\
where  Gianni went\\}
\z
\z

\citet{Rizzi1991} proposed that this is because \textit{perché} ‘why’ is base-generated (i.e. externally merged) in the C-domain. \citet{Rizzi2001} suggests that \textit{perché} is externally merged in SpecINT, a position higher than the landing site of other, moved wh-phrases. INT is an \isi{interrogative} head marked with an [uWh] feature. This feature is checked/valued by movement of the wh-phrase to SpecC1, or by an externally merged wh-phrase in SpecINTP like \textit{lesh} ‘why’. This can also be the case for \textit{lesh} ‘why’. If \textit{lesh} ‘why’ is externally merged in the C-domain, the \isi{EPP} feature on C1 can still attract a subject to its Spec, which explains subject intervention between the wh-phrase and the verb in \REF{ex:sulaiman:11a}.

While V2\is{verb second} appears as a restrictive order in Germanic languages, V2\is{verb second} is only triggered in certain constructions like wh-questions in Syrian \ili{Arabic}. V3 orders are possible in wh-questions provided that the first constituent is initially merged in that position as is the case with a topic preceding the wh-phrase, or a base generated wh-adjunct. This can be accounted for following \citegen{Poletto2002} theory that languages can vary with regards to whether left-peripheral functional features are distributed over a hierarchy of distinct heads, each with its own Spec, assuming \citegen{Rizzi2001} hierarchy of [\isi{Force}, \isi{Focus}, \isi{Topic}, and Finiteness\is{finiteness}], and that only one Spec-position higher than the head hosting the finite\is{finiteness} verb can be filled by movement, which can be the case in Syrian \ili{Arabic}, or whether they are encoded in one head, with one Spec-position.


\section{Conclusion}


The fact that the subject cannot intervene between a wh-phrase and the inflected verb in main questions renders Syrian \ili{Arabic} similar to residual V2\is{verb second} languages; however, different facts prove Syrian \ili{Arabic} not to be a V2\is{verb second} or a residual V2\is{verb second} language, yet this restrictive V2\is{verb second} order can best be accounted for following \citegen{Holmberg2014verbsecond} analysis of V2\is{verb second} languages. The assumption that a lower functional head in the left periphery is specified for a feature that attracts a finite\is{finiteness} verb, and an \isi{EPP} feature that can attract a subject or a wh-phrase allowing movement of only one constituent past its specifier can justify this restrictive Wh-V-S order in most questions.


%\section*{Abbreviations}
\section*{Acknowledgements}
This work is dedicated to Anders Holmberg to express my gratitude for his support and feedback.


\printbibliography[heading=subbibliography,notkeyword=this]
\end{document}
