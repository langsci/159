\documentclass[output=paper]{LSP/langsci} 
\author{Luigi Rizzi       	\affiliation{University of Geneva, University of Siena}}

\title{Uniqueness of left peripheral focus, “further explanation”, and Int.} 
% \epigram{Change epigram}
\abstract{In this paper I would like to address the uniqueness of the focus 
    position in the left periphery of the clause and the fixed order of 
    focus with respect to Int, the left-peripheral position hosting 
    interrogative complementizers corresponding to English \emph{if}, Italian \emph{se}, 
    etc.  \citep{Rizzi2001}. Such properties may be amenable to “further 
    explanations” in terms of principles applying at the interfaces of the 
    syntactic component, or in terms of locality principles directly 
    operating on syntactic computations. I will discuss the interplay 
    between these two modes of explanation in the cases at issue, and will 
    briefly explore the consequences of this analysis for ordering 
    constraints involving focus and certain main-clause left peripheral 
    particles such as Sicilian \emph{chi}  \citet{BianchiCruschina2016} and Finnish 
    \emph{-ko} \citet{Holmberg2013,Holmberg2014finnishquestion}.}

\ChapterDOI{10.5281/zenodo.1117718}
\maketitle

\begin{document}
 


  
\section{Introduction}


Cartographic research uncovers properties of functional sequences such as ordering and co-occurrence constraints. The observed generalizations are in need of “further explanations” in terms of more elementary ingredients of linguistic computations, or of principles ruling the interpretive processes at the interfaces with sound and meaning. In this short paper I would like to address certain properties of focus in the left periphery of the clause, such as the uniqueness of the focus position and the fixed order of focus with respect to \isi{Int}, the left-peripheral position hosting \isi{interrogative} complementizers corresponding to English \textit{if}, \ili{Italian} \textit{se}, etc.  \citep{Rizzi2001}. Such properties may be amenable to “further explanations” in terms of principles applying at the interfaces of the syntactic component, or in terms of locality principles directly operating on syntactic computations. I will discuss the interplay between these two modes of explanation in the cases at issue, and will briefly explore the consequences of this analysis for ordering constraints involving focus and certain main-clause left peripheral particles such as \ili{Sicilian} \textit{chi}  \citealt{BianchiCruschina2016} and \ili{Finnish} \textit{–ko} \citep{Holmberg2013,Holmberg2014finnishquestion}.


\section{Uniqueness of focus}


A traditional observation in the study of the left periphery of the clause is that left peripheral focus is typically unique. This was observed, e.g. in \ili{Italian} (\citealt{Rizzi1997}; capitalization is used to express the special prosody of focal constituents, on which see \citealt{Bocci2013}):

\ea%1
    \label{ex:rizzi:1}
\ea  \label{ex:rizzi:1a}
       A MARIA devi dare il tuo libro {\longrule} (, non a Giulia).\\ 
  \glt ‘TO MARIA you should give your book, non to Giulia.’
    \ex  \label{ex:rizzi:1b} IL TUO LIBRO devi dare {\longrule} a Maria (non il disco).
\glt ‘YOUR BOOK you should give to Maria, not the record.’
\ex * A MARIA(,) IL TUO LIBRO   devi dare (non a Giulia, il disco).
\glt  'To Maria   your book   you should give, not to Giulia the record.'
\ex * IL TUO LIBRO(,) A MARIA devi dare (non il disco, a Giulia).
\glt     ‘Your book, to Maria you should give, not the record to Giulia.’
\z
\z

As the negative tag suggests, \REF{ex:rizzi:1a} and \REF{ex:rizzi:1b} are instances of corrective focus. If somebody says \textit{I have to give my book to Giulia}, I can correct him/her by uttering \REF{ex:rizzi:1a}, and similar felicity conditions are observed for \REF{ex:rizzi:1b}; but (\ref{ex:rizzi:1}c--d), correctively focalizing both the direct and indirect object, are ungrammatical, whatever order is selected.   

The uniqueness requirement holds for corrective focus and for “mirative” focus, the other major kind of focus which in \ili{Italian} (and other \ili{Romance} languages) licenses movement to the left periphery:

\ea%2
    \label{ex:rizzi:2}
 
	 Pensa un po’…

\glt      ‘Just think of it…'

\ea   UNA FERRARI vogliono regalare a Mario! Che pazzia!

 \glt ‘A FERRARI they want to give to Mario! That’s crazy!'

\ex  A MARIO vogliono regalare una Ferrari! Che pazzia!

\glt ‘TO MARIO they want to give a Ferrari! That’s crazy!'

\ex * A MARIO(,) UNA FERRARI vogliono regalare! Che pazzia!

 \glt ‘TO MARIO A FERRARI they want to give! That’s crazy!’

\ex * UNA FERRARI(,) A MARIO vogliono regalare! Che pazzia!      

\glt ‘A FERRARI TO MARIO they want to give! That’s crazy!'
\z
\z

As pointed out by \citet{BianchiEtAl2015}, mirative focus differs from corrective focus both in the \isi{pragmatic} conditions for its felicitous use (it expresses surprise for a state of affairs falling outside natural expectations, rather than correcting someone else’s statement) and  intonational contour. But it shares the uniqueness requirement with corrective focus. Both focal constructions were collapsed under the label “contrastive focus” in \citet{Rizzi1997}; new information focus differs from these special focal constructions in that, in \ili{Romance}, it is expressed in a lower position, internal to the IP  \citep{Belletti2004}.

The special left peripheral focus constructions sharply contrast with topic constructions such as \isi{Clitic Left Dislocation}, which is consistent with a proliferation of topics in any order:

\ea%3
    \label{ex:3}
\ea   Il tuo libro, a Maria, glielo devi dare al più presto.

\glt ‘Your book, to Mary, you should give it to her as soon as possible.’

\ex  A Maria, il tuo libro, glielo devi dare al più presto.

\glt ‘To Maria, your book, you should give it to her as soon as possible.’
\z
\z

The uniqueness of left peripheral focus is not a quirk of \ili{Italian}/\ili{Romance}. A similar constraint has been observed in Finno-Ugric \citep{Puskás2000}, West-African \citep{Hager2014}, and \ili{Jamaican} Creole \citep{Durrlemann2008}. A principled explanation thus seems to be in order.

\section{“Further explanation” of cartographic properties}%3.

Cartographic research has uncovered a variety of properties of functional elements occurring in specific zones of the clause, such as ordering and co-occurrence constraints. Some such properties are variable across languages, whereas other properties look cross-linguistically stable. These results raise the question of why one finds certain stable patterns rather than other imaginable alternative arrangements.  As pointed out in  \citet{CinqueRizzi2010} it is unlikely that such patterns may be absolute syntactic primitives, to be stated in UG as unrelated to other requirements or constraints: why should natural language syntax have evolved to express such complex and apparently unmotivated primitives? It is more plausible that functional hierarchies and their properties (to the extent to which they are invariant) may be rooted elsewhere, hence be amenable to “further explanations”.  There are two major imaginable kinds of such explanations (here I follow the discussion in \citet{Rizzi2013}: they may invoke

\begin{itemize}
\item Interface principles: certain observed properties of the hierarchies may be derived from principles ruling the interfaces between syntax and the interpretive systems of sound and meaning; 
\item Formal syntactic principles: “further explanations” may also invoke formal principles internal to the syntactic box, for instance principles of locality (\citealt{Abels2012}; \citealt{Haegeman2012b}) or labeling \citep{Rizzi2016}. 
\end{itemize}


For the case at issue, \citet{Rizzi1997} proposed an explanation of the uniqueness of left-peripheral focus by invoking the routines that interpret discourse related “criterial” configurations at logical form. Suppose that the interpretive principle triggered by the Foc\is{Focus} head is something like the following, in terms of a traditional terminology of, e.g., \citet{Jackendoff1972}:

\ea%4
    \label{ex:rizzi:4} \is{Focus}\is{presupposition}
\gll {[}    {}     ]   Foc     [  {}  ]   {}\\
               ~ “Focus”  ~  ~      ~ “Presupposition”\\
\z

i.e., a Foc\is{Focus} head goes with the instruction: “interpret my specifier as the focus (corrective or mirative), and my complement as the presupposed part”. 

So, let’s suppose, counterfactually, that the Foc\is{Focus} head could recursively occur in the left periphery, giving rise to a double focus. The derived representation would be something like the following:

\ea%5
    \label{ex:rizzi:5} 
	 *  [[A MARIA]    \textbf{Foc1}    [    [ IL TUO LIBRO ]   \textbf{Foc2}      [   devi dare  \_\_ \_\_ ]  ]  ].

\glt ‘TO MARIA                       YOUR BOOK                        (you) should give.’
\z

Here IL TUO LIBRO should be interpreted as a focus of the relevant kind, being the Spec of Foc\is{Focus} 2; but the whole sequence   [ [ \textit{IL TUO LIBRO} ]   [ \textbf{\textit{Foc2}} [ \textit{devi dare} ] ] ]  is the complement of Foc1, hence, under \REF{ex:rizzi:4}, it  should be interpreted as the presupposed part. So\textit{, il tuo libro} should at the same time be focus and presupposed; but presumably a clash arises between the two notions, hence Foc\is{Focus} \isi{recursion} in the left periphery never gives rise to a properly interpretable structure. The observed syntactic constraint is thus derived from a natural interpretive requirement. 

I phrased the argument in terms of the traditional definition of the relevant articulation as \isi{Focus} – \isi{Presupposition}\is{presupposition} \citep{Jackendoff1972}. If the articulation is more neutrally characterized as \isi{Focus} – Background, as in much recent work, as far as I can tell the idea of the interpretive clash arising in configurations of double left-peripheral focus would remain unchanged: something cannot be simultaneously focus and background.

Now, there is a natural competitor to an interface explanation in this case.  \citet{Abels2012} argues that major ordering properties observed in the left periphery in \citet{Rizzi1997,Rizzi2001} can be derived from the theory of locality and in particular from a featural characterization of \isi{Relativized Minimality} (fRM) (as in \citealt{Starke2001}; \citealt{Rizzi2004}, building on \citealt{Rizzi1990minimality}).

Consider a representation like \REF{ex:rizzi:5}. When two foci are moved to the left periphery, one inevitably crosses over the other. So, in \REF{ex:rizzi:5} the attempt to connect A MARIA to its variable would cross a featurally identical focus position filled by IL TUO LIBRO, a configuration which would be ruled out by fRM. So, the theory of locality seems to provide a competing “further explanation” for the uniqueness of left-peripheral focus.

There is, of course, another possibility:  it could be the case that \REF{ex:rizzi:5} is excluded by both orders of considerations. In general, a natural strategy of theory construction is to avoid redundancies, so that, if two paths of explanation compete, it is quite generally productive to try to eliminate the redundancy. Nevertheless, the elimination of redundancies is a useful research strategy, not a dogma: it is imaginable that two optimally stated principles may have areas of overlap, as well as areas of non-overlap in their respective empirical domains. Let us consider the case at issue by looking at other related but distinct effects.
 
\section{Double focus exclusion without intervention}


Consider the case of a complex sentence with a complement of the main verb and a complement of the embedded verb. Either one can be focused and moved to the left\largerpage periphery of the clause immediately containing it, but they cannot be focalized simultaneously:

\ea%6
    \label{ex:rizzi:6}
\ea[]{Devi dire a Gianni che comprerai questo libro.\\
 \glt ‘You should say to Gianni that you will buy this book.’}

 \ex[]{   A GIANNI devi dire che comprerai questo libro (non a Piero).\\
\glt ‘To GIANNI you should say that you will buy this book (not to Piero).'}

 \ex[]{  Devi dire a Gianni che  QUESTO LIBRO comprerai (non quest’altro).\\
 \glt ‘You should say to Gianni that THIS BOOK you will buy (not that one).'}

 \ex[*]{\label{ex:rizzi:6d} A GIANNI devi dire che QUESTO LIBRO comprerai (non a Piero, quest’altro).\\
 \glt ‘TO GIANNI you should say that THIS BOOK you will buy (not to Piero, that one).’}
\z
\z

Sentence \REF{ex:rizzi:6d} is sharply deviant, but in this case no crossing or intervention configuration arises because each focalized element is moved to the periphery of its own clause. So, this example is solely excluded by the interface considerations based on \REF{ex:rizzi:4}: even if the two FocP’s are not adjacent and occur in different clauses, the \isi{presupposition} of the higher focus still contains the lower focus, hence the interpretive clash arises here as well.

Other cases not amenable to locality considerations, but excluded by interface considerations, are those which involve clause initial adjuncts which plausibly are externally merged directly in the left periphery, such as “scene setting” adjuncts  \citep{BenincàPoletto2004}:

\ea%7
    \label{ex:rizzi:7}
\ea \label{ex:rizzi:7a}  Nella foto di Gianni, Piero dà un bacio a Maria.

\glt ‘In Gianni’s picture, Piero gives a kiss to Maria.’

 \ex \label{ex:rizzi:7b}  NELLA FOTO DI GIANNI, Piero dà un bacio a Maria (non in quella di Giuseppe).

 \glt ‘In GIANNI’s picture, Piero gives a kiss to Maria (not in Giuseppe’s picture).'

 \ex  \label{ex:rizzi:7c}  Nella foto di Gianni,  A MARIA Piero dà un bacio (non a Giulia).

 \glt ‘In Gianni’s picture, TO MARIA Piero gives a kiss (not to Giulia).'

 \ex \label{ex:rizzi:7d}  * NELLA FOTO DI GIANNI, A MARIA Piero dà un bacio (non in quella di Giuseppe, a Giulia).

 \glt ‘In GIANNI’s picture, TO MARIA Piero gives kiss (not in Giuseppe’s, to Giulia).’  
\z
\z

In \REF{ex:rizzi:7a} the PP \textit{Nella foto di Gianni} is not a locative fronted from a lower position but an adverbial setting the scene for the following sentence; as such, it is plausibly externally merged directly in the periphery of the sentence \citep{Reinhart1981}. It can be a corrective focus, as in \REF{ex:rizzi:7b}, but it cannot co-occur with another corrective focus, as in \REF{ex:rizzi:7d}. Again, there is no locality effect to be invoked here, whereas the interface explanation based on \REF{ex:rizzi:4} correctly rules out the structure. 

\section{The strict ordering of Focus and Int}%5

There are also cases that are naturally analyzable as involving intervention, whereas they don’t seem to immediately fall under the effect of interface principles. Consider the relative ordering of \isi{Int}, the position hosting yes-no \isi{interrogative} complementizers like \ili{Italian} \textit{se} and English \textit{if} \citep{Rizzi2001} in embedded questions, and \isi{Focus}. The only possible order is \isi{Int} \isi{Focus}:

\ea%8
    \label{ex:rizzi:8}
\ea \label{ex:rizzi:8a}  Mi domando se IL TUO LIBRO abbiano comprato (non il mio).

 \glt ‘I wonder if YOUR BOOK they bought (not mine).'

\ex \label{ex:rizzi:8b}  * Mi domando IL TUO LIBRO se abbiano comprato (non il mio).

\glt ‘I wonder YOUR BOOK if the bought (not mine).'
\z
\z

In this case, focus sharply contrasts with topic, which can both precede and follow \isi{Int}:

\ea%9
    \label{ex:rizzi:9}
\ea  \label{ex:rizzi:9a}  Mi domando se, il tuo libro, lo abbiano comprato.

 \glt ‘I wonder if, you book, they bought it.’

\ex \label{ex:rizzi:9b}  Mi domando, il tuo libro, se lo abbiano comprato.

\glt ‘I wonder, your book, if they bought it.’
\z
\z

The ordering constraint illustrated by \REF{ex:rizzi:8} is not straightforwardly amenable to an interface account of the kind ruling out a double focus, whereas it can be captured by a locality account à la \citet{Abels2012}: \textit{se} presumably hosts a yes-no operator in its Spec (and, in any event, it is marked by a +Q feature, an operator feature). Hence, movement of an element plausibly involving an operator feature, like focus, determines a violation of \isi{Relativized Minimality}, under the approach based on feature classes developed in \citet{Rizzi2004}. No violation is expected in the case of a topic construction like \REF{ex:rizzi:9b}, if such constructions  in \ili{Italian}/\ili{Romance} do not involve any operator feature (\citealt{Cinque1990}, \citealt{Rizzi2004}). Putting together \REF{ex:rizzi:8} and (\ref{ex:rizzi:6}--\ref{ex:rizzi:7}), we thus seem to reach the conclusion that both an interface approach and a locality approach are needed on independent grounds, and the two approaches may overlap in some cases, such as the exclusion of double focus constructions like (\ref{ex:rizzi:1}--\ref{ex:rizzi:2}).

There are additional elements of complexity, though. \citet{Callegari2014} and  \citet{CinqueKrapova2013} point out a problem for a simple locality account in cases analogous to \REF{ex:rizzi:8b}: long distance \isi{extraction} of a focal element from an indirect \isi{question} is more acceptable than local movement across \textit{se}. In my judgement, there is a detectable difference of severity of ill-formedness: long distance focalization from an indirect \isi{question} is only mildly deviant, and more acceptable than local focus movement across \textit{if}, as in \REF{ex:rizzi:8b}:

\ea%10
    \label{ex:rizzi:10}
  ? IL TUO LIBRO mi domando se abbiano comprato, non il mio.

\glt ‘YOUR BOOK I wonder if they bought, not mine.’
\z

The status of \REF{ex:rizzi:10} suggests that movement of focus across \textit{se} does indeed cause a locality violation, but the local movement in cases like \REF{ex:rizzi:8b} must involves a mild locality violation, plus an aggravating factor. What could it be? Let us briefly reconsider the interface idea.  \REF{ex:rizzi:8b} would have, for the relevant part, the following representation:

\ea%11
    \label{ex:rizzi:11}
     … IL TUO LIBRO Foc\is{Focus} [ se\textsubscript{Q} abbiano comprato ]

     \glt ‘...YOUR BOOK            if they bought'
\z

We have seen that, according to \REF{ex:rizzi:4}, the complement of Foc\is{Focus} must be interpreted as the presupposed part. We may speculate that a \isi{question} (a clausal constituent headed and labeled by Q) is not the natural syntactic object expressing a presupposed part, whereas a FinP\is{finiteness} expressing a propositional content would be, as in the ordering in  \REF{ex:rizzi:8a} corresponding to the following configuration:

\ea%12
    \label{ex:rizzi:12}
   ... se\textsubscript{Q}    … IL TUO LIBRO Foc\is{Focus} [\textsubscript{IntP} abbiano comprato ]

    \glt ‘...if             YOUR BOOK                    they bought’
\z

If an interface analysis along these or similar lines can be made precise, it may well be that both factors play a role in the ill-formedness of \REF{ex:rizzi:8b}, with the mild locality violation observed in \REF{ex:rizzi:10} getting “reinforced”, as it were, by an interface factor in \REF{ex:rizzi:8b}. (For a compositional analysis requiring the order Q > Foc\is{Focus} in the case of mirative focus, see \citealt{BianchiEtAl2015}.) 

\section{Some comparative considerations}%6.  

Main yes-no questions with left peripheral focus are also possible (\citealt{Holmberg2013,Holmberg2014finnishquestion} on \ili{Finnish};  \citealt{BianchiCruschina2016} on \ili{Italian} and other \ili{Romance} varieties), e.g., 

\ea%13
    \label{ex:rizzi:13}
  IL TUO LIBRO hanno comprato? (non volevano quello di Gianni?)

\glt ‘YOUR BOOK they bought? (didn’t they want Gianni’s?)’
\z

This case differs from the embedded \isi{question} case in that the \isi{Int} marker is null in main environments in \ili{Italian}. Where does the structural position \isi{Int} appear in main clauses? An extension of what was observed in embedded clauses  suggests that the unpronounced \isi{Int} of main clauses  should be higher than Foc\is{Focus}, much as the pronounced \isi{Int} of embedded clauses: the opposite ordering Foc\is{Focus} > \isi{Int} would be ruled out both by locality and the interface requirement discussed in connection with the embedded environment. 

When a given order is assumed in a certain language on the basis of abstract consideration, it is good practice to check if in other languages the assumed order is visibly manifested. In fact,  \citet{BianchiCruschina2016} show that in \ili{Sicilian}, which has a pronounced form of main \isi{Int} (\textit{chi}), the order is \textit{chi} > Foc\is{Focus} in main questions, as would be expected:

\ea%14
    \label{ex:rizzi:14}
  Chi  a Maria salutasti?

 \glt ‘\textit{Chi}  to  Maria  (you) greeted? = Did you greet Maria?’
\z

\citeauthor{Belletti2004} (p.c.) points out that in the colloquial variety of \ili{Italian} spoken in Rome, main yes-no questions are introduced by the particle \textit{che, }homophonous to the declarative Complementizer\is{complementizer}, which thus appears to be a plausible candidate for the overt lexicalization of \isi{Int} in main clauses. As expected, \textit{che} must precede a left-peripheral corrective focus:

\ea%15
    \label{ex:rizzi:15}
\ea  Che MARIA hai salutato (e non Gianni)?

\glt ‘\textsc{int} MARIA you greeted (and not Gianni)?’ 

   \ex \label{ex:rizzi:15b} *MARIA che hai salutato (e non Gianni)?

 \glt ‘Maria \textsc{int} you greeted (and not Gianni)?’
\z
\z

Belletti also observes that in this variety topics differ from foci in that a \isi{clitic} left-dislocated element can both precede and follow main clause \isi{Int}:

\ea%16
    \label{ex:rizzi:16}
\ea    Maria, che la conosci già? ~~

 \glt ‘Maria, \isi{Int} (you) already know her?’

   \ex  Che Maria, la conosci già?

 \glt ‘\isi{Int} Maria, (you) already know her?'
\z
\z

The pattern is thus identical to the one visible only in embedded clauses is Standard \ili{Italian} (as in \ref{ex:rizzi:8}--\ref{ex:rizzi:9}): the sharply ungrammatical \REF{ex:rizzi:15b} is ruled out both as a locality violation and, plausibly, as a violation of interface requirements, as per our previous discussion; neither kind of violation arises with a topic, as in (\ref{ex:rizzi:16}a--b).

A potential problem arises in \ili{Finnish}. \citet{Holmberg2013,Holmberg2014finnishquestion} shows that in main environments in this language the \isi{question} particle \textit{–ko} occurs after a focalized element:

\ea%17
    \label{ex:rizzi:17}
   Pariisissa\textbf{ko}   Matti  on   käynyt?

 \glt ‘Paris  Matti has  visited?’
\z

This order would be unexpected if \textit{–ko} were to be analyzed as the \ili{Finnish} equivalent of \isi{Int}. But \citeauthor{Holmberg2014finnishquestion} argues in detail that (this instance of) \textit{–ko} must be analyzed as affixed to the focused constituent, and then moved to the left periphery with it, rather than as a particle externally merged in the left peripheral spine (as \isi{Int} is, under our analysis). 

Other considerations support this analysis. For instance, Karoliina Lohiniva (p.c.)  observes that, in her native Oulu dialect, \textit{–ko} can be repeated after each conjunct in case of a coordination of focused DPs:

\ea%18
    \label{ex:rezzi:18}
    \gll Isä-s-kö ja äiti-s-kö se-n auto-n ost-i?\\
         father-poss-\textsc{q} and mother-poss-\textsc{q} that-\textsc{acc} car-\textsc{acc} buy-3\textsc{sg}.past\\
 \glt ‘Was it your father and your mother who bought that car?’
\z

This is expected if \textit{–ko} is affixed to the focal DP(s), whereas it is inconsistent with an analysis of  \textit{–ko} as a lexicalization of \isi{Int}. Not surprisingly, a bona fide lexicalization of \isi{Int} such as Roman \textit{che }cannot be reduplicated in such cases (\citeauthor{Belletti2004}, p.c.):

\ea%19
    \label{ex:19}
 Che MARIA e (*che)  FRANCESCA hai salutato?
\glt       ‘\isi{Int}   MARIA and (*\isi{Int}) FRANCESCA (you)  greeted?’ 
\z

The observed ordering in \ili{Finnish} thus is consistent, under Holmberg's analysis of \textit{–ko}, with the general ordering constraint \isi{Int} > Foc\is{Focus} that we have discussed.

\section{Conclusion}

The necessity of seeking for “further explanations” of cartographic properties leads us to take into consideration the possible explanatory role of interpretive principles applying at the interfaces, and of locality principles applying within the syntactic box. In the attempt to study the interplay between these two orders of considerations, we have been led to look at the case of the exclusion of multiple left peripheral foci in \ili{Romance} and other languages. Here, simple considerations on the functioning of interpretive principles of the scope-discourse articulations involving left peripheral focus may provide a straightforward analysis of the observed restriction: in cases of double left peripheral focus, the lower focalized element would inevitably end up in the presupposed, or background part, of the higher focus, giving rise to an interpretive clash. On the other hand, under minimal assumptions on the functioning of intervention, locality also marks cases of double focus movement as deviant, as one focalized constituent would necessarily be moved across the other. In the attempt to disentangle these two modes of further explanation, we have observed cases in which locality considerations are not (immediately) relevant, e.g. in ill-formed cases in which two focused elements come from different clauses and are moved to the respective immediate left peripheries: in such cases there is no crossing, and still the structures are deviant. So, such cases appear to be “pure” cases of violation of interface constraints. We have also looked at a potential case of “pure” locality violation: the case of the obligatory ordering \isi{Int} > Foc\is{Focus} in \ili{Romance}.  Here a violation of featural \isi{Relativized Minimality} is certainly involved in the ordering Foc\is{Focus} > \isi{Int}, as the focal operator is crossing the Q-marked Complementizer\is{complementizer} particle, also a position marked by an operator feature. But the hypothesis that this is a pure case of locality violation has to face the problem of the different degree of deviance of local and long distance movement of a focal element across \isi{Int} (as in \ref{ex:rizzi:8b}--\ref{ex:rizzi:10}). The more severe violation of local movement suggests that some other consideration in addition to locality is operative here: this has led us to look again at the role of interface principles.  Finally, comparative considerations on overt manifestations of the ordering phenomena in main clauses have lead us to assume, with  \citet{BianchiCruschina2016}, that \ili{Sicilian} \textit{chi} is a genuine \isi{Int} marker, exhibiting the expected order \isi{Int} > Foc\is{Focus} also in main clauses, and to adopt \citeauthor{Holmberg2013}'s (\citeyear{Holmberg2013, Holmberg2014finnishquestion}) analysis of \ili{Finnish} \textit{–ko} as an affix attached to the focused constituent, rather than as a manifestation of the \isi{Int} head in the left peripheral sequence.

\section*{Acknowledgments}

This paper is dedicated to Anders, who showed us the intricacies of the languages of the Northern countries, and their theoretical underpinning.
I would like to thank Adriana Belletti, Valentina Bianchi and Karoliina Lohiniva for very helpful comments. This research was supported in part by the ERC Grant n. 340297 “SynCart”. 


{\sloppy\printbibliography[heading=subbibliography,notkeyword=this]}
\end{document}