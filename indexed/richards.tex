\documentclass[output=paper]{LSP/langsci} 
\author{Norvin Richards\affiliation{MIT}
}
\title{Nuclear stress and the life cycle of operators}
\abstract{I offer a new argument in this paper for the proposal that empty categories left by extraction of DP may be of different kinds, depending on the nature of the extraction (\citealt{Perlmutter1972,Cinque1990,Postal1994,Postal1998,Postal2001,O’Brien2015,Stanton2016}, and much other work). The argument is based on Postal’s (\citeyear{Postal1994,Postal1998,Postal2001}) discussion of the interaction of extraction with antipronominal contexts, together with \citegen{Bresnan1971} observations about the effects of extraction on the position of nuclear stress.}

\ChapterDOI{10.5281/zenodo.1117716}
\maketitle

\begin{document}
 


\section{Introduction}


I will offer a new argument in this paper for the frequently defended proposal that empty categories left by \isi{extraction} of DP may be of different kinds, depending on the nature of the \isi{extraction} (\citealt{Perlmutter1972,Cinque1990,Postal1994,Postal1998, Postal2001,O’Brien2015,Stanton2016}, and much other work).  The argument will lean on two discoveries in previous work.

  \citet{Postal1994,Postal1998,Postal2001} (and following him, \citealt{Stanton2016} and \citealt{O’Brien2015}) draws a distinction between two kinds of A-bar \isi{extraction}.  He notes that some kinds of \isi{extraction} cannot take place out of positions that could not be occupied by an unstressed pronoun (Postal's \textit{antipronominal contexts}), while other kinds of \isi{extraction} can move DPs from such positions:


\ea%1
\label{ex:richards:1}
\ea
  That Porsche cost \$50,000/*it. 
\ex
      *  \$50,000, which that Porsche cost, (...is a lot of money.) 
\ex  What did that Porsche cost?
\z
\z


Postal, Stanton, and O’Brien all claim that extractions which are impossible out of antipronominal contexts leave something like a pronoun in the position of the gap, while other kinds of extractions leave some other kind of empty category behind.  I will adopt and argue for this proposal.

  \citet{Bresnan1971} also draws a distinction between two kinds of A-bar \isi{extraction}.  Her observations are about the interaction of \isi{extraction} with the distribution of nuclear stress.  We will see that Bresnan's distinction is the same as Postal's and Stanton's, and I will argue that it has the same explanation; \isi{extraction} of a DP from a position of nuclear stress has different effects, depending on what kind of DP is left behind in the gap position.  In the end, Bresnan's observation will allow for an explanation of the ban on certain kinds of \isi{extraction} from antipronominal positions, which will be based on the principle that nuclear stress may not be assigned to phonologically null material.  We will see that English uses various means of creating PF representations that avoid stress on phonologically null material, and that when these means fail, the structure is ill-formed.  The account will rely on an approach to syntax and phonology, like the one in \citet{Bresnan1971} and in \citet{Richards2016}, in which the narrow-syntactic computation is responsible for the creation of certain kinds of phonological representations; in particular, syntax will have to determine the position of nuclear stress.

  \citet{Stanton2016} proposes that some kinds of \isi{extraction} involve the creation of an operator out of a pronoun via liberal use of Late \isi{Merge}; such extractions, she claims, therefore leave a null pronoun behind in the \isi{extraction} site, which is why they cannot take place out of antipronominal contexts.  Similarly, \citet{Johnson2012} and \citet{O’Brien2015} argue that movement involves the creation of multidominance structures, which again effectively allow an operator to grow in size as it moves up the tree.  Either of these approaches will be consistent with the facts to be discussed here.  We will see that the proposal offers new support for something like \citegen{Johnson2012} version of \citegen{Fox2003} operation of Trace\is{trace} Conversion; the syntactic differences between the operator and its \isi{trace}, rather than being created by a rule at LF, are present throughout the derivation.

  The next section will describe some facts of English nuclear stress that will be important for the coming arguments.  I will then review the data discussed by \citet{Postal1994,Postal1998,Postal2001,Stanton2016}, and \citet{O’Brien2015}, and show how an account based on conditions on assignment of stress might capture them.  We will then turn to \citegen{Bresnan1971} observations about \isi{extraction} and nuclear stress, and I will show that the categories of movement identified by Bresnan are the same as those identified by Postal, Stanton, and O’Brien.  Finally, the paper will conclude with some observations about consequences of the proposal, and suggestions for possible future work.

  It may also be worth directly pointing out several things I will not attempt to do in this paper.  The account will rest on two ideas from previous work: that certain types of \isi{extraction} leave behind null pronouns in the \isi{extraction} site, and that there are positions in the sentence where a pronoun must, unusually, receive nuclear stress.  Both of these ideas raise a host of interesting questions:  what determines which types of \isi{extraction} leave pronouns behind?  Why must pronouns receive stress in certain positions, though not in others?  I will offer some speculations about the answers to these questions, but will leave serious investigation of them for the future.  For answers to the first {question}, in particular, see \citet{Stanton2016} and \citet{O’Brien2015}.

\section{English nuclear stress}

In English, and possibly universally, nuclear stress in examples like \REF{ex:richards:2} typically appears on the direct object (I will mark nuclear stress throughout with \textsc{small capitals}):


\ea%2
\label{ex:richards:2}
\ea  Bill saw a \textsc{car}.
\ex   Mary bought a \textsc{book}.
\ex  I heard \textsc{music}.
\z
\z


There are various kinds of expressions that typically do not get nuclear stress, even in object position, including pronouns and words like \textit{something}, \textit{someone}, \textit{anything}, \textit{anyone}, which I will refer to here as `simple indefinites'.  When a direct object is a pronoun or a simple indefinite, nuclear stress appears on the verb:


\ea%3
\label{ex:richards:3}
\ea  Bill \textsc{saw} me.
\ex  Mary \textsc{bought} something.
\ex  I don't \textsc{hear} anything.
\z
\z


\citet{Bresnan1971} notes that DPs which are immediately anaphoric to a preceding DP, even if definite, exhibit the same behavior (\citealt[258]{Bresnan1971}):

\ea%4
\label{ex:richards:4}
  	  John knows a woman who excels at karate, and he \textsc{avoids} the woman.
\z

  Pronouns, simple indefinites, and immediately anaphoric DPs, then, all avoid nuclear stress; in all of the examples just discussed, nuclear stress then shifts to the verb.  I will review some other salient cases in the following section.  In the section after that, we will turn to Postal's antipronominal contexts, where we will see the facts change.

\subsection{Nuclear stress in PP}\label{sec:richards:2.1}

Objects of prepositions are a common locus of nuclear stress, and when the object cannot bear nuclear stress, stress is relocated, typically either to the \isi{preposition} or to the verb.  The main  point of this section is simply to demonstrate that for most contexts, pronouns, simple indefinites, and anaphoric DPs are generally treated alike.  

  Many prepositions, apparently including both arguments and adjuncts, behave like verbs in accepting nuclear stress from their unstressable complements:


\ea%5
\label{ex:richards:5}
\ea  The car crashed \textsc{into} it/me/you...
\ex  The car crashed \textsc{into} something.
\ex  (John spent most of the race well away from the perimeter wall, but then...)    he crashed \textsc{into} the wall.
\z
\z



\ea%6 
\label{ex:richards:6}
\ea The hikers climbed up the \textsc{mountain}.
\ex The hikers climbed \textsc{up} it.
\ex The hikers climbed \textsc{up} something.
\ex (I spent my whole childhood admiring the mountain from afar, and today...) I         climbed \textsc{up} the mountain.
\z\z


\ea%7
    \label{ex:richards:7}
\ea Mary is walking beside the \textsc{car}.
\ex Mary is walking \textsc{beside} it/me/you...
\ex Mary is walking \textsc{beside} something.
\ex (Mary ran up to the car at the head of the parade, and now...)       she is walking \textsc{beside} the car.
\z
\z

\ea%8
    \label{ex:richards:8}
\ea The waves crashed against the \textsc{rock}.
\ex The waves crashed \textsc{against} it.
\ex The waves crashed \textsc{against} something.
\ex (I threw a rock towards the ocean as I left, and turned to watch as...)       the waves crashed \textsc{against} the rock.
\z
\z

The examples in (\ref{ex:richards:5}--\ref{ex:richards:8}) do not all have identical stress patterns; to my ear, the verbs in (\ref{ex:richards:7}--\ref{ex:richards:8}) have a secondary stress which the corresponding verbs in (\ref{ex:richards:5}--\ref{ex:richards:6}) lack, perhaps reflecting adjunct status for the PPs in (\ref{ex:richards:7}--\ref{ex:richards:8}).\footnote{We might imagine, for example, that these adjunct PPs represent a stress domain of their own, unlike argument PPs, and that the verb receives nuclear stress in its domain, while the object of the \isi{preposition} receives sentence-level nuclear stress.}

Another set of prepositions, all of which are plausibly arguments of some kind, pass nuclear stress to the verb when their object is incapable of bearing it:


\ea%9
    \label{ex:richards:9}
\ea They're talking about the \textsc{movie}.
\ex They're \textsc{talking} about it/me/you...
\ex They're \textsc{talking} about something.
\ex (They just got back from a movie, and now...) they're \textsc{talking} about the movie.
\z
\z

\ea%10
    \label{ex:richards:10}
\ea They're fighting over the \textsc{food}.
\ex They're \textsc{fighting} over it/me/you...
\ex They're \textsc{fighting} over something.
\ex (They bought a bunch of food to cook together, but now...)       they're \textsc{fighting} over the food.
\z
\z


\ea%11
    \label{ex:richards:11}
\ea They're running from the \textsc{predator}.
\ex They're \textsc{running} from it/me/you...
\ex They're \textsc{running} from something.
\ex (The ecologists came here to study a certain predator, but now...)      they're \textsc{running} from the predator.
\z
\z

The facts in (\ref{ex:richards:5}--\ref{ex:richards:11}) suggest that the distribution of stress is at least partly relatable to the verb's argument structure; (\ref{ex:richards:9}--\ref{ex:richards:11}) involve verbs which are plausibly unergative\is{ergative}, while the verbs in (\ref{ex:richards:5}--\ref{ex:richards:6}) may be unaccusative (and the PPs in (\ref{ex:richards:7}--\ref{ex:richards:8}) could be adjuncts).  Some prepositions seem to vary in their behavior depending on whether a direction or location meaning is intended:\footnote{Compare \citegen[231--232]{Wagner2005} observation about a contrast in stress placement in \ili{German}:
\ea
\ea
\gll Sie hat im   Garten  \textsc{getanzt}.    \\
she has in.the.\textsc{dat} garden  danced \\
\glt   'She danced in the garden.'

\ex 
\gll Sie ist in  den  \textsc{Garten} getanzt.\\
    she is  in  the.\textsc{acc}  garden   danced\\
\glt    'She danced into the garden.'
\z
\z

The version of the sentence with the location meaning, in (ia), has nuclear stress on the verb, while the \isi{directional} meaning in (ib) gives nuclear stress on the object of the \isi{preposition}.  I find a corresponding contrast in English when the object of the \isi{preposition} is a pronoun:

\ea\upshape
\ea She \textsc{danced} in it.
\ex She danced \textsc{into} it.
\z
\z
}


\ea%12
    \label{ex:richards:12}
\ea They're walking under the \textsc{trees}.  [location]
\ex They're \textsc{walking} under them.
\ex They're \textsc{walking} under something.
\ex (There's a nice grove of trees behind my house, so whenever I need to take a         break...) I \textsc{walk} under the trees.
\z
\z


\ea%13
    \label{ex:richards:13}
\ea They're walking under the \textsc{awning}.  [direction]
\ex They're walking \textsc{under} it.
\ex They're walking \textsc{under} something.
\ex (It was just starting to rain as they made it to the town square and saw an awning,      so...) they walked \textsc{under} the awning.
\z
\z

Some temporal PPs seem to me to be ill-formed with pronoun or indefinite objects:


\ea%14
    \label{ex:richards:14}
\ea John has been sleeping since the \textsc{party}.
\ex     *  John has been \textsc{sleeping} since it.
\ex      *  John has been \textsc{sleeping} since something.
\z
\z

On the other hand, immediately anaphoric DPs do not sound bad:


\ea%15
    \label{ex:richards:15}
	  (Most of us really enjoyed the party, but John sort of overdid it, and...)       he's been \textsc{sleeping} since the party.
\z


With this exception, the behavior of the various nuclear-stress-avoiding nominals seems to be more or less parallel in all of the above examples.  In structures in which nuclear stress would ordinarily appear on a pronoun, a simple indefinite, or an immediately anaphoric DP, stress is shifted instead to a verb or a \isi{preposition}.  The choice between stress on a verb and stress on a \isi{preposition} is apparently determined by a number of factors, possibly including argument/adjunct status of the PP and the argument structure of the verb.

  Developing a predictive theory of these facts is, again, beyond the scope of this paper.  Descriptively, just for these data, we could say that nuclear stress that is shifted from the object of a \isi{preposition} is realized on the verb just when the PP is an argument and the verb is unergative\is{ergative}, and is otherwise realized on the \isi{preposition}.\footnote{The account sketched below does not extend to the behavior of prepositions like \textit{since} in \REF{ex:richards:14}.  Perhaps this \isi{preposition}, not unlike the pronoun \textit{it}, is simply incapable of bearing stress.}  An imaginable account of these facts might divide the VP into domains in which stress may be realized, with domain boundaries around adjunct PPs and the underlying position of subjects (perhaps because stress-shifting takes place before unaccusative subjects move out of the VP, or at least before the \isi{trace} of such movement has been definitely determined to be phonologically null).  The general principle would then be that stress is shifted as far left as possible within the domain to which stress would ordinarily be assigned:

\settowidth\jamwidth{[\textit{unaccusative verb}]}
\ea%16
    \label{ex:richards:16prim} %example \isi{number} 16 doubly assigned
\ea They're {\textbar} \textsc{talking} about it. {\textbar}
\ex The car {\textbar} crashed {\textbar} \textit{the car}\textit{} {\textbar} \textsc{into} it. {\textbar}    \jambox{[\textit{unaccusative verb}]}
\ex She's {\textbar} walking {\textbar}\textsc{ beside} it. {\textbar}{\textbar}      \jambox{[\textit{adjunct PP}]}
\z
\z

For what follows, however, the only important point of this section will be that the various types of nominals that avoid nuclear stress are generally treated in the same way.

\subsection{Nuclear stress and antipronominal contexts}

There are contexts in which pronouns are treated differently from both simple indefinites and anaphoric DPs:


\ea%16
    \label{ex:richards:16bis}
\ea Frank turned into a \textsc{frog} (after drinking the potion).
\ex Frank turned into \textsc{me}.
\ex Frank turned \textsc{into} something.
\ex (Frank's always been afraid of frogs, which is ironic, because today...)       he turned \textsc{into} a frog.
\z
\z

The object of \textit{into} in \REF{ex:richards:16bis} is one of Postal's (\citeyear{Postal1994,Postal1998,Postal2001}) examples of what he calls an \textit{antipronominal context}, by which he means a context in which unstressed pronouns cannot appear.  Pronouns like \textit{it}, which cannot bear stress unless focused, are impossible in such contexts; \REF{ex:richards:17} is an odd response to a \isi{question} like  \textit{Where did that frog come from?}:


\ea%17
    \label{ex:richards:17}
       *  Frank turned into it.
\z

As \REF{ex:richards:16bis} shows, simple indefinites and anaphoric DPs are unlike pronouns in this context, in that they can appear without stress, with nuclear stress appearing, in this case, on the \isi{preposition}.  

  The contrast in \REF{ex:richards:16bis} seems to be general in Postal's antipronominal contexts (though in some of them, at least for me, the only pragmatically reasonable pronouns are inanimates that are incapable of bearing stress at all, yielding ungrammaticality rather than a stressed pronoun).  Examples include:


\ea%18
\label{ex:richards:18}
\ea Frank became a \textsc{frog}.
\ex Frank became \textsc{me}.
\ex Frank \textsc{became} something.
\ex (Frank's always been afraid of frogs, which is ironic, because today...)       he \textsc{became} a frog.
\z
\z

\ea%19
    \label{ex:richards:19}
\ea Frank is \textsc{Hamlet}.
\ex Frank is \textsc{me} (that is, I am a character in the play, and Frank plays me).
\ex \label{ex:richards:19c} \textsc{Frank} is somebody (that is, Frank has a part in the play).
\ex (Frank has always wanted to play the murderer, and in this play,)       Frank \textsc{is} the murderer.\footnote{Here stress is on the verb, rather than on the subject as in \REF{ex:richards:19c}, perhaps because the subject is also immediately anaphoric.} 
\z
\z



\ea%20
    \label{ex:richards:20}
  \ea The movie stars \textsc{Madonna}.
\ex The movie stars \textsc{me}.
\ex The movie doesn't \textsc{star} anyone.
\ex (The director swore to the producers that he wouldn't cast his wife, but in the       end,) the movie \textsc{stars} his wife.
\z
\z


\ea%21
    \label{ex:richards:21}
\ea Those facts mean [that he is \textsc{guilty}].
\ex    * (These facts may not mean that he's guilty, but) those facts mean it.\footnote{Compare (i), with a different meaning for \textit{mean}:
  \begin{exe}
  \ex 
    I \textsc{mean} it.
  \end{exe}
By (i), the speaker means something like “I was serious in what I just said.”}\textsuperscript{,}\footnote{Note that the problem here is not just that \textit{it} cannot refer to the clausal complement of \textit{mean}:

  \begin{exe}
  \ex
    Those facts mean that he's guilty, though he denies it.
  \end{exe}
}
\ex Those facts \textsc{mean} something.
\ex (She's spent a decade fighting to avoid the end of the company, but in the end,)       these facts \textsc{mean} the end of the company.
\z
\z


\ea%22
    \label{ex:richards:22}
    
\ea[]{Those remarks betrayed disregard for human \textsc{rights}.}
\ex[*]{ Those remarks betrayed it.{\footnote{\citet[226]{Postal2001} claims that this example would be well-formed with a stressed \textit{them} as its object.  The examples crucially involve \textit{betray} with a meaning something like 'inadvertently reveal', and contrast with similar examples in which \textit{betray} has a meaning more like `treat treacherously', which are not antipronominal:
\begin{exe}
\ex He betrayed his \textsc{country}.
\ex He \textsc{betrayed} it/me...
\ex He \textsc{betrayed} someone.
\end{exe}}}}
\ex[]{Those remarks \textsc{betrayed} something.}
\ex[]{(John has spent his whole life fighting against widespread disregard for human       rights, but, actually, in the end...)       his remarks \textsc{betrayed} disregard for human rights.}
\z
\z

\ea%23
    \label{ex:richards:23}
\ea[]{The Porsche cost \textsc{fifty thousand dollars.}}
\ex[*]{The Porsche cost it.}
\ex[]{The Porsche \textsc{cost} something.}
\ex[]{(When he saw my car, he said he'd give half his fortune for it, and in the end,)      the Porsche \textsc{cost} half his fortune.}
\z
\z

  I will not attempt to formally characterize the distribution of antipronominal contexts in this paper; see \citet{Stanton2016}, in particular, for one proposal.  All of the examples above have subjects that are non-agentive, and in many cases inanimate, and several of the predicates are clearly unaccusative, some with transitive variants that have the same antipronominal positions as their intransitive versions:


\ea%24
    \label{ex:richards:24prim} %the example numbers 24 to 27 are doubly assigned.
\ea She turned John into a \textsc{frog}.
\ex She turned John into \textsc{me}/*it.
\ex She turned John \textsc{into} something.
\z
\z

At the end of \sectref{sec:richards:2.1} above I sketched an account of the behavior of stress shift that suggested that unaccusative subjects can block shift of nuclear stress from the object of a \isi{preposition} to the verb:


\ea%25
    \label{ex:richards:25prim}
    
  The car {\textbar} crashed {\textbar}\textit{the car}{\textbar} \textsc{into} it.{\textbar}
\z

On this view, examples like (18b-c) above might have relevant representations like those in \REF{ex:richards:26prim}:


\ea%26
    \label{ex:richards:26prim}
  \ea  \label{ex:richards:26prima} Frank {\textbar} became {\textbar} \textit{Frank} {\textbar} \textsc{me}. {\textbar}
\ex   \label{ex:richards:26primb} Frank {\textbar} \textsc{became} {\textbar} \textit{Frank} {\textbar} something. {\textbar}
\z
\z

Antipronominal status would then reflect the grammar's attempts to realize the nuclear stress that would ordinarily appear on the last object; the grammar must choose between stressing a stress-avoiding nominal (as it does with the pronoun in \REF{ex:richards:26prima}) and shifting stress across the base position of the subject (as in \ref{ex:richards:26primb}).  I have nothing insightful to say about why the grammar makes the choices that it does in this case.  Moreover, this cannot be the whole story, not least because there are examples with the relevant argument structure which lack antipronominal properties; contrast \REF{ex:richards:27prim} with \REF{ex:richards:24prim}:


\ea%27
    \label{ex:richards:27prim}
  \ea She turned John against his \textsc{country}.
\ex She turned John \textsc{against} me/it.
\ex She didn't turn John \textsc{against} anything.
\z
\z

  The speculations of the last paragraph are logically separable from the rest of the account.  All that will truly be relevant for our purposes is that antipronominal contexts have a special status with respect to nuclear stress.  In general, as we saw in (\ref{ex:richards:3}--\ref{ex:richards:15}) above, nuclear stress is shifted from pronouns, simple indefinites, and anaphoric DPs to some preceding head, either a verb or a \isi{preposition} in all the examples considered.  Just in antipronominal contexts, we have now seen, this general pattern breaks down; simple indefinites and anaphoric DPs may shift stress away from themselves, but pronouns are obligatorily stressed (and if the pronoun is one which cannot receive stress, like \textit{it}, the result is ungrammaticality).

\subsection{Summary}

The previous two sections have sketched the lay of the land.  We have seen that in English, nuclear stress is generally retracted from pronouns, simple indefinites, and immediately anaphoric DPs, either to the verb or to a \isi{preposition} (with the choice between verbs and prepositions conditioned by factors that I have not tried to explore in any depth).  We have also seen that just in Postal’s antipronominal contexts, this retraction of stress is blocked specifically for pronouns, which must receive stress.

  The following sections will build on this understanding of the behavior of English nuclear stress.  In section 3 below, we will review Postal’s (\citeyear{Postal1994,Postal1998,Postal2001}) observation that some forms of \isi{A-bar movement} are blocked out of anti-pronominal contexts: following much work, we can understand this as evidence that the forms of \isi{A-bar movement} in {question} leave behind null pronominals at the \isi{extraction} site, and that these null pronominals are incapable of receiving nuclear stress.  \sectref{sec:richards:4} will turn to \citegen{Bresnan1971} observations about the interactions of various forms of \isi{A-bar movement} with the placement of nuclear stress.  We will see that Bresnan’s insights, just like Postal’s, may be described as an observation that certain forms of \isi{A-bar movement}, but not others, leave behind null pronominals, which behave like pronominals for purposes of nuclear stress placement.  Moreover, we will see that Bresnan and Postal identify the same set of A-bar extractions as the ones that leave null pronominals at the \isi{extraction} site. 

\section{Avoiding nuclear stress}

\citet{Postal1994,Postal1998,Postal2001} notes that some but not all forms of \isi{A-bar movement} are blocked out of anti-pronominal contexts:

\settowidth\jamwidth{[\textit{non-restrictive relative}]}
\ea%24
    \label{ex:richards:24bis}
\ea[]{ What kind of frog did he turn into?                                                                       \jambox{[\textit{wh-movement}]             }}
\ex[]{ Any frog that he turns into is going to be poisonous.                                                     \jambox{[\textit{restrictive relative}]    }}
\ex[]{ The best kind of frog to turn into is a poison dart frog.                                                 \jambox{[\textit{infinitival relative}]    }}
\ex[]{ When we were in wizard school together, you always disliked  whatever kind of frog you turned into.  \jambox{[\textit{free relative}]           }     }
\ex[*]{      This frog, which John turned into, is ugly.                                                       \jambox{[\textit{non-restrictive relative}]}  }
\ex[*]{     Frogs, her victims always turn into.                                                               \jambox{[\textit{topicalization}]          }  }
\ex[*]{      Not a single frog did any of her victims turn into.                                               \jambox{[\textit{negative inversion}]      }  }
\ex[*]{      What kind of frog did he eat before turning into?                                                 \jambox{[\textit{parasitic gap}]           }  }
\ex[*]{       That kind of frog is too ugly to turn into.                                                      \jambox{[\textit{gapped degree phrase}]    }  }
\ex[*]{       That kind of frog would be tough to turn into.                                                   \jambox{[\textit{tough-movement}]          }  }
\z
\z

Postal refers to the types of \isi{extraction} in (\ref{ex:richards:24bis}a-d) as A-extractions, and the ones in (\ref{ex:richards:24bis}e-j) as B-extractions.  He proposes (and see \citealt{Perlmutter1972,Cinque1990,O’Brien2015}, and \citealt{Stanton2016} for related proposals and important discussion) that the empty category left behind by B-\isi{extraction} is a pronoun, and hence banned from appearing in antipronominal contexts.

  Given the facts about nuclear stress reviewed in the last section, we can understand the ban on B-\isi{extraction} out of antipronominal contexts as an consequence of some condition like \REF{ex:richards:25bis}:


\ea%25
    \label{ex:richards:25bis}
  	  Phonologically null material may not receive nuclear stress.
\z


Recall that what distinguishes antipronominal contexts from the other positions of nuclear stress that we considered is that in antipronominal contexts, pronouns must receive nuclear stress (though other nuclear stress-avoiding elements need not).  B-\isi{extraction} out of an antipronominal context, then, will leave a pronominal empty category behind in a necessarily stressed position, in violation of \REF{ex:richards:25bis}.  A-\isi{extraction} from an antipronominal context, by contrast, will leave behind some kind of empty category that is not a pronoun, and nuclear stress may safely be shifted away from the \isi{extraction} site.  

This way of thinking about the facts involves thinking of \REF{ex:richards:25bis} as a filter imposed by the PF interface on a syntactic representation in which nuclear stress has been irrevocably assigned.  We have seen that the position of nuclear stress is apparently sensitive to a number of syntactic and semantic\is{semantics} notions; the preceding informal discussion has made {reference} to ‘unaccusatives’, to ‘adjuncts’, to ‘indefinites’, and so forth.  In every case in which a phrase is unable to bear stress for these semantic\is{semantics} and syntactic reasons, stress is shifted to another element.  It is not hard to imagine a system in which \REF{ex:richards:25bis} is enforced by shifting nuclear stress at PF away from phonologically null elements to which it might otherwise be assigned; such a system might treat phonologically null elements, for example, the way simple indefinites are treated.  But if we are to use \REF{ex:richards:25bis} to rule out B-\isi{extraction} from antipronominal contexts, then it must not in fact be possible for the grammar to shift stress in this way.  Rather, nuclear stress is assigned to a structure in which, perhaps, nothing is phonologically null, and PF then filters out structures in which nuclear stress has been assigned inappropriately.  In other words, stress is apparently assigned to a structure that contains syntactic and semantic\is{semantics} information (``unaccusative'', ``indefinite''), but not phonological information (``phonologically null'').  For \isi{extraction} from positions of nuclear stress to succeed, the phrases left behind by movement must be safely free of nuclear stress, not simply because they are phonologically null, but because they belong to the categories of elements that avoid nuclear stress when overt (pronouns, simple indefinites, anaphoric DPs).  The ban on B-\isi{extraction} from antipronominal contexts, on this view, represents a failure to avoid the effects of \REF{ex:richards:25bis}; B-\isi{extraction} leaves behind a pronoun, which is ordinarily enough to avoid nuclear stress, but just in antipronominal contexts, nuclear stress is assigned to the null pronoun in the \isi{extraction} site, violating \REF{ex:richards:25bis}.  To the extent that the account outlined above is successful, it represents an argument for approaches to the interface between syntax and phonology in which the building of phonological representations begins during the narrow syntax (e.g., \citealt{Bresnan1971}, \citealt{Richards2016}).  

\section{Nuclear stress and movement}\label{sec:richards:4}

\citet{Bresnan1971} also observes a difference between kinds of \isi{extraction}, this one having to do with effects on nuclear stress.  In particular, she points out that \isi{wh-movement} and restrictive \isi{relative} formation (both A-extractions) interact with nuclear stress differently from tough-movement (a B-\isi{extraction}).  Consider the \isi{wh-movement} examples in \REF{ex:richards:26bis} (\citealt{Bresnan1971}, 259):


\ea%26
    \label{ex:richards:26bis}
    
\ea\label{ex:richards:26bisa} What has Helen \textsc{written}?
\ex\label{ex:richards:26bisb} What \textsc{books} has Helen written?
\z
\z

As Bresnan points out, the wh-phrases have effects on the position of nuclear stress like those of corresponding indefinites:


\ea%27
    \label{ex:richards:27bis}
\ea\label{ex:richards:27bisa} Helen has \textsc{written} something.
\ex Helen has written some \textsc{books}.
\z
\z

Bresnan concludes that the derivation of \REF{ex:richards:26bis} involves assignment of nuclear stress prior to \isi{wh-movement}.  The simple wh-phrase in \REF{ex:richards:26bisa}, like the simple indefinite in \REF{ex:richards:27bisa}, triggers shift of nuclear stress to the verb, and the complex expressions in the (b) examples receive nuclear scope in the standard way for direct objects.  

  The \isi{relative} clauses in \REF{ex:richards:28}, Bresnan suggests, can be given a similar analysis (\citealt[259]{Bresnan1971}):


\ea%28
    \label{ex:richards:28}
  \ea George found someone he'd like you to \textsc{meet}.
\ex George found some \textsc{friends} he'd like you to meet.
\z
\z

Assuming a raising analysis of \isi{relative} clauses, Bresnan proposes an account of the contrast in \REF{ex:richards:28} which is essentially identical to the one developed for the parallel \isi{wh-movement} facts in \REF{ex:richards:26bis} above; nuclear stress is assigned before the head of the \isi{relative} clause becomes external, and the contrast in \REF{ex:richards:28} is to receive the same account as the one in \REF{ex:richards:29}:


\ea%29
    \label{ex:richards:29}
\ea He'd like you to \textsc{meet} someone.
\ex He'd like you to meet some \textsc{friends}.
\z
\z

  On the other hand, Bresnan claims that \textit{tough-}movement behaves differently from \isi{wh-movement} and relativization, never assigning nuclear stress to the \textit{tough}{}-moved phrase (\citealt[265]{Bresnan1971}):


\ea%30
    \label{ex:richards:30}

  	  That theorem was tough to \textsc{prove}.
\z

Bresnan assumes that \textit{tough}{}-movement involves literal movement of the promoted subject (here, \textit{that theorem})\textit{} from the gap site.  On this assumption, the position of nuclear stress in \REF{ex:richards:30} demonstrates that \textit{tough}{}-movement, unlike \isi{wh-movement} and relativization, bleeds assignment of nuclear stress.  

  We can contrast \REF{ex:richards:30} with the examples in \REF{ex:richards:31}:


\ea%31
    \label{ex:richards:31}
  
\ea What \textsc{theorem} did she prove?

\ex George found a \textsc{theorem} he'd like you to prove.
\z
\z

The facts in \REF{ex:richards:31} are familiar from the discussion of \isi{wh-movement} and relativization above, and both involve, in Bresnan's terms, assignment of nuclear stress to a moved phrase prior to movement.  In \REF{ex:richards:30}, by contrast, it is the verb that is stressed.  Bresnan takes the (\ref{ex:richards:30}--\ref{ex:richards:31}) contrast as evidence that while \isi{wh-movement} and relativization take place after assignment of nuclear stress, \textit{tough}{}-movement precedes nuclear stress assignment.

  Another approach to \REF{ex:richards:30}, given the observation that \textit{tough}{}-movement, like other B-extractions, cannot take place out of antipronominal contexts, would be to follow Postal, O’Brien, and Stanton in claiming that the empty category left behind by \textit{tough}{}-movement is a pronoun.  The position of stress in \REF{ex:richards:30}, on this account, has the same explanation as the position of stress in \REF{ex:richards:32}:


\ea%32
    \label{ex:richards:32}
	  It's tough to \textsc{prove} it.
\z


In both \REF{ex:richards:32} and \REF{ex:richards:30}, the object of \textit{prove} is a pronoun, and nuclear stress therefore appears on the verb.  Antipronominal contexts, we have now seen, are just those contexts in which nuclear stress appears on pronouns, and B-\isi{extraction} from such contexts is therefore ruled out by the general ban on nuclear stress on phonologically null material.  

\subsection{A-extractions, B-extractions, and nuclear stress  }

In this section I will outline an account of the effects of different types of \isi{extraction} on stress.  The account will be based on the following generalizations:

\ea%33
    \label{ex:richards:33}
    
\ea B-extractions leave null pronouns behind in the \isi{extraction} site, which must not be     stressed.

\ex A-extractions leave behind full DPs in the \isi{extraction} site.

\ex Nuclear stress assigned to the tail of a movement chain may be realized on the head.  
\z
\z

Let us consider kinds of extractions in turn, beginning with A-\isi{extraction}.  What type of empty category does A-\isi{extraction} leave behind?

  We have some reason to think that the answer to this {question} may vary depending on what is A-extracted.  As we have seen, Bresnan observes that the position of nuclear stress in clauses with A-extractions depends on the nature of the moved phrase (\citealt[259]{Bresnan1971}):


\ea%36
    \label{ex:richards:36}
\ea\label{ex:richards:36a} What has Helen \textsc{written}?
\ex\label{ex:richards:36b} What \textsc{books} has Helen written?
\z
\z

This contrast is subject to two possibly related complications, which we will need to bear in mind as discussion proceeds.  One has to do with an asymmetry in the judgments in \REF{ex:richards:36}.  For me, at least, \REF{ex:richards:36a} does indeed represent the only possible position of nuclear stress, unless some other constituent (like \textit{Helen}) is being contrastively focused.  The judgment in \REF{ex:richards:36b}, on the other hand, is not nearly as sharp; nuclear stress on the verb in \REF{ex:richards:36b} is not obligatory, as it is in \REF{ex:richards:36a}, but for me at least it is certainly an option.

  The other complication about the contrast in \REF{ex:richards:36} has to do with the distance over which the contrast can be maintained.  We can find instances of the contrast over a fairly large structural distance, I think:


\ea%37
    \label{ex:richards:37}
    
\ea What do you think Helen will claim she has \textsc{written}?

\ex What \textsc{books} do you think Helen will claim she has written?
\z
\z

However, at least for me, the contrast vanishes if \isi{wh-movement} takes place across any constituent that contains its own nuclear stress, like a complex subject:\footnote{The facts are reminiscent of the suggestion in footnote 2 above, that nuclear stress shift from PP to the verb is blocked in unaccusatives, perhaps because the subject intervenes between the verb and the PP at the relevant level of representation.  Perhaps nuclear stress can never be shifted across an intervening sentential stress.}


\ea%38
    \label{ex:richards:38}
  
\ea What has [every author with an interest in \textsc{history}]       claimed to have \textsc{written}?

\ex What \textsc{book} has [every author with an interest in \textsc{history}]       claimed to have \textsc{written}?
\z
\z

Bearing these caveats in mind, let us consider the distribution of Bresnan's effect.

 Bresnan notes that there are complex wh-phrases that do not attract nuclear stress.  Her example is in \REF{ex:richards:39} (\citealt[259]{Bresnan1971}):


\ea%39
    \label{ex:richards:39}
	  Which books has John \textsc{read}?
\z


Bresnan claims that ``the \isi{interrogative} \textit{which} is inherently contrastive'', and that \REF{ex:richards:39} involves contrasting \textit{read} with some other activity (such as \textit{skim}).  It is not clear to me that contrastive focus on the verb is necessary; \REF{ex:richards:39} is a natural way to ask a \isi{question} about a known list of books, for example.  On the other hand, if there is no particular list of books under discussion, and I am asking just because I think finding out which books someone has read is a good way to learn about them, then \REF{ex:richards:40} sounds better:


\ea%40
    \label{ex:richards:40}
  	  Which \textsc{books} has John read?  
\z


  The key examples in \REF{ex:richards:41} can be understood as having derivations that begin as in \REF{ex:richards:42}, with nuclear stress as marked:


\ea%41
    \label{ex:richards:41}
  
\ea What has Helen \textsc{written}?

\ex What \textsc{books} has Helen written?

\ex Which books has John \textsc{read}?    [preestablished set of books]

\ex Which \textsc{books} has John read?    [no preestablished set]

\z
\z

\ea%42
    \label{ex:richards:42}

\ea{} [\textit{\textsubscript{v}}\textsubscript{P} Helen \textsc{write} something]

\ex{}  [\textit{\textsubscript{v}}\textsubscript{P} Helen write some \textsc{books}]

\ex{}  [\textit{\textsubscript{v}}\textsubscript{P} John \textsc{read} the books]    [preestablished set of books]

\ex{}  [\textit{\textsubscript{v}}\textsubscript{P} John read the \textsc{books}]    [no preestablished set]
\z
\z

Wh-movement will then create a new copy of the object at the edge of \textit{v}P, simultaneously converting the object to a wh-phrase, perhaps via Late \isi{Merge} as in \citet{Stanton2016}:

\vspace*{\baselineskip}
\ea%43
    \label{ex:richards:43}
  
\ea{} \label{ex:richards:43a} [\textsubscript{\textit{v}P} \tikz[remember picture,baseline] \node[anchor=base,inner sep=0pt] (richardsWhat) {what}; Helen \textsc{write} \tikz[remember picture,baseline] \node[anchor=base,inner sep=0pt] (richardsSomething) {something};]
\vspace*{.5\baselineskip}
\ex{} \label{ex:richards:43b} [\textsubscript{\textit{v}P} what \tikz[remember picture,baseline] \node[anchor=base] (richardsBooks) {\textsc{books}}; Helen write \tikz[remember picture,baseline] \node[anchor=base] (richardsSome) {some}; books]
\vspace*{.5\baselineskip}
\ex{} \label{ex:richards:43c} [\textsubscript{\textit{v}P} which \tikz[remember picture,baseline] \node[anchor=base] (richardsBooks2) {books}; John \textsc{read} the \tikz[remember picture,baseline] \node[anchor=base] (richardsBooks3) {books};]    [preestablished set of books]
\vspace*{.5\baselineskip}
\ex{} \label{ex:richards:43d} [\textsubscript{\textit{v}P} which \tikz[remember picture,baseline] \node[anchor=base] (richardsBooks5) {\textsc{books}}; John read the \tikz[remember picture,baseline] \node[anchor=base] (richardsBooks4) {books};]    [no preestablished set]
\z
\z
\begin{tikzpicture}[overlay, remember picture]
 \draw[-{Stealth[]}] (richardsSomething) -- ++(0,\baselineskip) -| (richardsWhat);
 \draw[-{Stealth[]}] (richardsSome) -- ++(0,\baselineskip) -| (richardsBooks);
 \draw[-{Stealth[]}] (richardsBooks3) -- ++(0,\baselineskip) -| (richardsBooks2);
 \draw[-{Stealth[]}] (richardsBooks4) -- ++(0,\baselineskip) -| (richardsBooks5);
\end{tikzpicture}


In \REF{ex:richards:43a} and \REF{ex:richards:43c}, the direct object of the verb is unstressable as soon as it is introduced (by virtue of being a simple indefinite, in \REF{ex:richards:43a}, and anaphoric on preceding discourse, in \REF{ex:richards:43c}), and nuclear stress therefore appears on the verb.  

  In \REF{ex:richards:43b} and \REF{ex:richards:43d}, by contrast, the direct object receives stress, and this stress is realized on the head of the chain.  Currently popular approaches to movement generally arrive at the conclusion that the moved element is effectively in two places at once.  If movement involves, for example, the creation of a multidominance structure, then in an example like \REF{ex:richards:43d} there is literally a single DP node with two mothers, one inside the VP and another inside the phrase \textit{which book}.  Similarly, on the copy theory of movement, movement in \REF{ex:richards:43d} creates two instances of a single object \textit{which book}, and we must understand phonology as being directed to pronounce the higher of these two instances.  On either account, we can say that the factors that determine the pronounced position of the displaced element dictate, not only that the segments that make up the DP are to be pronounced near the beginning of the clause, but that the nuclear stress assigned to the DP by virtue of its position as an object is to be realized in its pronounced position.  

  Contrast the derivation for B-extractions like the ones in \REF{ex:richards:44}:


\ea%44
    \label{ex:richards:44}
  
\ea Something was tough to \textsc{prove}.

\ex Some theorems were tough to \textsc{prove}.

\ex The theorems were tough to \textsc{prove}.
\z
\z

Derivations for all of these examples will begin with the \textit{v}P in \REF{ex:richards:45}:


\ea%45
    \label{ex:richards:45}
  	  [\textit{\textsubscript{v}}\textsubscript{P} PRO \textsc{prove} it]
\z

Since the object in \REF{ex:richards:45} is a pronoun, nuclear stress appears on the verb.  Subsequent movement might then convert the pronoun into one of the various kinds of subjects in \REF{ex:richards:44}, but will have no effect on nuclear stress; the object left behind by \textit{tough}{}-movement has no stress that can be realized on the head of its chain, and nuclear stress will therefore invariably be on the verb, as desired.

  These two types of derivations will have different consequences for \isi{extraction} from an antipronominal position.  A-\isi{extraction}, as desired, will be able to shift stress away from the \isi{extraction} site:


\ea%46
    \label{ex:richards:46prim} %example \isi{number} 46-48 are doubly assigned
  
\ea \label{ex:richards:46prima} 
    What did that Porsche \textsc{cost} ?
\ex 
    \label{ex:richards:46primb}
    How much \textsc{money} did that Porsche cost?
\z
\z


\ea%47
    \label{ex:richards:47prim}
  
\ea That Porsche \textsc{cost} something.

\ex That Porsche cost some \textsc{money}.
\z
\z

The A-extractions in \REF{ex:richards:46prim} will have the underlying forms in \REF{ex:richards:47prim}, with stress being realized either on the verb or on the moved phrase, as desired.  B-\isi{extraction}, on the other hand, is correctly predicted to be impossible:


\ea%48
    \label{ex:richards:48prim}
  
\ea   \label{ex:richards:48prima}     *  Fifty thousand dollars would be tough for that Porsche to cost.

\ex   \label{ex:richards:48primb}     *  It would be tough for that Porsche to cost \textsc{it}.
\z
\z

\REF{ex:richards:48prima} will be doomed by the fact that its derivation must begin with \REF{ex:richards:48primb}, in which the pronoun, by virtue of being in an antipronominal context, must receive stress.  

  Why is the stress on the null pronoun in the \isi{extraction} site in \REF{ex:richards:48prima} not simply transferred to the head of the movement chain, as in an example like \REF{ex:richards:46primb}?  Here I think we can take advantage of a difference between A-\isi{extraction} and B-\isi{extraction}.  The nominal Merged\is{Merge} in object position in \REF{ex:richards:46primb}, by hypothesis, is something like \textit{some money}, and nuclear stress is assigned in the usual way to \textit{money}.  Subsequent operations determine that \textit{money} is to be pronounced, not in object position, but at the head of a movement chain, and we have seen that wherever it is pronounced, \textit{money} retains the nuclear stress assigned to it at an earlier point in the derivation.

  In \REF{ex:richards:48prima}, by contrast, nuclear stress has been assigned, by hypothesis, to \textit{it}.  This pronoun will not be pronounced anywhere; no matter what we think about whether the head of the \textit{tough}{}-movement chain in \REF{ex:richards:48prima} is \textit{fifty thousand dollars} or a null operator in a structurally lower position, the pronoun is pronounced in neither of these positions.  Consequently, the nuclear stress assigned to the pronoun also cannot be realized anywhere else; it must be realized on the null pronoun itself, leading to ungrammaticality.

  The derivations sketched above allow us to capture both Postal's and Bresnan's observations about the different effects of different kinds of \isi{extraction}.  By claiming that B-extractions leave pronouns behind, we account both for the fact that they are banned out of antipronominal contexts and for the fact that they invariably leave nuclear stress on the verb.  A-extractions, by contrast, leave behind non-pronominal DPs, which means both that \isi{extraction} is licit out of antipronominal contexts and that the position of nuclear stress is more variable than it is for B-extractions. 

\subsection{Extending Bresnan's paradigm}

Bresnan discusses a distinction between \isi{wh-movement} and relativization, on the one hand, and \textit{tough}{}-movement, on the other.  Let us consider the other instances of A- and B-\isi{extraction}.

\subsubsection{ A-extractions}

We have already discussed \isi{wh-movement} in some detail:


\ea%46
    \label{ex:richards:46bis}
    
\ea What has Helen \textsc{written}?

\ex What \textsc{books} has Helen written?
\z
\z

Most of Bresnan's examples of relativization have no overt relativization operators, and the position of stress is determined by the head of the \isi{relative} clause:


\ea%47
    \label{ex:richards:47bis}
  
\ea George found someone he'd like you to \textsc{meet}.
\ex George found some \textsc{friends} he'd like you to meet.
\z
\z

These facts are consistent with the account offered above; Bresnan assumes a raising analysis of \isi{relative} clauses, but a matching analysis could presumably also be made consistent with the facts, as long as `matching' is understood as being relevantly like the pronunciation of the head of a movement chain, forcing the pronounced head of the \isi{relative} clause to be phonologically identical to the moved operator.

  Relative clauses with overt operators seem to me to also be consistent with the theory above:


\ea%48
    \label{ex:richards:48bis}
\ea \label{ex:richards:48bisa}George found someone who he'd like you to \textsc{meet}.
\ex \label{ex:richards:48bisb}George found someone whose \textsc{mother} he'd like you to meet.
\ex \label{ex:richards:48bisc}George found some friends who he'd like you to \textsc{meet}.
\ex \label{ex:richards:48bisd} George found some friends whose \textsc{mother} he'd like you to meet.
\z
\z

The readings in \REF{ex:richards:48bis} all follow from the proposal on offer.  In particular, \REF{ex:richards:48bisb} and \REF{ex:richards:48bisd} have the option of not stressing the verb of the \isi{relative} clause, regardless of the form of the head of the \isi{relative} clause, presumably because of the form of the \isi{relative} operator.  Stress may also appear on the verb in these examples, as is standard for A-extractions (see section 4.1 above for discussion).

  \isi{Infinitival} relatives pattern with restrictive relatives (without overt operators):


\ea%49
    \label{ex:richards:49}
\ea George found someone to \textsc{talk} to.
\ex George found some \textsc{friends} to talk to.
\z
\z

  Free relatives seem to me to invariably place nuclear stress outside the operator:


\ea%50
    \label{ex:richards:50}
\ea He'll attack whatever I \textsc{talk} about.
\ex He'll attack whatever theory I \textsc{talk} about.
\z
\z

This is not because free relatives have been misclassified as A-extractions; they are compatible with \isi{extraction} from antipronominal contexts:


\ea%51
    \label{ex:richards:51}
\ea\label{ex:richards:51a} Whatever he becomes, I'll always love him.
\ex I'll pay whatever this Porsche costs.
\ex You won't have heard of whoever this movie stars.
\z
\z

I suspect that free relatives may have their own intonational cues that make them appear to be counterexamples.  Fully understanding their intonation is well beyond the scope of this paper, but for me, there are at least two kinds of intonational ‘tunes’ that may go with free relatives.  In one, which is more or less obligatory in examples like \REF{ex:richards:51a}, there is a high pitch that begins at the end of the \isi{relative} operator and lasts until the end of the free \isi{relative} clause, at which point pitch falls.  The same intonation is possible for me in examples like (51b-c) in which the free \isi{relative} appears in an argument position, as is an alternative featuring a high pitch on the matrix verb, and then a low and flat pitch for most of the \isi{relative} clause, ending in another high pitch at the end of the \isi{relative}.  I conclude that these pitch excursions make the true position of nuclear stress in free relatives difficult to determine.

\subsubsection{B-extractions}

The B-\isi{extraction} already discussed was \textit{tough}{}-movement, which invariably has nuclear stress outside the moved element.  This is true regardless of the properties of the mover:


\ea%52
    \label{ex:richards:52}

\ea That theorem was tough to \textsc{prove}.
\ex What theorem was tough to \textsc{prove}?
\ex It was tough to \textsc{prove}.
\ex I consider some theorems tough to \textsc{prove}.
\z
\z

I have followed Postal, O’Brien, and Stanton in claiming that \textit{tough}{}-movement leaves a null pronominal in the gap site; as a result, regardless of the identity of the moving phrase, nuclear stress will appear on the verb in examples like \REF{ex:richards:52}, just as it would if the object of the verb were an overt pronoun.

  I have said nothing about why \textit{tough}{}-movement differs from the A-extractions in invariably leaving a pronoun in the gap site (see \citealt{Stanton2016} for one proposal).  I think it is fair to say that the syntax of \textit{tough}{}-movement is still not completely clear.  \citet{Bresnan1971} assumed that the subjects of the examples in \REF{ex:richards:52} begin the derivation at the gap site; another well-attested approach to \textit{tough}{}-movement (\citealt{Chomsky1977} and much subsequent work) posits a null operator within the embedded clause, and we could imagine that this operator is a null pronoun.  

  Other B-extractions which reliably place nuclear stress outside the moved phrase include:


\ea%53
    \label{ex:richards:53}

\ea That theorem, I won't \textsc{talk} about.      [\textit{topicalization}\footnote{\citet{Bresnan1971} also notes that topicalization has this property (footnote 18, pp. 276--277); she excludes it from her discussion on the grounds that it has a distinctive intonation that may blur the relevant distinctions, much as I have done for free relatives above.}]
\ex Some theorems, I won't \textsc{talk} about.  
\z
\z

\ea%54
    \label{ex:richards:54}

	    Not a single theorem did she \textsc{talk} about.    [\textit{negative inversion}]
\z

\ea%55
    \label{ex:richards:55}

\ea That theorem is too complicated to \textsc{talk} about.  [\textit{gapped degree phrase}]

\ex Some theorems are too complicated to \textsc{talk} about.
\z
\z


\ea%56
    \label{ex:richards:56}
  
\ea Caligula, [who I've \textsc{talked} about], was a terrible emperor.  [\textit{non-restr. relative}]
\ex Caligula, [whose \textsc{mother} I've \textsc{talked} about], was a terrible emperor.
\z
\z

Parasitic gaps, intriguingly, have the B-\isi{extraction} pattern of stress both for the parasitic gap and for the main gap:


\ea%57
    \label{ex:richards:57}
  	  What books has she \textsc{published} after \textsc{writing}?
\z

And, as \citet{Postal2001} observes, in parasitic gap constructions, both the parasitic gap and the main gap are sensitive to antipronominal contexts:


\ea%58
    \label{ex:richards:58}
    
\ea What color did she paint the house \_\_?

\ex     *  What color did she paint the house \_\_ [after discussing \_\_ with Abigail]?

\ex      *  What color did she discuss \_\_ with Abigail [after painting \_\_ the house]?
\z
\z

Again, sensitivity to antipronominal contexts seems to coincide with the placement of nuclear stress.

  \citet{O’Brien2015} discusses another case in which forms of \isi{extraction} which are ordinarily A-extractions take on the antipronominal-sensitivity of B-\isi{extraction}.  \isi{Extraction}\is{extraction} from \isi{islands}, he points out, makes \isi{wh-movement} and restrictive relativization sensitive to antipronominal contexts:


\ea%59
    \label{ex:richards:59}

  \ea \label{ex:richards:59a}     *  Coppe painted his bike orange, and Ted painted his car \textbf{it}.
\ex\label{ex:richards:59b} That’s the color that I think I’ll paint my bike \_\_.
\ex\label{ex:richards:59c}     *  That’s the color that I don’t know why [anyone would paint their bike \_\_ ].
\ex\label{ex:richards:59d}      ?  That’s the bike that I don’t know why [anyone would paint \_\_ that color].
\z
\z

\REF{ex:richards:59a} reminds us that the color-denoting expression following a verb like \textit{paint} is an antipronominal context.  \REF{ex:richards:59b} demonstrates that restrictive relativization is normally immune to antipronominal contexts.  The contrast in (59c-d) shows that relativization out of a wh-\isi{island} becomes impossible out of antipronominal contexts: \REF{ex:richards:59c}, which takes place out of an antipronominal position, is much worse than \REF{ex:richards:59d}, which is merely somewhat degraded.  I refer interested readers to \citet{O’Brien2015} for much further discussion of the pattern in \REF{ex:richards:59}, which seems to be general across a number of kinds of \isi{islands}.

  \isi{Extraction}\is{extraction} out of \isi{islands} also appears to behave like B-\isi{extraction} for purposes of stress placement, reliably placing stress on the verb:


\ea%60
    \label{ex:richards:60}
  \ea      ?  What books are you wondering [why she would have \textsc{written} \_\_ ]?
\ex     *  What \textsc{books} are you wondering [why she would have written \_\_ ]?
\ex      ?  What book should she have resigned [after \textsc{writing} \_\_]?
\ex     *  What \textsc{book} should she have resigned [after writing \_\_ ]?
\z
\z

I have not offered an account of why \isi{extraction} types that ordinarily pattern with A-\isi{extraction} become B-\isi{extraction} just under particular circumstances, and will not, for reasons of space; see \citet{O’Brien2015} for an account of the \isi{island} facts.  The important observation, for purposes of this paper, is just that the properties of B-\isi{extraction} seem to pattern together reliably.

\subsection{Conclusion}

It seems that Bresnan's observations divide types of \isi{extraction} in the same way as Postal's.  B-extractions leave null pronouns behind in the gap site, and this has consequences both for their distribution (they cannot take place out of antipronominal contexts) and for their effects on nuclear stress (stress is invariably retracted to a verb or \isi{preposition}).  A-extractions, by contrast, leave an anaphoric DP behind; this DP may be safely left in antipronominal contexts, where it can avoid receiving stress, and it will sometimes pass nuclear stress along to the head of the movement chain, and sometimes to a verb or \isi{preposition}, depending on the nature of the moving phrase.  The two types of properties—sensitivity to antipronominal contexts and effects on placement of nuclear stress—appear to pattern together reliably.

\section{ Some other consequences and conclusions}

This paper has been an attempt to demonstrate that various people were right, and that their proposals support each other.  

  In fact, not only were \citet{Postal1994,Postal1998,Postal2001,Stanton2016,O’Brien2015}, and \citet{Bresnan1971} right, but \citet{Fox1999,Fox2003,Sauerland1998,Sauerland2004}, and \citet{Johnson2012} were also right.  Let us turn to this fact next.

  \citet{Chomsky1993} proposes the copy theory of movement, which claims that examples like \REF{ex:richards:61a} should have a syntactic representation like that in \REF{ex:richards:61b}:


\ea%61
    \label{ex:richards:61}    
  \ea \label{ex:richards:61a}  Which book did Mary read?
  \ex  \label{ex:richards:61b} [which book] did Mary read [which book] 
  \z
\z

As Chomsky notes, the structure in \REF{ex:richards:61b} is difficult to interpret as an operator-variable construction.  He posits a later operation altering \REF{ex:richards:61b} to \REF{ex:richards:62}:


\ea%62
    \label{ex:richards:62}
	  which\textsubscript{x} did Mary read book x
\z

  \citet{Fox1999,Fox2003} and \citet{Sauerland1998,Sauerland2004} posit versions of a process which \citet{Fox2003} names \textit{Trace\is{trace} Conversion}, which converts the lower of two copies of movement into a definite description, yielding something like \REF{ex:richards:63} as an LF representation of \REF{ex:richards:61a}:


\ea%63
    \label{ex:richards:63}
      which book x did Mary read [\textbf{the book x}]
\z


  Chomsky, Fox, and Sauerland all assume that the syntax creates a structure with multiple copies of a moving operator, and that the lowest copy is converted to something else postsyntactically, for reasons having to do with interpretation.  If the account developed here is right, the direct object of \textit{read} in (\ref{ex:richards:61}--\ref{ex:richards:63}) has, not only the \isi{semantics} of a definite description, but also some of the phonological properties of a definite description—in particular, it patterns with definite descriptions with respect to the placement of nuclear stress.  If the gap site is a definite description both at LF and at PF, we should probably regard it as a definite description in the narrow syntax.

  This is \citegen{Johnson2012} suggestion, which \citet{O’Brien2015} adopts and extends.  They posit movement operations that create A-bar operators out of definite descriptions, ultimately creating multidominant representations like the one in \REF{ex:richards:64b}:\pagebreak


\ea%64
    \label{ex:richards:64}
\ea Which story about her should no linguist forget?
\ex \label{ex:richards:64b}   \is{complementizer}  
\begin{forest}
 [CP, s sep=2cm[QP\textsubscript{2},name=qp [Q] [,no edge]] [CP [C\\\textbf{should},base=top,align=center] [TP [DP [\textbf{no linguist},roof]] [T' [T] [VP [V\\\textbf{forget},base=top,align=center] [DP,l=4\baselineskip,name=dp [D\\\textbf{which},base=top,align=center [\textit{the}] [2]] [NP [\textbf{story about her},roof]]]]]]]]
 \path (qp.south) edge [bend right,in=210] (dp.south);
\end{forest}
\z
\z

In \REF{ex:richards:64b}, QP and VP both immediately dominate the DP \textit{which story about her}, and Q has converted the D \textit{the} into \textit{which} by Agreeing with it.  Formally similar, though lacking \citegen{Johnson2012} appeal to multidominance, is \citegen{Stanton2016} use of Wholesale Late Merger, which creates operators for B-\isi{extraction} out of pronouns (represented as instances of D) by Late Merging an NP to D.

  Both of these mechanisms have the property, which I think is appealing, of preserving the original insight and virtues of Chomsky's Copy Theory; movement is just another instance of \isi{Merge}.  Where they depart from Chomsky is in identifying the repeatedly Merged\is{Merge} phrase, not as an instance of the A-bar operator, but as a subpart of that operator, expanded on structurally in the course of the movement operation.

  Let me end with three notes for future research.

  The first has to do with covert movement.  \citet{Fox1999,Fox2003} and \citet{Sauerland1998,Sauerland2004} intend Trace\is{trace} Conversion to apply, not only to overt movement, but to QR and wh-in-situ.  I leave for future work the \isi{question} of how the proposal here can be generalized to such cases; I hope that such work will shed further light on the syntax of covert movement.\footnote{The account developed here avoids a `look-ahead' problem connected with wh-in-situ.  Consider pairs like the one in (i-ii):
  
(i)  Which book did the teacher say we should read?

(ii) Which teacher said we should read which book?

\noindent In a bottom-up derivation, (i-ii) will have the same starting point, Merging \textit{which book} as the object of \textit{read}.  On a certain set of assumptions, this poses a look-ahead problem: if movement is successive-cyclic, and if the decision about whether to perform overt successive-cyclic movement of \textit{which book} must be made before the matrix subject is Merged\is{Merge}, then it is difficult to see how this decision can be made without potentially crashing the derivation.  There are various ways of avoiding the look-ahead problem, of course, some of which involve denying the assumptions just sketched (or being at peace with the idea that derivations can crash).  But in the theory offered here, the problem does not arise; the matrix subject is invariably something like \textit{the teacher}.  If \textit{which book} is successive-cyclically moving past it, the grammar must simply refrain from converting \textit{the teacher} into \textit{which teacher}, yielding (i) instead of (ii); if \textit{which book} elects to remain in situ, then the grammar can convert \textit{the teacher} into a wh-phrase, yielding (ii).}

  A second area of future research has to do with languages other than English.  Languages vary in how they treat the kinds of phrases that avoid nuclear stress in English.  Spanish, for example, assigns nuclear stress, when appropriate, to pronouns, simple indefinites, and immediately anaphoric DPs \citep{Zubizarreta1998,Hualde2007,NavaZubizarreta2009,NavaZubizarreta2011}:


\ea%65
    \label{ex:richards:65}

\langinfo{Spanish}{}{Karlos Arregi, p.c.; \citealt[61]{Hualde2007};  \citealt[179]{NavaZubizarreta2009}}\\
\ea 
\gll  He  hablado  con  \textbf{ella}.\\
  I.have  talked  with  her\\
\glt  \textup{‘I’ve} talked \textup{with her}.’

\ex
\gll  Vi  \textbf{algo}.\\
  I.saw  something\\
\glt `I saw \textup{something'}.

\ex
\gll  Por\_qué  compras  ese  sello  tan  viejo? ---  Porque  colecciono  \textsc{sellos}.\\
	why  you.buy  that  stamp  so  old  --- because  I.collect  stamps\\
\glt `Why are you buying that old stamp?' --- `Because I collect stamps.’ 
\z
\z

If the account developed here is correct, and if it is safe to regard Spanish nuclear stress and English nuclear stress as the same phenomenon, then Spanish cannot be using null pronouns for the tails of B-extractions.  We might imagine, for example, that Spanish B-\isi{extraction} chains have null \isi{clitic} pronouns as their tails, since \isi{clitic} pronouns do not bear nuclear stress in Spanish.  This should have consequences for the distribution of B-\isi{extraction} in Spanish (for instance, it should only be possible to B-extract DPs from positions of nuclear stress if they could in principle be \isi{clitic} pronouns).  Investigation of these consequences, and of similar facts in other languages, is a topic I will have to leave for the future.

  Another topic for the future is the phonology of \isi{extraction} of phrases other than DPs.  AP \isi{extraction}, for example, seems to me to be able to participate in alternations of nuclear stress position like those \citet{Bresnan1971} identifies for A-extractions:


\ea%66

\ea\label{ex:richards:66a} How do you \textsc{feel}?
\ex\label{ex:richards:66b} How \textsc{dizzy} do you feel?
\z
\z

An example like \REF{ex:richards:66b}, in particular, allows me to destress the verb, particularly if the topic of dizziness is not yet part of the discourse (imagine, for example, that I see you in some kind of distress and am trying to diagnose your problem; under these circumstances, \REF{ex:richards:66b} seems like a natural way of saying the sentence).  

  Partly for reasons of space, I have had to leave these and many other mysteries largely unexplored.  I have tried to demonstrate that Postal’s division of \isi{A-bar movement} types into A- and B-extractions is mirrored in Bresnan’s observations about different effects of movement on nuclear stress.  I have also argued that this parallelism represents an argument for something like \citegen{Johnson2012} narrow-syntactic rendition of Trace\is{trace} Conversion; heads and tails of chains are different, not only in the semantic\is{semantics} representation, but also in the phonological representation, and we should therefore posit a difference between them in the syntax.
 
\section{Acknowledgements}

Many thanks to Karlos Arregi, Juliet Stanton, David Pesetsky, audiences at MIT and the University of Pennsylvania, and two anonymous reviewers for comments on this work. I'm also very grateful to Anders Holmberg, for everything he's done for the field; it's a pleasure to be able to offer this paper in his honor. Responsibility for remaining errors is mine.

{\sloppy\printbibliography[heading=subbibliography,notkeyword=this]}
\end{document}