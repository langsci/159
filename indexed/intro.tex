\documentclass[output=paper]{langsci/langscibook}
\author{Michelle Sheehan\affiliation{Anglia Ruskin University}\lastand Laura R. Bailey\affiliation{University of Kent}}
\abstract{\noabstract}
\title{Introduction: Order and structure in syntax}

\ChapterDOI{10.5281/zenodo.1117704}
\maketitle
\begin{document}
Hierarchical structure and argument structure are two of the most pervasive and widely studied properties of natural language.\footnote{All of the papers in this volume were written on the occasion of Anders Holmberg’s 65\textsuperscript{th} birthday in recognition of the enormous contribution he has made to these issues.} The papers in this set of two volumes further explore these aspects of language from a range of perspectives, touching on a {number} of fundamental issues, notably the relationship between linear order and hierarchical structure and variation in subjecthood properties across languages. The first volume focuses on issues of word order and its relationship to structure, while the second turns to argument structure and subjecthood in particular. In this introduction, we provide a brief overview of the content of this first volume, drawing out important threads and questions which they raise.   

This first volume, consisting of 12 papers and six squibs, addresses the important {question} of what word order can tell us about syntactic structure and by implication the syntax/\isi{semantics} interface. In some cases, the claim is that (some aspects of) word order should not be encoded in the narrow syntax (Zwart; Haddican \& Extepare; Julien; and Erteschik-Shir \& Josefsson) because PF-based explanations are sufficient or even more explanatory. In other cases, it is claimed that word order gaps are best explained by a theory in which word order is encoded narrow syntactically (Biberauer), and the implications of this for the narrow syntax or syntax/\isi{semantics} interface are explored. 

The first three papers (by Djärv, Heycock \& Rohde; Zwart; and Poole) focus on the \isi{verb second} property (henceforth V2\is{verb second}), which is characteristic of most of the Germanic language family as well as certain diachronic and synchronic \ili{Romance} varieties, where\-by the finite\is{finiteness} verb in matrix (and a subset of embedded) clauses occupies the second position and is (usually) preceded by a single constituent (see \citealt{Holmberg2015verbsecond} for an overview). The papers address either how V2\is{verb second} is derived (Poole, Zwart) or its semantic\is{semantics}/discourse function (Djärv, Heycock \& Rohde), providing novel observations and analyses on a much studied topic. On one hand, Zwart argues that V2\is{verb second} must be a PF phenomenon, based on the fact that auxiliary\is{Auxiliary} verbs undergo V2\is{verb second} movement and yet periphrastic \isi{tenses} must be inserted late in the \isi{morphology}. Poole, on the other hand, argues that V2\is{verb second} in \ili{Old Spanish} is derived in the syntactic manner proposed by \citet{Holmberg2014verbsecond} for the Germanic family: via head and XP movement to the same phrase. Djärv, Heycock \& Rohde focus on the \isi{semantics}\slash \isi{pragmatics} of V2\is{verb second} rather than its syntax and are concerned with establishing the precise distribution of V2\is{verb second} clauses and embedded \isi{root phenomena} more generally, based on novel survey data from \ili{Swedish} and English. These three papers touch on different aspects of the well-studied V2\is{verb second} phenomenon, highlighting very clearly that the connection between order and structure cannot be taken for granted and nor can the mapping between syntactic and semantic\is{semantics}/\isi{pragmatic} structure. The issues addressed in Djärv, Heycock \& Rohde’s paper are taken up again in Nikanne’s paper (chapter 4), which sketches a new way of thinking about word order in \ili{Finnish}, a language that displays complex word order patterns, depending on both \isi{morphology} and information structure. Finally, Sulaiman’s squib on \isi{verb movement} in Syrian \ili{Arabic} argues that although this language is not generally held to be V2\is{verb second}, certain word order patterns are best explained if a similar mechanism to that found in V2\is{verb second} languages is present in this \ili{Arabic} variety. 

The next pair of papers (by Erteschik-Shir \& Josefsson and Woolford) and the squib by Vikner, Christensen \& Nyvad all focus on another curious word order phenomenon: object shift, a process by which some subset of objects undergoes obligatory or optional movement to the left of adverbs\slash \isi{negation} in certain contexts. This phenomenon was studied at length by Anders Holmberg, who observed a curious connection between object shift and \isi{verb movement} in the \ili{Scandinavian} languages (Holmberg’s Generalization; \citealt{Holmberg1986}, \citeyear{Holmberg1999}). Once again, while one paper argues, based on prosodic evidence, that this is a PF operation (Erteschik-Shir \& Josefsson), the other takes it to be syntactic and active in languages well beyond those Germanic languages in which it was first observed (Woolford). Woolford’s paper argues that in Aleut, ergative case occurs wherever the object of V is null because these null pronouns undergo obligatory object shift out of VP, triggering ergative case (see \citealt{woolford2015ergativity}).\largerpage 

Chapter 7–9 focus on a peculiar word order gap (the \isi{Final-over-Final Condition}, henceforth FOFC\is{Final-over-Final Condition}), which was first discovered by \citet{Holmberg2000} and then developed by \citet{biberaueretal2014,BiberauerEtAl2017book}. FOFC\is{Final-over-Final Condition} is based on the observation that a head final phrase cannot dominate a head-initial phrase in the same domain (where different definitions of the relevant notion of domain have been offered). Haddican \& Extepare consider certain word order gaps in \ili{Basque} verb clusters, showing that the repairs which occur raise challenges for a narrow syntactic view of FOFC\is{Final-over-Final Condition}. Biberauer and Julien both discuss the relevance of FOFC\is{Final-over-Final Condition} to the adpositional domain. Biberauer considers the complex adpositional system of \ili{Afrikaans} in the contexts of broader cross-linguistic patterns and defends a narrow syntactic view of FOFC\is{Final-over-Final Condition}. Julien, on the other hand, focuses on data from Sámi, a language which also has both prepositions and postpositions, but argues for a PF-based account, departing from previous approaches.   

Finally, chapters 10–12 and the squibs by Rizzi, Platzack and Kayne focus on word order and other issues connected to the left periphery of the clause. Wiltschko’s and Tsoulas’ contributions focus on questions, answers and responses, showing that complex structures lie behind simple response particles such as \textit{yes} and \textit{no} (see also \citealt{holmberg:15}). While Wiltschko adopts the idea that particles are simplex and their complex meaning arises from the clausal structures into which they are inserted, Tsoulas argues that particles themselves contain internal structure. 

Rizzi’s squib considers the uniqueness condition on focus and whether this effect should be explained by locality or interface conditions. He argues, based on the fact that the uniqueness condition is preserved even in complex sentences containing multiple clauses, that locality based explanations are insufficient. He further shows, however, that locality may be required to rule out word order restrictions between foci and \isi{interrogative} complementisers, the conclusion being that both kinds of explanations may be necessary in order to explain cartographic generalisations. Kayne’s squib adopts an explanation for the different landing sites of \isi{wh-movement} in questions vs. \isi{relative} clauses, in terms of locality. He goes on to show, however, that the derivation of \isi{relative} clauses is more complex than previously thought as it is possible to form \isi{relative} clauses containing multiple wh-phrases. Such examples, he argues, can be accounted for if \isi{relative} pronouns are actually determiners\is{determiner} which get stranded when their NP complement moves to a higher position. Platzack’s squib turns to word order effects in a different kind of wh-clause: wh-root-\isi{infinitive} clauses in \ili{Swedish}. He proposes, based on word order facts and the unavailability of overt subjects, that these kinds of clauses lack a T \isi{projection}. 

Richards’ paper focuses on movement operations and how they contribute to syntactic structure building, bringing together several different strands of research to argue for two distinct kinds of \isi{A-bar movement}: one which leaves a null pronoun and another which leaves a null definite description. 

Lastly, Emonds’ paper uses word order differences between Old and \ili{Middle English} amongst other grammatical differences to further defend \citegen{EmondsFaarlund2014} proposal that Modern English is a North Germanic language. While Old English was an OV language (with some complications), \ili{Middle English} has umarked VO order in both main and dependent clauses. It also has \isi{preposition} stranding, parasitic gaps, subject+\isi{tense} tag questions, all features which it shares with North Germanic but not \ili{West Germanic}. 

The papers in this first volume address different word order-related issues and focus on data from a wide range of languages including \ili{Afrikaans}, Aleut, \ili{Basque}, \ili{Danish}, \ili{Dutch}, English, \ili{Finnish}, \ili{German}, \ili{Greek}, \ili{North Sámi}, \ili{Norwegian}, \ili{Old Spanish}, and \ili{Swedish}. They all share the desire to better understand the relationship between linear order, syntax and \isi{semantics}, using intricate data from the detailed study of individual languages informed by broader cross-linguistic patterns. Anders Holmberg has been a pioneer of this kind of careful syntactic investigation for the past 30 years, and continues to be so to this day. 

\printbibliography[heading=subbibliography,notkeyword=this]
\end{document}